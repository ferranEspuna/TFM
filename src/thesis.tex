\documentclass[12,twoside]{mammeTFM}
\usepackage{amsthm,amsmath,amssymb,amsfonts,amscd}
\usepackage{algorithm}
\usepackage{algorithmicx}
\usepackage{pdfpages}

% shady stuff
\makeatletter
\AddToHook{env/algorithmic/before}{\def\@currentcounter{ALG@line}}
\makeatother
\usepackage[nameinlink]{cleveref}
\crefalias{ALG@line}{line}
% end of shady stuff

\usepackage{enumerate}
\usepackage[all]{xy}
\usepackage{booktabs}
\usepackage{algpseudocode}
\usepackage{tikz}

%%%%%Author packages if necessary
\usepackage{mathtools}
\usepackage{relsize}
\usetikzlibrary{calc}
% Theorem Environments: add extra ones at the end if you need it.
\newtheorem*{theoremA}{Theorem A}
\newtheorem{theorem}{Theorem}[section]

\newtheorem{proposition}[theorem]{Proposition}
\newtheorem{lemma}[theorem]{Lemma}
\newtheorem{corollary}[theorem]{Corollary}
\newtheorem{conjecture}[theorem]{Conjecture}

\theoremstyle{definition}
\newtheorem{definition}[theorem]{Definition}
\newtheorem{example}[theorem]{Example}


\theoremstyle{remark}
\newtheorem{remark}[theorem]{Remark}
\newtheorem*{remarknonumber}{Remark}
\newtheorem{observation}[theorem]{Observation}

\usepackage{stmaryrd}
%%%%%%%%%%%%%%%%%%
% macros/abbreviations: Include here your own.
%%%%%%%%%%%%%%%%%%
\newcommand{\N}{\ensuremath{\mathbb{N}}}
\newcommand{\ex}[2]{\ensuremath{\text{ex} \left( #1, #2 \right)}}
\newcommand{\compoverset}[2]{\ensuremath{K\left(#2, \overset{#1}{\dots}, #2\right)}}
\newcommand{\compdots}[2]{\ensuremath{K\left(#1, \dots, #2\right)}}
\newcommand{\zaroverset}[3]{\ensuremath{z\left(#2, \overset{#1}{\dots}, #2; #3, \overset{#1}{\dots}, #3\right)}}
\newcommand{\zaroversetdots}[4]{\ensuremath{z\left(#2, \overset{#1}{\dots}, #2; #3, \dots, #4\right)}}
\newcommand{\zardots}[4]{\ensuremath{z\left(#1, \dots, #2; #3, \dots, #4\right)}}
\newcommand{\bigO}[1]{\ensuremath{\mathcal{O}\left(#1\right)}}
\newcommand{\link}[3]{\ensuremath{L_{#1}\left(#2; #3\right)}}
\newcommand{\compdotssuperindex}[3]{\ensuremath{K^{(#1)}\left(#2, \dots, #3\right)}}
\newcommand{\completesuperindex}[2]{\ensuremath{K^{(#1)}_{#2}}}
\newcommand{\compdotssuperindexoverset}[3]{\ensuremath{K^{(#1)}\left(#3, \overset{#2}{\dots}, #3\right)}}

\newenvironment{delayedproof}[1]
 {\begin{proof}[\raisedtarget{#1}Proof of \Cref{#1}]}
 {\end{proof}}
\newcommand{\raisedtarget}[1]{%
  \raisebox{\fontcharht\font`P}[0pt][0pt]{\hypertarget{#1}{}}%
}
\newcommand{\proofref}[1]{\hyperlink{#1}{proof}}
% Body of document

\titol{Finding Partite Hypergraphs Efficiently}
\titolcurt{Finding Partite Hypergraphs Efficiently}
\authorStudent{Ferran Espuña Bertomeu}
\supervisors{Richard Lang}
\monthYear{June 2025}

%\msc[2010]{Primary 55M25, 57P10, Secondary 55P15, 57R19, 57N15.}

\paraulesclau{hypergraph, algorithm, graph, partite, extremal}
\agraiments{
    I would like to extend my gratitude to the many people who have accompanied me throughout my academic and personal journey.
    This list is not exhaustive, but it seeks to acknowledge how fortunate I have been to receive support from so many along the way.

    First and foremost, I would like to thank my parents for their unconditional love, support, and encouragement,
    and for all the gentle nudges away from local optima.
    Your perspective will always be invaluable.

    I am also grateful to my friends from the Cinema Club (Adrià, Nacho, and many others),
    who to this day remind me that being a nerd is actually kind of cool.

    Thanks to teacher Israel, who recognized my love for mathematics early on and always encouraged me to pursue it.

    To my university classmates Miguel, Carlos, Nazar, Gerard, Nacho, and Severino:
    thank you for the endless laughs, the fruitful discussions, the shared struggles,
    and maybe one too many late-night study sessions.
    It really is about the friends we make along the way.

    I would like to thank Doctor Rubén Ballester and Professors Carles Casacuberta and Sergio Escalera
    for guiding my first steps into the world of mathematical research
    and encouraging me to pursue a master's degree in mathematics.

    Thanks as well to Professor Oriol Serra and Doctors Guillem Perarnau, Juanjo Rué, and Patrick Morris
    for introducing me to the fascinating world of discrete mathematics.

    Finally, I am deeply grateful to my supervisor, Doctor Richard Lang,
    for his guidance, support, and patience throughout this thesis.
    I truly felt that you cared not only about the project, but also about my personal and academic growth.
    Thank you for the opportunity to work with you and learn from your expertise.

}

\abstracteng{
    Turán-type problems are a central theme in extremal (hyper)graph theory.
    Given a fixed $k$-graph $G$,
    they ask for the maximum number $\ex{n}{G}$ of edges a $k$-graph on $n$ vertices can have
    without containing $G$ as a subgraph.
    This thesis deals with degenerate Turán-type problems, in which $\ex{n}{G} = o(n^k)$.
    In particular, we focus on $t$-blowups of an edge,
    denoted $\compoverset{k}{t}$.

    Classical existence theorems by Kővari, Sós, and Turán (for $k=2$) and Erdős (for $k \ge 2$)
    guarantee that $k$-graphs with constant edge density contain $\compoverset{k}{t}$ as a subgraph, where $t$
    grows with the number of vertices $n$ (typically $t$ is on the order of $(\log n)^{1/(k-1)}$).
    These proofs are non-constructive,
    and locating such large subgraphs efficiently
    is challenging as brute-force search becomes superpolynomial in $n$.

    This thesis presents a deterministic, polynomial-time algorithm that bridges this gap in the constant density regime.
    Given a $k$-graph with $n$ vertices and $m$ edges, our algorithm finds a $\compoverset{k}{t}$ where $t$
    explicitly depends on $n$, $k$, and the density $d=m/n^k$, matching the best possible order of magnitude.
    Our method generalizes work on the bipartite $2$-graph case by Mubayi and Turán, using a recursive strategy on link graphs.
}

\begin{document}
    
    \includepdf{portada.pdf}

    \maketitle

    \tableofcontents

    \section{Introduction}\label{sec:introduction}

Graph theory provides fundamental tools for modeling relationships and networks across diverse fields.
A natural and powerful extension of graphs is the concept of \emph{hypergraphs},
where edges can connect more than two vertices.
Specifically, a $k$-uniform hypergraph, or $k$-graph,
consists of a set of vertices and a collection of edges, each being a $k$-element subset of the vertices.
For example, a $2$-graph is simply an undirected graph with no loops or parallel edges.
$k$-graphs arise naturally in areas ranging from combinatorics and computer science to data analysis and
computational biology.

A central branch is extremal (hyper)graph theory.
This field seeks to understand the maximum or minimum size of a combinatorial structure satisfying certain properties.
For instance, \emph{Turán-type problems} ask how many edges a $k$-graph can have, as a function of its number of vertices $n$,
without containing a specific subgraph $G$.
The maximum such number of edges is called the \emph{Turán number} of $G$ on $n$ vertices and is denoted by $\ex{n}{G}$.
A key result is Turán's Theorem~\cite{Turan1941},
which determines $\ex{n}{K_r}$ for all $n \geq r \geq 2$, where $K_r$ is the complete graph on $r$ vertices.
Furthermore, the Erdős--Stone--Simonovits Theorem~\cite{erdos1946structure}
asymptotically estimates $\ex{n}{G}$ for any fixed $2$-graph $G$ as $n \to \infty$,
and as a function of the chromatic number of $G$.

We do not yet understand how to extend these theorems to hypergraphs,
as the combinatorial structures become significantly more complex.
The asymptotic behavior of $\ex{n}{G}$ as $n \to \infty$
is characterized by the \emph{Turán density} of $G$, defined as
\[
    \pi(G) = \lim_{n \to \infty} \frac{\ex{n}{G}}{\binom{n}{k}}.
\]
Determining the exact value of $\pi(G)$ for $k$-graphs when $k > 2$
is a notoriously difficult open problem for many families,
including even small hypergraphs like the complete $3$-graph on $4$ vertices,
$K_4^{(3)}$~\cite{keevash2011hypergraph, razborov20103}.
This thesis focuses on the \emph{degenerate} case, where $\pi(G) = 0$.
A fundamental result states that this is the case if and only if $G$ is $k$-partite
(meaning its vertices can be partitioned into $k$ sets such that no edge has two vertices in the same set).
We are particularly interested in the problem of forbidding complete balanced $k$-partite $k$-graphs,
denoted $\compoverset{k}{t}$, which consist of $k$ disjoint sets of $t$ vertices each,
and all $t^k$ edges formed by selecting one vertex from each set.
The classical Kővari--Sós--Turán Theorem~\cite{Kovari1954, Hylten1958}
provides the following upper bound for $k=2$.
\[
    \ex{n}{K(s, t)} = \bigO{n^{2 - \frac{1}{\min\{s, t\}}}}.
\]
Erdős~\cite{Erods1964} found an analogous bound for complete balanced $k$-partite $k$-graphs for $k \ge 2$,
showing that
\begin{equation} \label{eq:erdos64-intro}
    \ex{n}{\compoverset{k}{t}} = \bigO{n^{k - \frac{1}{t^{(k-1)}}}}.
\end{equation}

Upper bounds for Turán numbers, like~\eqref{eq:erdos64-intro},
often involve counting or probabilistic arguments,
which are inherently non-constructive.
They guarantee the existence of the desired subgraph $\compoverset{k}{t}$
in dense enough hypergraphs but do not typically provide an efficient algorithm to \emph{find} such a subgraph.
If we focus on a fixed guest $k$-graph $G = (V, E)$ and let the number $n$ of vertices of the host $k$-graph $H$ grow,
this is not considered a problem,
as a brute-force search over all ordered sets of $|V|$ vertices in $H$ yields a polynomial-time algorithm
for finding a copy of $G$ in $H$.
However, the situation becomes more complex when we consider the case where $G$ is not fixed, but rather grows with $n$.
We focus on the case where the edge density of $H$ is fixed (that is, $H$ has at least $\epsilon \binom{n}{k}$ edges),
and $G = \compoverset{k}{t}$ for some $t$ that can grow with $n$.
Careful analysis of the proof of Erdős' bound shows that this guarantees that $G$ is a subgraph of $H$ for some
\begin{equation} \label{eq:t-lower-intro}
    t = \delta(\epsilon) (\log n)^{\frac{1}{k-1}}.
\end{equation}
Running a brute-force search checking all $\binom{n}{kt}$ sets of $kt$
vertices in $H$ would then \textbf{not} yield a polynomial-time algorithm,
because $\binom{n}{kt}$ grows superpolynomially with $n$.

The main contribution of this thesis is bridging this gap by providing an efficient algorithmic solution.
We develop and analyze a deterministic, polynomial-time algorithm that,
given a $k$-graph $G$ with $n$ vertices and at least $dn^k$ edges,
finds a complete balanced $k$-partite subgraph $\compoverset{k}{t}$ within $G$, where
\[
    \left\lfloor \left(  \frac{\log \left(n/2^{(k-1)}\right)}{\log (3/d)} \right)^{\frac{1}{k-1}} \right\rfloor,
\]
matching the order of magnitude of~\eqref{eq:t-lower-intro}.
This result not only provides a constructive proof for the upper bounds of the type established by Erdős,
but in fact reaches the best possible value of $t$ up to a constant factor depending on $d$, as
can be shown by probabilistic arguments (see~\Cref{prop:probabilistic-lower-bound} and
the beginning of \Cref{sec:algorithm}, where the algorithm is introduced).
Our algorithm generalizes the approach used by Mubayi and Turán for the bipartite case ($k=2$)~\cite{MUBAYI2010174}.
It employs a recursive strategy that mirrors the inductive proof structure of Erdős' bound,
iteratively reducing the uniformity $k$ by constructing appropriate link graphs.

\textbf{Organization of the Thesis:}
\Cref{sec:preliminaries} formally introduces some families of hypergraphs
(including complete $k$-partite hypergraphs),
as well as basic operations like restrictions, links, and blow-ups.
We also use this section to introduce asymptotic notation, which is used throughout the thesis.
\Cref{sec:extremal} provides an overview of relevant theoretical results for Turán-type problems,
proving central theorems like the Turán Theorem (\Cref{thm:turan}), the Kővari--Sós--Turán Theorem (\Cref{thm:kst}),
Erdős' bound (\Cref{thm:erdos64}) and a more precise version of it (\Cref{thm:erdos64-constant-density}),
thus showing that finding a complete balanced $k$-partite $k$-graph of part sizes in the order of~\eqref{eq:t-lower-intro}
exists in $H$ when it has constant positive density.
\Cref{sec:algorithm} presents our main algorithm (\Cref{alg:kpartite}),
provides a rigorous proof of its correctness and analyzes its polynomial runtime complexity (\Cref{thm:kpartite}).
Finally, \Cref{sec:conclusions} summarizes the main results of this thesis, and discusses some open problems for future research.

    \section{Preliminaries}\label{sec:preliminaries}
In this section we introduce some basic definitions and results that will be used throughout the thesis.

% todo change embeddings from f to \varphi
\begin{definition}

    For an integer $k \geq 2$ a finite \emph{$k$-graph}
    is a tuple $G = (V, E)$ where $V$ is a finite set
    and $E \subseteq \binom{V}{k}$.
    We call the elements of $V \eqqcolon V(G)$ its \emph{vertices}
    and those of $E \eqqcolon E(G)$ its \emph{edges}.
\end{definition}

\begin{remark}
    If we let $k=2$ we recover the usual definition of a graph.
\end{remark}

% TODO define degrees, complete graph, etc.

\begin{definition}
    Let $G = (V, E)$ and $H = (W, F)$ be $k$-graphs.
    A \emph{homomorphism} from $G$ to $H$ is a map $f: V \to W$
    such that for every edge $e \in E$ the set $f(e) \coloneqq \{f(v) \mid v \in e\}$
    is an edge in $H$ (that is, $f(e) \in F$). If such a homomorphism exists
    and is injective, we say that $f$ is an embedding of $G$ on $H$
    and that $H$ contains $G$ as a subgraph.
    If, furthermore, $f^{-1}: \text{Im}(f) \to V$ is a homomorphism, we say that $f$
    is an \emph{induced} embedding and that $H$ contains $G$ as an \emph{induced}
    subgraph.
    We write $G \subseteq H$.
    If, in addition, $f$ is a bijection, we say that $f$ is an \emph{isomorphism}
    and that $G$ is \emph{isomorphic} to $H$.
    We write $G \cong H$.
\end{definition}

\begin{remark} % TODO maybe turn into proposition + proof?

    It is elementary to check that
    (induced) inclusion is an order relation and that
    isomorphism is an equivalence relation.
    Furthermore, isomorphism preserves (induced) inclusion.
    Therefore, we can talk about the (induced) subgraph
    condition up to isomorphism, both in the \emph{host} $k$-graph
    ($H$) and in the \emph{guest} $k$-graph ($G$).
\end{remark}

\begin{remark} \label{rem:change_vertices}
    Given a $k$-graph $G = (V, E)$ and a set $W$ satisfying $|V| = |W|$,
    we can define an edge set $E'$ on $W$ such that $G \cong (W, E')$
    by taking any bijection $f: V \to W$ and setting $E' = \{f(e) \mid e \in E\}$.
    This frees us, up to isomorphism, to change or reorder
    the vertices of a $k$-graph.
\end{remark}

\begin{proposition} \label{prop:extremal}
    Let $G = (V, E)$ be a $k$-graph with nonempty edge set and $n \geq |V|$ be an integer.
    Then there exists an integer $M_0 = ex(n, G) \in \left[ 0, \binom{n}{k}\right)$ such that
    the condition
    \[
        \text{``All $k$-graphs with $n$ vertices and $m$ edges contain $G$ as a subgraph''}
    \]
    is true for all $\binom{n}{k} \geq m > M_0$ and false for all $0 \leq m \leq M_0$.

    \begin{proof}
        Note that, if $M_0$ exists, clearly it is unique.
        Also, the condition is clearly false for $m = 0$ and
        true for $m = \binom{n}{k}$
        (the only graph $H$ with vertex set $W$, $|W|=n$ and $\binom{|V|}{k}$ vertices
        is the one having all $k$-sets of vertices so any injective map $f: V \to W$
        is an embedding of $G$ in $H$).
        We only need to show that if the condition is true for $m$ then it is true for
        all $m' \geq m$.
        Suppose it is true for $m$ and let $m' \geq m$.
        Let $H = (W, F)$ be a $k$-graph with $n$ vertices and $m'$ edges.
        We can just take $F' \subseteq F$ with $|F'| = m$.
        By hypothesis, the graph $H' = (W, F')$ contains $G$ as a subgraph,
        and the identity map in $W$ is an embedding of $H'$ in $H$:
        \[
            G \subseteq H' \subseteq H \implies G \subseteq H \qedhere
        \]
    \end{proof}

\end{proposition}

\begin{remark}
    We call $ex(n, G)$ the \emph{extremal number} of $G$.
    It is clearly invariant under isomorphism.
\end{remark}

\begin{definition}
    for an integer $p \geq k$, a $k$-graph $G = (V, E)$ is \emph{$p$-partite}
    if there exists a partition $V = V_1 \cup \dots \cup V_p$
    such that every edge $e \in E$ intersects every part $V_i$ in at most one vertex.
    We may write $G = (V_1, \dots, V_p; E)$ and say that
    $G$ is a partite $k$-graph on $V_1, \dots, V_p$.
\end{definition}

\begin{remark}
    If $p=k$, every edge intersects every part in exactly one vertex,
    so we can identify the edges with a subset of $ V_1 \times \dots \times V_k$.
\end{remark}

\begin{definition}
    A $k$-partite $k$-graph $G = (V_1, \dots, V_k; E)$ is \emph{complete}
    if every $k$-set of vertices $(v_1, \dots, v_k)$ with $v_i \in V_i$
    satisfies $\{v_1, \dots, v_k\} \in E$.
    We write $G = K(V_1, \dots, V_k)$.
\end{definition}

\begin{remark}
    $V_1, \dots, V_k, W_1, \dots, W_k$ are disjoint sets,
    and $|V_i| = |W_i| \eqqcolon a_i$ for all $i$ then it is elementary to check that
    \[
        K(V_1, \dots, V_k) \cong K(W_1, \dots, W_k)
    \]
    by a construction very similar to the one in Remark~\ref{rem:change_vertices}.
    This allows us to talk about \emph{the} complete $k$-partite $k$-graph on
    $a_1, \dots, a_k$ vertices, which we denote by $K(a_1, \dots, a_k)$.
\end{remark}

\begin{remark}
    All $k$-partite $k$-graphs with part sizes $b_1 \leq a_1, \dots, b_k \leq a_k$
    are contained in $K(a_1, \dots, a_k)$ as subgraphs.
    This lets us follow the exact same argument as in Proposition~\ref{prop:extremal}
    to define the following:
\end{remark}

\begin{definition}\label{def:zarankiewicz}
    let $0 < t_1 \leq v_1, \dots, 0 < t_k \leq v_k$ be integers.
    Then the \emph{generalized Zarankiewicz number} $z(v_1, \dots, v_k; t_1, \dots, t_k)$
    is the largest integer $0 \leq z < v_1  \dots v_k$ for which there exists $k$-partite $k$-graph
    $H$ with part sizes $ |V_1| = v_1, \dots, |V_k| = v_k$ and $z$ edges
    such that no embedding $f$ of $K(T_1, \dots, T_k)$ with $|T_i| = t_i$ in it exists
    satisfying $f(T_i) \subseteq V_i$ for all $i$.
\end{definition}

From now on, every time we talk about embeddings from one $k$-partite $k$-graph
onto another we will assume that the the condition $f(T_i) \subset V_i$.

\begin{remark}\label{rem:zar_vs_turan}
    Finding this number can help us upper bound the extremal number of $K(t_1, \dots, t_k)$ asymptotically:
    Assume that G is a $K(t_1, \dots, t_k)$-free $n$-vertex $k$-graph with $m$ edges.
    pick $v_1, \dots, v_k$ such that $\sum_{i} v_i = n $ and $v_i \sim n/k $
    (For example $\lfloor n/k \rfloor \leq v_i \leq \lceil n/k \rceil$)
    Let $V_1, \dots, V_k$ be a random partition of $V(G)$ with $|V_i| = n_i$.
    for an edge $e \in E(G)$, the probability that $e$ is an edge in $K(V_1, \dots, V_k)$ is
    greater than
    \[k! \prod_i n_i \sim k! (1/k)^k\]
    which is independent of $n$.
    Therefore, the expected number of edges satisfying this condition is a positive fraction of $m$.
    Applying the first moment method, we can conclude that
    \[ex(n, K(t_1, \dots, t_k)) = O(z(\lceil n / k \rceil , \overset{k}{\cdots}, \lceil n / k \rceil; t_1, \dots, t_k))\]

\end{remark}


The problem on finding the Zarankiewicz number was first posed by K. Zarankiewicz in 1951 for the
case of bipartite 2-graphs (that is, finding $z(u, w; s, t)$),
in terms of finding all-1 minors in a $0-1$ matrix.
An upper bound for it in the case $m=n, s=t$ was found by Kővari, Sós and Turán in~\cite{Kovari1954} in 1954.
This was generalized to arbitrary complete
partite 2-graphs by C. Hyltén-Cavallius in~\cite{Hylten1958}
in 1958.
The result is stated and proved here for completeness:

% m, n --> u, w
% TODO: replace \leq with < ??
\begin{theorem}\label{thm:kst}
    Let $0 < m \leq s$ and $0 < n \leq t$ be integers. 
    Then 
    \[z(u, w; s, t) \leq (s - 1)^{1 / t}(w - t + 1)u^{1 - 1 / t} + (t - 1)u\]
    \begin{proof}
        Suppose that we have a bipartite graph $G = (U, W; E)$
        with $|U| = u$, $|W| = w$ and $|E| = z$ exceeding the bound stated above.
        Let us consider the set
        \[
            P = \left\{ (x, Y) \in U \times \binom{W}{t}
            \Big| \forall y \in Y: \{x, y\} \in E \right\}
        \]
        Counting on the first coordinate, and using Jensen's inequality, we get
        \[
            |P|
            = \nsum_{x \in U} \binom{d_G(x)}{t}
            = \nsum_{x \in U} f(d_G(x))
            \geq u  \nsum_{x \in U} f(z/u)
            = u \binom{z / u}{t}
        \]
        Where we define
        \[
            f(x) \coloneqq
            \begin{cases}
                \binom{x}{t}, & \text{if } x \geq t - 1 \\
                0, & \text{otherwise}
            \end{cases}
        \]
        Which is convex, meaning we get the inequality as Jensen's inequality.
        The other equalities come from the fact that $f(d)$ agrees
        with $\binom{d}{t}$ for all integers $d \geq 0$;
        and that by our bound on $z$, $z \geq (t-1)u \implies z/u \geq t - 1$.

        If we had $s$ different elements of $P$ with the same second coordinate $T$,
        they would all necessarily have different first coordinates
        (say $S = \{x_1, \dots, x_s\}$).
        But now, by definition of $P$, for all $a \in S, b \in T$, we have $\{a, b\} \in E$.
        This would mean that the inclusion map from $ S \cup T$ to $U \cup W$ is an embedding of
        $K(s, t)$ in $G$, as described in Definition~\ref{def:zarankiewicz}.
        Supposing that this is not the case, by the pigeonhole principle, we have:
        \[
            |P| \leq (s - 1) \binom{w}{t}
        \]
        Putting the two inequalities together, we get:
        \[
            u \binom{z / u}{t} \leq (s - 1) \binom{w}{t}
        \]
        Now, because we can see $E$ as a subset of $U \times W$,
        we get $z \leq uw \implies z/u \leq w$.
        In particular, we have:
        \[
            \frac{(z/u - (t - 1))^t}{\binom{z/u}{t}} \leq \frac{(w - (t - 1))^t}{\binom{w}{t}}
        \]
        which is true for each factor when expanding the denominators.
        Multiplying the two inequalities, we get:
        \[
            u \, (z/u - (t - 1))^t \leq (s - 1)(w - (t - 1))^t
        \]
        which, by algebraic manipulation, gives
        \[
            z \leq (s - 1)^{1 / t}(w - t + 1)u^{1 - 1 / t} + (t - 1)u
        \]
        In contradiction with our assumption. \qedhere


    \end{proof}

\end{theorem}

\begin{remark}
    Following Remark~\ref{rem:zar_vs_turan}, we can use this bound to get an upper bound on the extremal number of $K(s, t)$:
    \[
        ex(n, K(s, t)) = O\left((s - 1)^{1 / t}(n - t + 1)n^{1 - 1 / t} + (t - 1)n\right) = O\left(n^{2 - 1 / t}\right)
    \]
    Note that if $s < t$, we get the better bound $O\left(n^{2 - 1 / s}\right)$ by interchanging the roles of $s$ and $t$.
\end{remark}

In 1964, Erdős~\cite{Erods1964} generalized this result to arbitrary complete partite $k$-graphs in the following theorem:

\begin{theorem}\label{thm:erdos64}
    ex$(n, K(t, \overset{k}{\cdots}, t)) = O\left(n^{k - \frac{1}{t^{k-1}}}\right)$
\end{theorem}

A more modern proof of this result can be found in~\cite{carvajal2024canonical},
which also generalizes it to arbitrary complete
$k$-partite $k$-graphs (not necessarily with equal part sizes).
They in fact prove a bound for the generalized Zarankiewicz number
in a similar way we proved the bound for the Zarankiewicz number in Theorem~\ref{thm:kst},
which then following Remark~\ref{rem:zar_vs_turan} gives the result in Theorem~\ref{thm:erdos64}.




    \section{Hypergraph Turán Problems}\label{sec:extremal}

\subsection{Turán-Type Problems}\label{subsec:turan}

Now we can state the \emph{forbidden subgraph problem} for $k$-graphs.
Informally, given a $k$-graph $G$, and an integer $n \geq |V(G)|$,
we want to find the smallest $M_0$ such that all $k$-graphs with $n$ vertices and $m > M_0$ edges
contain $G$ as a subgraph.

\begin{proposition} \label{prop:extremal}
    Let $G = (V, E)$ be a $k$-graph with nonempty edge set and $n \geq |V|$ be an integer.
    Then there exists an integer $M_0 = \ex{n}{G} \in \left[ 0, \binom{n}{k}\right)$ such that
    the condition
    \[
        \text{``All $k$-graphs with $n$ vertices and $m$ edges contain $G$ as a subgraph.''}
    \]
    is true for all $\binom{n}{k} \geq m > M_0$ and false for all $0 \leq m \leq M_0$.

    \begin{proof}
        Note that, if $M_0$ exists, clearly it is unique.
        Also, the condition is clearly false for $m = 0$ and
        true for $m = \binom{n}{k}$
        (the only graph $H$ with vertex set $W$, $|W|=n$ and $\binom{n}{k}$ edges
        is the one having all $k$-sets of vertices so any injective map $f: V \to W$
        is an embedding of $G$ in $H$).
        We only need to show that if the condition is true for $m$ then it is true for
        all $m' \geq m$.
        Suppose it is true for $m$ and let $m' \geq m$.
        Let $H = (W, F)$ be a $k$-graph with $n$ vertices and $m'$ edges.
        We can take $F' \subset F$ with $|F'| = m$.
        By hypothesis, the graph $H' = (W, F')$ contains $G$ as a subgraph,
        and the identity map in $W$ is an embedding of $H'$ in $H$.
        Then, $G \subset H' \subset H$ implies $G \subset H$ by transitivity of the embedding
        relation (Proposition~\ref{prop:embedding_properties}).

    \end{proof}

\end{proposition}

We call the integer $\ex{n}{G}$ the \emph{Turán number} of $G$ on $n$ vertices.
The Turán number is increasing both in $n$ and under graph inclusion.
The first property can be seen by taking a $G$-free $k$-graph on $H$ with $n$ vertices
and $\ex{n}{G}$ vertices; and adding a vertex $v$ and no edges to it, obtaining $H'$.
If $f: G \to H$ is an embedding and $v$ has a preimage $x$, it must be a vertex in $H$ with degree $0$,
so it can be replaced by any other vertex in $H$ outside the image of $f$.
Restricting this new mapping to $H$, we get that $G \subset H$, against our assumption.
For the second, suppose that $G \subset G'$.
Because $H$ is $G$-free, it is also $G'$-free, which means that it has at most $\ex{n}{G'}$ edges.
Therefore, $\ex{n}{G} \leq \ex{n}{G'}$.
As a consequence of this, we also get that the Turán number is invariant under isomorphism of $G$.

There are very few $k$-graphs $G$ for which an exact formula for $\ex{n}{G}$ is known.
Of these, the most famous family of examples are the complete $2$-graphs $\completesuperindex{2}{r}$,
for which Turán numbers were first studied by Turán~\cite{Turan1941} in 1941.
The result is the following.

\begin{theorem}[Turán's Theorem]
    \label{thm:turan}
    Let $r > 2$ be an integer and let $n \geq r$.
    Let $a_1, \dots, a_{r-1}$ be integers such that $a_1 + \dots + a_{r-1} = n$
    and $\lfloor n / (r-1) \rfloor \leq a_i \leq \lceil n / (r-1) \rceil$ for all $i$.
    Then
    \begin{equation} \label{eq:turan}
        \ex{n}{\completesuperindex{2}{r}} = \sum_{\{x, y\} \in \binom{[r-1]}{2}} a_x \cdot a_y
    \end{equation}
    Furthermore, if $G$ is a $2$-graph with $\ex{n}{\completesuperindex{2}{r}}$ edges
    and $G$ does not contain $K_r^{(2)}$ as a subgraph, then
    \[
        G\cong \compdotssuperindex{2}{a_1}{a_{r-1}}.
    \]

\end{theorem}

Before the \proofref{thm:turan} of Turán's theorem, we introduce two lemmas.

\begin{lemma}\label{lem:same_degree}
    Let $G = (V, E)$ be a $2$-graph with $n$ vertices and
    $\ex{n}{\completesuperindex{2}{r}}$ edges.
    If $x, y \in V$ are different vertices and $\{x, y\} \notin E$, then $d_G(x) = d_G(y)$.
    \begin{proof}
        We argue by contradiction.
        Suppose, without loss of generality, that $d_G(x) > d_G(y)$.
        We argue that we can construct a $2$-graph $G'$ with $n$ vertices
        and more edges than $G$ that does not contain $K_r^{(2)}$ as a subgraph, against the definition of
        the Turán number.

        The new graph $G' = (E', V')$ is constructed from $G$ by removing from $V$ the vertex $y$
        (and all edges containing it)
        and adding a copy $x'$ of $x$, connected to the same vertices (that is, $\{x', v\} \in E'$
        if and only if $\{x, v\} \in E$).
        Clearly, $|V'| = |V|$ and $|E'| = E - d_G(y) + d_G(x) > |E|$.
        To see that $G'$ does not contain $K_r^{(2)}$ as a subgraph,
        suppose that $G'[T']$ is complete for some $T' \subset V'$ of size $r$.
        Because $\{x, x'\}$ is not an edge in $G'$, $T'$ cannot contain both $x$ and $x'$.
        Because the edges not containing $x'$ are the same as in $G$, which contains no $K_r^{(2)}$,
        we deduce that $T'$ contains $x'$ and therefore does not contain $x$.
        Now, let $T = (T' \setminus \{x'\}) \cup \{x\} \subset V$, also of size $r$.
        We argue that the graph $G[T] = G'[T]$ must be complete, reaching a contradiction.
        If it were not, then there would exist $v \in T, v \neq x$, such that $\{x, v\} \notin E$.
        This implies that $\{x', v\} \notin E'$, but $v \in T' \setminus \{x'\}$,
        against the completeness of $G'[T']$.
    \end{proof}
\end{lemma}

\begin{lemma} \label{lem:turan_complete_partite}
    Let $G = (V, E)$ be a $2$-graph with $n$ vertices and
    $\ex{n}{\completesuperindex{2}{r}}$ edges.
    Then, $G$ is a complete $p$-partite graph for some $p \geq 2$.
    \begin{proof}
        Equivalently, we show that the relation defined by non-adjacency on $V$ (that is, $x \sim y$ when
        $\{x, y\} \notin E$) is an equivalence relation, so we can divide $V$ into equivalence classes
        by this relation, which means that $\{x, y\} \in E$ if and only if they are in different parts.

        The reflexivity and symmetry of the relation are clear.
        Suppose, by way of contradiction, that there exist $x, y, z \in V$ such that
        $x \sim z$ and $y \sim z$, but $x \nsim y$.
        We now construct a different graph $G'$ with the same number of vertices as $G$
        that also does not contain $K_r^{(2)}$ as a subgraph, reaching a contradiction.
        $G'$ is constructed from $G$ by removing the vertices
        $x$ and $y$ (and all the associated edges) and adding the two new vertices
        $z_1$ and $z_2$ and the edges $\{\{v, z_i\} | \{v, z\} \in E, i \in \{1, 2\}\}$.

        First, we show that $G'$ contains no embedding of $K_r^{(2)}$.
        We make a similar argument as in the proof of Lemma~\ref{lem:same_degree}.
        By way of contradiction, suppose that $G'[T']$ is complete for some $T' \subset V'$ of size $r$.
        Because $z$, $z_1$ and $z_2$ pairwise non-edges of $G'$, only one of them can be
        an image of a vertex in $K_r^{(2)}$.
        However, $G'[V \setminus \{x, y\}] \subset G$ has no embedding of $K_r^{(2)}$,
        so at least one of the vertices in $K_r^{(2)}$ must be mapped to $z_1$ or $z_2$.
        Without loss of generality, we can write $T' = \{x_1, x_2, x_3, \dots, x_{r-1}, z_1\}$,
        with $x_i \notin \{z_2, z\}$ for all $i$.
        However, $\{z_1, x_i\}$ is an edge in $G'$ if an only if $\{z, x_i\}$ is an edge in $G$,
        which means that $G'[\{x_1, x_2, x_3 \dots, x_{r-1}, z\}] = G[\{x_1, x_2, x_3 \dots, x_{r-1}, z\}]$ is complete,
        against our assumption.

        Now, we show that $G'$ has more edges than $G$.
        By Lemma~\ref{lem:same_degree}, $d_G(x) = d_G(z)$ and $d_G(y) = d_G(z)$,
        so the three vertices $x, y, z$ have the same degree $d$ in $G$.
        The edges containing $x$ and the edges containing $y$ intersect at exactly the edge $\{x, y\} \in E$.
        Therefore, by removing all of them from $G$ we are removing $2d - 1$ edges.
        Furthermore, for each edge containing $z$ we are adding two edges,
        and these sets of edges do not intersect because $z$ is not adjacent to $x$ or $y$ (so $\{z_1, z_2\} \notin E'$).
        We conclude that $G'$ has $|E'| = |E| - (2d - 1) + 2d = |E| + 1 > |E| $ edges, as desired.
    \end{proof}
\end{lemma}

Now we are ready to prove Turán's theorem.
\begin{delayedproof}{thm:turan}
    We have shown in Lemma~\ref{lem:turan_complete_partite} that $G = (V_1, \dots V_p; E)$ is complete.
    In fact, we can set $p = r - 1$:
    If $p < r - 1$, we can always add empty parts to $G$; and if it has more than $r - 1$ nonempty parts
    (without loss of generality, $x_1 \in V_1, \dots, x_r \in V_r$), then $G[\{x_1, \dots, x_r\}]$ is complete,
    which is a contradiction.
    Furthermore, any complete $(r-1)$-partite $2$-graph is $\completesuperindex{2}{r}$-free,
    because $\completesuperindex{2}{r}$ is not $(r-1)$-partite.

    This means that we only need to show that the choice of the part sizes $a_1, \dots, a_{r-1}$ summing to $n$
    in the statement maximizes the expression~\eqref{eq:turan}.
    The condition that $\lfloor n / (r-1) \rfloor \leq a_i \leq \lceil n / (r-1) \rceil$ for all $i$
    is equivalent to requiring that the part sizes are as equal as possible, that is,
    $|a_i - a_j| \leq 1$ for all $i, j$.
    Suppose, by way of contradiction and without loss of generality, that $a_1 > a_2 + 1$.
    Let $a_1' = a_1 - 1$, $a_2' = a_2 + 1$ and $a_i' = a_i$ for all $i \geq 3$.
    Then,
    \begin{align*}
        \sum_{\{x, y\} \in \binom{[r-1]}{2}} a_x' \cdot a_y'
        =& \, (a_1 - 1)(a_2 + 1) + (a_1 - 1) \sum_{i \geq 3} a_i + (a_2 + 1) \sum_{i \geq 3} a_i + \sum_{3 \leq x < y} a_x a_y \\
        =& \sum_{\{x, y\} \in \binom{[r-1]}{2}} a_x \cdot a_y - a_2 + a_1 - 1
        > \sum_{\{x, y\} \in \binom{[r-1]}{2}} a_x \cdot a_y,
    \end{align*}
    in contradiction to the number of edges in $G$ being maximal.
\end{delayedproof}

Because of the difficulty of finding exact Turán numbers for $k$-graphs,
we usually look for asymptotic approximations of them.
In particular, we are interested in how the expression
$\ex{n}{G}$ grows with $n$ for any fixed $k$-graph $G$.
This is known as the \emph{Turán problem} for the graph $G$.
For an example, we turn to the complete $2$-graph $\completesuperindex{2}{r}$,
for which we already have an exact formula.
In expression~\eqref{eq:turan}, we can see that
$a_i = n/(r-1) + \bigO{1}$ for all $i$.
Therefore,
\begin{equation} \label{eq:turan_asymptotic}
    \ex{n}{\completesuperindex{2}{r}} = \sum_{\{x, y\} \in \binom{[r-1]}{2}} a_x \cdot a_y
    = \binom{r-1}{2} \cdot \left( \frac{n}{r-1} + \mathcal{O}(1) \right)^2
    = \frac{(r-2)}{2(r-1)} n^2 + \mathcal{O}(n).
\end{equation}
Note that the maximum number of edges in a $2$-graph on $n$ vertices is
\[
    \binom{n}{2} = \frac{1}{2} n^2 + \mathcal{O}(n).
\]
The two quantities are comparable as they are both quadratic in $n$.
This allows us to restate equation~\eqref{eq:turan_asymptotic} as
\begin{equation} \label{eq:turan_asymptotic_density}
    \ex{n}{\completesuperindex{2}{r}} =
    \frac{r-2}{r-1} \binom{n}{2} + \mathcal{O}(n) =
    (1 + o(1))\binom{n}{2},
\end{equation}
which means that, asymptotically, the maximum \emph{edge density} of a $2$-graph on $n$ vertices
without $K_r^{(2)}$ as a subgraph is $(r-2)/(r-1) < 1$, so we must exclude a nontrivial fraction of edges
to avoid any particular complete $2$-graph.
The following general theorem greatly restricts the growth of Turán numbers
for all $k$-graphs.

\begin{theorem}
    Let $G = (V, E)$ be a $k$-graph.
    The limit
    \begin{equation} \label{eq:turan_density}
        \pi(G) = \lim_{n \to \infty} \frac{\ex{n}{G}}{\binom{n}{k}}
    \end{equation}
    exists and is between $0$ and $1$.
    It is called it the \emph{Turán density} of $G$.
    \begin{proof}
        The sequence
        \[
            a_n = \frac{\ex{n}{G}}{\binom{n}{k}}
        \]
        Is bounded between $0$ and $1$ for all $n \geq |V(G)|$, by Proposition~\ref{prop:extremal}.
        Furthermore, it is less than $1$ for all $n \geq |V(G)| + 1$.
        To see this, consider a graph $H = (W, F)$ with $n$ vertices and $\binom{n}{k} - 1$ edges.
        Its edge density is less than $1$.
        Without loss of generality, we can suppose that $F = \binom{W}{k} \setminus \{\{x, y\}\}$.
        This means that $H[W \setminus \{x\}]$ is a complete $k$-graph on $n - 1$ vertices,
        which must contain $G$ as a subgraph.

        We show that the sequence $(a_n)$ is non-increasing, so it must converge to a value $0 \leq \pi(G) < 1$.
        Let $n \geq |V(G)|$.
        There exists a graph $H = (W, F)$ with $n+1$ vertices and $\ex{n+1}{G}$ edges that does not contain
        $G$ as a subgraph.
        For each vertex $w \in W$, the graph $H_w = H[W \setminus \{w\}]$ has $n$ vertices
        and does not contain $G$ as a subgraph.
        Therefore, it must contain at most $\ex{n}{G}$ edges.
        Consider the set
        \[
            \mathcal{P} = \left\{ (w, e) \in W \times F \mid e \in E(H_w) \right\}.
        \]
        Counting on the first coordinate, we get
        \begin{equation} \label{eq:densityUpperBound}
            |\mathcal{P}| = \sum_{w \in W} |E(H_w)| \leq (n+1)\, \ex{n}{G}.
        \end{equation}
        On the other hand, for every edge $e \in F$, $e \in E(H_w)$
        for all $w \in W \setminus e$.
        Therefore, counting on the second coordinate, we get
        \begin{equation} \label{eq:densityLowerBound}
            |\mathcal{P}| = (n + 1 - k) |F| = (n + 1 - k)\, \ex{n+1}{G}.
        \end{equation}
        Combining equations~\eqref{eq:densityUpperBound} and~\eqref{eq:densityLowerBound},
        we get
        \[
            (n + 1 - k)\, \ex{n+1}{G} \leq (n + 1)\, \ex{n}{G}.
        \]
        Going back to the sequence $a_n$, we can write
        \[
            a_{n+1} = \frac{\ex{n+1}{G}}{\binom{n+1}{k}} \leq
            \frac{(n + 1)\, \ex{n}{G}}{(n + 1 - k) \binom{n+1}{k}} =
            \frac{\ex{n}{G}}{\binom{n}{k}} = a_n. \qedhere
        \]
    \end{proof}
\end{theorem}

We can now summarize expression~\eqref{eq:turan_asymptotic_density} as follows.
\begin{corollary} \label{cor:turan_density_kr}
    The Turán density of the complete $2$-graph $\completesuperindex{2}{r}$ is
    \[
        \pi\left(\completesuperindex{2}{r}\right) = \frac{r-2}{r-1} = 1 - \frac{1}{r-1}.
    \]
\end{corollary}

The first natural question that arises is for what graphs $G$ the Turán density $\pi(G)$ is positive
(in which case, we call the corresponding Turán problem \emph{non-degenerate}
and consider it solved if we can calculate $\pi(G)$).
The following gives a complete characterization.

\begin{proposition} \label{prop:degenerate}
    Let $G = (V, E)$ be a $k$-graph.
    Then $\pi(G) = 0$ if and only if $G$ is $k$-partite.
    \begin{proof}
        If $G$ is not $k$-partite, a construction similar to the one in the proof of Theorem~\ref{thm:turan}
        directly shows $\pi (G) > 0$.
        Indeed, for all $m$ the graph $\compoverset{k}{m}$ is $k$-partite so it cannot contain $G$ as a subgraph.
        Furthermore, its edge density is
        \[
             \frac{m^k}{\binom{km}{k}} \geq \frac{1}{k^k}.
        \]
        Because we can make $n = km = |V\left( \compoverset{k}{m} \right)|$ as large as we want,
        the limit~\eqref{eq:turan_density} bounded below by a positive constant.
        We defer the proof of the other direction to subsection~\ref{subsec:degenerate},
        where we study $k$-partite $k$-graph Turán problems
        in more depth (in particular, see Theorem~\ref{thm:erdos64}).
    \end{proof}
\end{proposition}

In fact, non-degenerate Turán problems for $2$-graphs are considered solved in this regard.
The following theorem gives the Turán density of all $2$-graphs.

\begin{theorem}[Erdős-Stone-Simonovits Theorem]
    \label{thm:erdos_stone_simonovits}
    Let $G = (V, E)$ be a $2$-graph and let $r = \chi(G)$.
    Then,
    \[
        \pi(G) = 1 - \frac{1}{r - 1}.
    \]
\end{theorem}

We defer the \proofref{thm:erdos_stone_simonovits} of this theorem to the next subsection,
where we will have more powerful tools at our disposal.
Note that letting $G = \completesuperindex{2}{r}$ we recover
Corollary~\ref{cor:turan_density_kr} of Theorem~\ref{thm:turan}.

Determining the Turán density of $k$-graphs for $k > 2$ is a much harder problem.
Famously, not even the Turán density of the tetrahedron $3$-graph $\completesuperindex{3}{4}$
(pictured in Figure~\ref{fig:complete_kgraph}) or the unique graph $\completesuperindex{3}{4-}$
obtained by removing one edge from it is known.
The best known bounds are
\[
    0.5555 = \frac{5}{9} \leq \pi\left(\completesuperindex{3}{4}\right) \leq 0.561666~\cite{keevash2011hypergraph,baber2011hypergraphs}
\]
and
\[
     0.2857 = \frac{2}{7} \leq \pi\left(\completesuperindex{3}{4-}\right) \leq 0.2871~\cite{frankl1984exact, baber2011hypergraphs}.
\]
The lower bounds are obtained by explicit constructions of $3$-graphs, and are believed to be optimal~\cite{keevash2011hypergraph},
while the upper bounds are obtained by the method of flag algebras~\cite{razborov2007flag},
which is a powerful tool for studying Turán problems that automates the search for relevant inequalities.

One of the few examples of success in obtaining Turán densities of $k$-graphs with uniformity $k > 2$ is the case of
the Fano plane $F^{(3)}_7$, a $3$-graph with $7$ vertices corresponding to the points
of the projective plane over the field $\mathbb{F}_2$,
and $7$ edges corresponding to the projective lines.
It is known that
\[
    \pi\left(F^{(3)}_7\right) = \frac{3}{4}~\cite{de2000maximum}.
\]

We know that $k$-graphs asymptotically below the Turán number of a $k$-graph $G$
may not contain $G$ as a subgraph.
We may also ask how many copies (different embeddings) of $G$ can be found in a $k$-graph $H$
exceeding the Turán density.
The following surprising result~\cite{erdHos1983supersaturated} shows that the number of copies of $G$ in $H$
is asymptotically guaranteed to be very large.

\begin{theorem}[Supersaturation \cite{erdHos1983supersaturated}] \label{thm:supersaturation}
    Let $G = (V, E)$ be a $k$-graph, and let $H_n = (W, F)$ be a $k$-graph with $n$ vertices
    and $\left(\pi(G) + \Omega(1) \right) \binom{n}{k}$ edges.
    Then, $H_n$ has $\Omega \left(n^{|V|} \right)$ copies of $G$ as a subgraph.
    \begin{proof}
        We follow the proof of a survey by Keevash~\cite{keevash2011hypergraph}.
        Suppose that, for $n$ large enough, $H_n$ has $ m \geq (\pi(G) + \gamma) \binom{n}{k}$ edges.
        Pick $t$ large enough so that $\ex{t}{G} \leq \left(\pi(G) + \frac{\gamma}{2}\right) \binom{t}{k}$.
        Suppose, furthermore, that $n \geq t$.
        Notice that
        \[
            \binom{n-k}{t-k}m = \sum_{T \in \binom{V}{t}} E(H_n[T]).
        \]
        This is because, for each edge in $H_n$, we can choose a set $T \subset V$ containing it in
        $\binom{n-k}{t-k}$ ways.
        We define
        \[
            P = \left\{ T \in \binom{V}{t} \middle| E(H_n[T]) > \left(\pi(G) + \frac{\gamma}{2}\right) \binom{t}{k} \right\}.
        \]
        If $T \in \binom{V}{t}$, the number of edges in $H_n[T]$ is at most $\binom{t}{k}$.
        Therefore,
        \[
            \binom{n-k}{t-k}(\pi(G) + \gamma)\binom{n}{k}
            \leq \binom{n-k}{t-k}m
            \leq |P| \binom{t}{k} + \left(\binom{n}{t} - |P|\right) \left(\pi(G) + \frac{\gamma}{2}\right) \binom{t}{k}.
        \]
        Rearranging and applying standard binomial coefficient identities, we can bound $|P|$ as
        \[
            |P| \geq \frac{\gamma}{2(1 - \pi(G)- \gamma /2)} \binom{n}{t} \geq \frac{\gamma}{2} \binom{n}{t}.
        \]
        Now, for each $T \in P$, $H_n[T]$ contains $G$ as a subgraph.
        Furthermore, each copy of $G$ is in at most $\binom{n - |V|}{t - |V|}$
        such sets.
        Therefore, the number of copies of $G$ in $H_n$ is at least
        \[
            \frac{\gamma}{2} \binom{n}{t} \frac{1}{\binom{n - |V|}{t - |V|}}
            = \frac{\gamma}{2 \binom{t}{|V|}} \binom{n}{|V|}
            = \Omega\left( n^{|V|}\right). \qedhere
        \]
    \end{proof}
\end{theorem}

\subsection{Degenerate Turán-Type Problems}\label{subsec:degenerate}

We now turn our attention to Turán problems for $k$-partite $k$-graphs,
which are the ones that have Turán density $0$ (we will prove so in this section).
All $k$-partite $k$-graphs with part sizes $b_1 \leq a_1, \dots, b_k \leq a_k$
are contained in $\compdots{a_1}{a_k}$ as subgraphs.
This allows us to follow the same argument as in Proposition~\ref{prop:extremal}
to define the following.

\begin{definition}\label{def:zarankiewicz}
    Let $1 < t_1 \leq v_1, \dots, 1 < t_k \leq v_k$ be integers.
    Then the \emph{generalized Zarankiewicz number} $z(v_1, \dots, v_k; t_1, \dots, t_k)$
    is the largest integer $0 \leq z < \prod_i{ v_i}$ for which there exists a $k$-partite $k$-graph
    $H$ with part sizes $ |V_1| = v_1, \dots, |V_k| = v_k$ and $z$ edges
    such that no embedding $f$ of $\compdots{T_1}{T_k}$ with $|T_i| = t_i$ in it exists
    satisfying $f(T_i) \subset V_i$ for all $i$.
\end{definition}

From now on, every time we talk about embeddings from one $k$-partite $k$-graph
$G = (T_1, \dots, T_k; E)$ to another $k$-partite $k$-graph $H = (V_1, \dots, V_k; F)$,
we assume the condition $f(T_i) \subset V_i$.
Similarly to the case of complete graphs,
$H$ contains $\compdots{t_1}{t_k}$ as a subgraph if and only if
for some sets $S_i \subset V_i$ of size $t_i$ for all $i$,
$H[S_1 \cup \dots \cup S_k]$ = $\compdots{S_1}{S_K}$,
and such an embedding is always induced.
Definition~\ref{def:zarankiewicz} is useful for studying the Turán problem for $k$-partite $k$-graphs
in the following way.

\begin{remark}\label{rem:zar_vs_turan}
    Finding Zarankiewicz numbers can help us upper bound the Turán number of $\compdots{t_1}{t_k}$ asymptotically.
    Assume that $H$ is a $\compdots{t_1}{t_k}$-free $n$-vertex $k$-graph with $m$ edges.
    pick $v_1, \dots, v_k$ such that $\sum_{i} v_i = n $ and $v_i \sim n/k $
    (For example $\lfloor n/k \rfloor \leq v_i \leq \lceil n/k \rceil$)
    Let $V_1, \dots, V_k$ be a uniform random partition of $V(H)$ with $|V_i| = v_i$.
    for an edge $e \in E(H)$, the probability that $e$ is an edge in $\compdots{V_1}{V_k}$ is
    greater than
    \[k! \prod_i \frac{v_i}{n} \sim \frac{k!}{k^k},\]
    which is independent of $n$.
    Therefore, the expected number of edges satisfying this condition is a positive fraction of $m$.
    Applying the first moment method, we can conclude that
    \[
        \ex{n}{\compdots{t_1}{t_k}} = \bigO{\zaroversetdots{k}{\lceil n / k \rceil}{t_1}{t_k}}.
    \]

\end{remark}

The problem on finding the Zarankiewicz number was first posed by K. Zarankiewicz in 1951 for the
case of bipartite 2-graphs (that is, finding $z(u, w; s, t)$),
in terms of finding all-1 sub-matrices in a $0-1$ matrix.
An upper bound for it in the case $u=w, s=t$ was found by Kővari, Sós and Turán~\cite{Kovari1954} in 1954.
This was generalized to arbitrary complete
bipartite 2-graphs by C. Hyltén-Cavallius~\cite{Hylten1958} in 1958.
The result is stated and proved here for completeness.

\begin{theorem}[Kővari-Sós-Turán Theorem] \label{thm:kst}
    Let $0 < s \leq u$ and $0 < t \leq w$ be integers.
    Then
    \[z(u, w; s, t) \leq (s - 1)^{1 / t}(w - t + 1)u^{1 - 1 / t} + (t - 1)u\]
    \begin{proof}
        Suppose, by way of contradiction, that we have a $K(s, t)$-free bipartite graph $G = (U, W; E)$
        with $|U| = u$, $|W| = w$ and $|E| = z$ exceeding the bound stated above.
        Let us consider the set
        \[
            P = \left\{ (x, T) \in U \times \binom{W}{t}
            \middle\vert\, \{x, y\} \in E \text{ for all } y \in T \right\}.
        \]
        Counting on the first coordinate, we get
        \begin{equation} \label{eq:kst_p_lower}
            |P| =
            \sum_{x \in U} \binom{d_G(x)}{t} =
            \sum_{x \in U} \varphi(d_G(x)) \geq
            u \varphi(z/u) =
            u \binom{z / u}{t},
        \end{equation}
        where we define
        \[
            \varphi(x) =
            \begin{cases}
                \binom{x}{t}, & \text{if } x \geq t - 1; \\
                0, & \text{otherwise.}
            \end{cases}
        \]
        The function $\varphi$ is convex, so we get the inequality in~\eqref{eq:kst_p_lower}
        as a consequence of Jensen's inequality.
        The other equalities come from the fact that $\varphi(d)$ agrees
        with $\binom{d}{t}$ for all integers $d \geq 0$;
        and that by our bound on $z$, $z \geq (t-1)u \implies z/u \geq t - 1$.

        If we had $s$ different elements of $P$ with the same second coordinate $T$,
        they would all necessarily have different first coordinates
        (say $S = \{x_1, \dots, x_s\}$).
        But now, by definition of $P$, for all $a \in S, b \in T$, we have $\{a, b\} \in E$,
        so $G[S \cup T] = K(S, T)$, against the assumption that $G$ is $K(s, t)$-free.
        Therefore, there are at most $s - 1$ different elements of $P$ for each $T \in \binom{W}{t}$:
        \begin{equation} \label{eq:kst_p_upper}
            |P| \leq (s - 1) \binom{w}{t}.
        \end{equation}
        Putting inequalities~\eqref{eq:kst_p_lower} and~\eqref{eq:kst_p_upper}
        together, we get
        \begin{equation} \label{eq:kst_chained}
            u \binom{z / u}{t} \leq (s - 1) \binom{w}{t}.
        \end{equation}
        Now, because we can see $E$ as a subset of $U \times W$,
        we get $z \leq uw \implies z/u \leq w$.
        We claim that this implies that
        \begin{equation} \label{eq:kst_binom}
            \frac{(z/u - (t - 1))^t}{\binom{z/u}{t}} \leq \frac{(w - (t - 1))^t}{\binom{w}{t}},
        \end{equation}
        because the function
        \[
            g(x) = \frac{(x - (t - 1))^t}{\binom{x}{t}}
        \]
        is increasing for $x \geq t - 1$.
        To see this, we expand the denominator into a product and absorb the $(x - (t - 1))^t$ factor.
        \begin{equation} \label{eq:g_expansion}
            g(x) = \prod_{i=0}^{t-1} (x-(t-1)) \frac{i+1}{x-i} = t! \prod_{i=0}^{t-1} \frac{x-(t-1)}{x-i}.
        \end{equation}
        Every factor in the product on the right side of~\eqref{eq:g_expansion} is increasing
        in $x$ for $x \geq t - 1 \geq i$, proving the claim.
        Multiplying inequalities~\eqref{eq:kst_chained} and~\eqref{eq:kst_binom} yields
        \[
            u \, (z/u - (t - 1))^t \leq (s - 1)(w - (t - 1))^t.
        \]
        Then, algebraic manipulation then gives
        \[
            z \leq (s - 1)^{1 / t}(w - t + 1)u^{1 - 1 / t} + (t - 1)u,
        \]
        In contradiction with our assumption. \qedhere
    \end{proof}

\end{theorem}

\begin{remark}
    Following Remark~\ref{rem:zar_vs_turan}, we can use this bound to get an upper bound on the Turán number of $K(s, t)$:
    \[
        \ex{n}{K(s, t)} =
        \bigO{(s - 1)^{1 / t}\left(\left\lceil\frac{n}{2}\right\rceil - t + 1\right)n^{1 - 1 / t} + (t - 1)\left\lceil\frac{n}{2}\right\rceil} =
        \bigO{n^{2 - 1 / t}}.
    \]
    Note that if $s < t$, we get the better bound $\bigO{n^{2 - 1 / s}}$ by interchanging the roles of $s$ and $t$.
\end{remark}

In 1964, Erdős~\cite{Erods1964} generalized this result to arbitrary complete partite $k$-graphs in the following theorem.

\begin{theorem}\label{thm:erdos64}
    For $k \geq 2$ and $2 \leq t \leq \frac{n}{k}$,
    $\ex{n}{\compoverset{k}{t}} = \bigO{n^{k - \frac{1}{t^{k-1}}}}$.
    \begin{proof}
        By Remark~\ref{rem:zar_vs_turan}, it suffices to show that
        \[
            z = \zaroverset{k}{w}{t} = \bigO{w^{k - \frac{1}{t^{k-1}}}}
        \]
        as $w \to \infty$.
        We prove this by induction on $k$.
        For $k=2$, this is obtained by setting $u = w$ and $s = t$ in Theorem~\ref{thm:kst}.
        For $k > 2$, suppose by way of contradiction that the theorem is false.
        For all ${w_0 \in \N}$, ${K \in \mathbb{R}^+}$, there exists a $k$-partite $k$-graph $G = (W_1, \dots, W_k; E)$ with part sizes
        $|W_i| = w \geq w_0$ and ${|E| \geq K w^{k - \frac{1}{t^{k-1}}}}$ such that no embedding of $\compoverset{k}{t}$ in it exists.
        Consider, for each set $T \in \binom{W_k}{t}$, the associated \text{$(k-1)$-link}
        \link{G}{T}{k-1}.
        We claim that it does not contain $\compoverset{k-1}{t}$ as a subgraph.
        If it did (say, $T_1 \times \dots \times T_{k-1} \in E(\link{G}{T}{k-1})$),
        then $T_1 \times \dots \times T_{k-1} \times T \in E$
        would contradict the assumption that $G$ does not contain $\compoverset{k}{t}$ as a subgraph.
        This means that
        \begin{equation} \label{eq:conditionlink}
            \link{G}{T}{k-1} \text{ has at most $z'$ edges for all } T \in \binom{W_k}{t},
        \end{equation}
        where
        \[
            z' = \zaroverset{k-1}{w}{t}.
        \]
        Now, consider the bipartite graph $G' = (U, W; E')$, where
        \begin{align*}
            U &= W_1 \times \dots \times W_{k-1}, \\
            W &= W_k, \\
            E' &= \{(X, y) \in U \times W \mid X \cup \{y\} \in E\}.
        \end{align*}
        Condition~\eqref{eq:conditionlink} is equivalent to saying that
        there is no embedding of $K(z' + 1, t)$ onto $G'$ respecting the respective partitions.
        Furthermore, $G'$ has the same number of edges as $G$.
        Finally, we invoke Theorem~\ref{thm:kst} with
        ${u = |U| = w^{k-1}}$ and
        ${s = z' + 1}$ to get
        \begin{equation} \label{eq:erdos64_induction}
            K w^{k - \frac{1}{t^{k-1}}} \leq
            |E| = |E'| \leq
            (z')^{1 / t}(w - t + 1)w^{(k-1)(1 - 1 / t)} + (t - 1)w^{k-1}.
        \end{equation}
        By the inductive hypothesis, for $w_0$ and $K'$ large enough, we can bound
        \begin{equation} \label{eq:erdos64prime}
            z' \leq K' w^{(k - 1) - \frac{1}{t^{k-2}}}.
        \end{equation}
        Substituting inequality~\eqref{eq:erdos64prime} into~\eqref{eq:erdos64_induction} and approximating yields
        \[
            K w^{k - \frac{1}{t^{k-1}}} \leq (K')^{1 / t} w^{k-\frac{1}{t^{k-1}}} + (t - 1)w^{k-1}.
        \]
        Combining like terms and picking $K > 2 (K')^{1 / t}$ gives
        \[
            \frac{1}{2}K w^{k - \frac{1}{t^{k-1}}} < (t - 1)w^{k-1},
        \]
        which we can rewrite as
        \begin{equation} \label{eq:erdos64_contradiction}
            \frac{1}{2}Kw^{1 - \frac{1}{t^{k-1}}} < (t - 1).
        \end{equation}
        This is a contradiction, because the right side of inequality~\eqref{eq:erdos64_contradiction}
        is constant in $w$, while the left side grows to infinity as $w$ increases.
    \end{proof}
\end{theorem}

This approach can be generalized to give a lower bound on the number of
copies (that is, embeddings with different image sets)
of $\compdots{t_1}{t_k}$ in a $k$-partite $k$-graph $G$
with different part sizes~\cite{carvajal2024canonical},
therefore upper bounding all generalized Zarankiewicz numbers.
Applying the same observations that we have made for the balanced case,
we arrive at
\begin{equation} \label{eq:general-partite-bound}
    \ex{n}{\compdots{t_1}{t_k}} = \bigO{n^{k - \frac{1}{\prod_{i=1}^{k-1} t_i}}}.
\end{equation}

Because all $k$-partite $k$-graphs can be embedded in a $\compoverset{k}{t}$,
Theorem~\ref{thm:erdos64} shows that the Turán density of all $k$-partite $k$-graphs is $0$,
completing the proof of Proposition~\ref{prop:degenerate}.
Knowing that the Turán density of $k$-partite $k$-graphs is $0$ gives little information on the growth of their Turán numbers.
In this case, we are usually satisfied with determining the growth up to a constant factor.

General lower bounds are usually obtained by probabilistic arguments,
which are often weak and do not reach the correct order of magnitude.
For example, see the following probabilistic construction.

\begin{proposition} \label{prop:probabilistic-lower-bound}
    Let $G = (T_1, \dots, T_k; E)$ be a $k$-partite $k$-graph with
    $t = \sum_{i=1}^{k} t_i = \sum_{i=1}^{k} |T_i|$ vertices
    and $e = |E| > 1$ edges.
    Then, $\ex{n}{G} = \Omega\left(n^{k - \frac{t - k}{e - 1}} \right)$, and the
    constant factor depends only on the number of edges $e$ (and not on the number of vertices $t$).
    \begin{proof}
        Let $n \geq t$.
        We use the so-called \emph{random alteration} method to construct a $k$-graph
        $H_n$ with $n$ vertices that does not contain $G$ as a subgraph.
        We first define $R_n = (V, E)$ to be a random $k$-graph on a vertex set $V$ of size $n$,
        where each edge $e \in \binom{V}{k}$ is included in $E$ independently at random
        with a certain probability $p \in (0, 1)$.
        The expected number of edges in $R_n$ is
        \[
            \mathbb{E}(|E|) = p \binom{n}{k} \geq p \left( \frac{n}{k} \right)^k.
        \]
        Let us now count the number of possible injective functions of $V(G)$ in $V$.
        They are defined by the (ordered) choice of the image of each vertex, so there are
        \[
            \prod_{j=1}^{t} (n - j + 1) \leq n^t
        \]
        of them.
        The probability that any particular injective function $f$ of $V(G)$ in $V$ is an embedding of $G$ in $R_n$
        is calculated as the product of the probabilities that each image of an edge is an edge in $R_n$,
        because the presence of edges in $R_n$ is independent.
        Therefore,
        \[
            \mathbb{P}(f \text{ is an embedding of } G) = p^{|E|} = p^{e}.
        \]
        If we define $X$ to be the number of embeddings of $G$ in $R_n$, by linearity of expectation we get
        \[
            \mathbb{E}(X) = \sum_{f} \mathbb{P}(f \text{ is an embedding of } G) \leq n^t p^{e}.
        \]
        We can now obtain a $G$-free $k$-graph $H_n$ by removing from $R_n$, for each embedding of $G$,
        the image of an edge of $G$.
        The expected number of edges in $H_n$ is
        \[
            \mathbb{E}(|E(H_n)|) = \mathbb{E}(|E|) - \mathbb{E}(X) \geq
            p \left( \frac{n}{k} \right)^k - n^t p^{e}.
        \]
        This quantity is maximized by setting
        \[
            p = \left( \frac{1}{ek^k} n^{k-t} \right)^{\frac{1}{e-1}}.
        \]
        This yields
        \[
            \mathbb{E}(|E(H_n)|) \geq
            m_0(n) =
            \left( \frac{(e-1)^{e-1}}{e^e k^{ek}} \right)^{\frac{1}{e-1}} n^{k - \frac{t-k}{e-1}}
            = \Omega_{e}\left(n^{k - \frac{t-k}{e-1}} \right).
        \]
        Therefore, the event that $|E(H_n)| \geq m_0(n)$
        must have positive probability, and in particular,
        there exists one such graph $\widehat{H_n}$, which is $G$-free by construction.

    \end{proof}

\end{proposition}

Note that this also implies that
\[
    \ex{n}{\compdots{t_1}{t_k}} = \Omega\left(n^{k - \frac{\left( \sum_i t_i \right) - k}{\left( \prod_i t_i \right) - 1}}\right).
\]
This leaves a very large gap between the upper and lower bounds for the Turán numbers of degenerate $k$-graphs.
For example, in the balanced case, where all $t_i$ are equal, we get
\begin{equation} \label{eq:balanced_upper_bound}
    \ex{n}{\compoverset{k}{t}} = \bigO{n^{k - \frac{1}{t^{(k-1)}}}}
\end{equation}
from Theorem~\ref{thm:erdos64}, but only
\begin{equation} \label{eq:balanced_lower_bound}
    \ex{n}{\compoverset{k}{t}} = \Omega\left(n^{k - \frac{k(t-1)}{t^k - 1}}\right)
\end{equation}
from Proposition~\ref{prop:probabilistic-lower-bound}.
The exponent in~\eqref{eq:balanced_lower_bound} is always
smaller than the one in~\eqref{eq:balanced_upper_bound},
as long as $t \geq 2$ and $k \geq 2$.

There are even fewer solved cases for degenerate Turán problems than in the non-degenerate case.
Among them, it is known that, for $K(2, t)$ (for $t \geq 2$) and $K(3, 3)$,
Theorem~\ref{thm:kst} is optimal in the sense that
\[
    \ex{n}{K(2, t)} = \Theta\left(n^{\frac{3}{2}}\right)~\cite{erdHos1966problem, furedi1996new},
\]
and also
\[
    \ex{n}{K(3, 3)} = \Theta\left(n^{\frac{5}{3}}\right)~\cite{brown1966graphs}.
\]
The theorem is also optimal for $K(s, t)$ when $s \geq t! + 1$~\cite{kollar1996norm}.
Some progress has been made in the case of $K(s, t)$ when $s$ and $t$ have similar sizes,
only for small values of $s \geq t \geq 4$.
For example, Theorem~\ref{thm:kst} gives
\[
    \ex{n}{K(5,5)} = \bigO{n^{\frac{9}{5}}} = \bigO{n^{1.8}},
\]
and by Proposition~\ref{prop:probabilistic-lower-bound} we get
\begin{equation} \label{eq:k55-probabilistic}
    \ex{n}{K(5,5)} = \Omega\left(n^{\frac{5}{3}}\right) = \Omega\left(n^{1.67}\right),
\end{equation}
but~\eqref{eq:k55-probabilistic} has been improved to
\[
    \ex{n}{K(5,5)} = \Omega\left(n^{\frac{7}{4}}\right) = \Omega\left(n^{1.75}\right)~\cite{ball2012asymptotic}.
\]
It is conjectured that Theorem~\ref{thm:kst} always gives the correct order of magnitude.
Even less is known about degenerate problems for graphs of higher uniformity.
For example, not even the growth rate of the Turán number for the octahedron 3-graph
($\ex{n}{K(2, 2, 2)}$, pictured in Figure~\ref{fig:222}) is known.
The best upper bound, again, comes from Theorem~\ref{thm:erdos64}, which gives
\[
    \ex{n}{K(2, 2, 2)} = \bigO{n^{\frac{11}{4}}} = \bigO{n^{2.75}},
\]
while the best know lower bound is
\[
    \ex{n}{K(2, 2, 2)} = \Omega\left(n^{\frac{8}{3}}\right) = \Omega\left(n^{2.67}\right)~\cite{conlon2020random}.
\]

The main difficulty for degenerate problems is that sharp lower bounds for the Turán numbers
often rely on specific geometric or algebraic constructions that work for very few cases,
such as the ones cited for $K(2, 2)$ and $K(3, 3)$.

Similarly to non-degenerate Turán problems, we also have interesting results when the
Turán density of a $k$-partite $k$-graph (namely, $0$) is exceeded.
The following observation is key.

\begin{remark} \label{rem:uniform-t}
    The constant factors in the estimates of Theorem ~\ref{thm:erdos64} can be made to depend only on $k$ and not on $t$.
    Reducing to the partite host case by Remark~\ref{rem:zar_vs_turan} only reduces the number of edges by a constant factor.
    For $k=2$, Theorem~\ref{thm:kst} applied with $s = t$ and $u = w$
    gives
    \[
        z(w, w, t, t) \leq (t-1)^{1/t}(w-t+1)w^{1-1/t} + (t-1)w \leq 2w^{2 - 1/t} + o\left(w^{1-1/t}\right).
    \]
    For $k \geq 3$, we can prove it inductively, by carrying out the computations of the theorem more carefully.
\end{remark}

Elementary computation then shows that if
$H_n$ is a family of $k$-partite $k$-graphs with constant positive density
(that is, $E(H_n) \geq \gamma \binom{n}{k} = \Omega\left( n^k \right)$ for some $\gamma > 0$),
for large enough $n$, $H_n$ contains a $\compdots{t_n}{t_n}$ as a subgraph, with
\[
    t_n = \Omega \left( (\log n)^{\frac{1}{k-1}}\right).
\]
The following theorem generalizes this notion to blow-ups of arbitrary $k$-partite $k$-graphs.

\begin{theorem} \label{thm:quant-blowup}
    Let $G = (V, E)$ be a $k$-graph and let $(W_n, F_n)$ be a family of $k$-graphs with $n$
    vertices and $\left( \pi(G) + \Omega (1) \right) \binom{n}{k}$ edges.
    Then, for $n$ large enough, $H_n$ contains $G(t_n)$ as a subgraph,
    where
    \[
        t_n = \Omega\left( (\log n)^{\frac{1}{|V|-1}} \right).
    \]
    \begin{proof}
        By Theorem~\ref{thm:supersaturation}, $H_n$ contains $\Omega\left( n^{|V|} \right)$ copies of $G$.
        Consider the $|V|$-graph $H_n' = (W_n, F_n')$, where
        \[
            F_n' = \{f(V) \mid f \text{ is an embedding of } G \text{ in } H_n\}.
        \]
        Because each set of $|V|$ vertices in $H_n$ can be the image of at most $|V|! = \bigO{1}$ embeddings of $G$,
        $H_n'$ has $\Omega\left(n^{|V|} \right)$ edges.
        Furthermore, by the same argument as in Remark~\ref{rem:zar_vs_turan}, $H_n'$, we can assume that
        $H_n'$ is $|V|$-partite (say, $H_n' = (W_1, \dots, W_{|V|}; F_n')$ with
        $|W_i| \geq \left\lfloor n / |V| \right\rfloor$
        without losing this condition.
        Each edge in this graph is the image of $V$ under an embedding $f$ of $G$ in $H_n$.
        Numbering the vertices of $G$ as $v_1, \dots, v_{|V|}$, $f$ corresponds to a permutation $\sigma$
        of the vertices of $G$, where $f(v_i) \in W_j$ when $\sigma(i) = j$.
        Picking the permutation $\hat{\sigma}$ that has the most edges corresponding to it
        (at least $|E_n'| / |V|! = \Omega\left( n^{|V|} \right)$), and without loss of generality,
        each edge in $(u_1, \dots, u_{|V|}) \in F_n' \subset W_1 \times \dots \times W_{|V|}$
        corresponds to the fact that for each edge $\{v_{i_1}, \dots, v_{i_k}\} \in E$,
        $\{u_{i_1}, \dots ,u_{i_k}\}$ is an edge of $H_n$.
        By Remark~\ref{rem:uniform-t}, we can find a \compoverset{|V|}{t_n} in $H_n'$,
        where
        \[
            t_n = \Omega\left( (\log n)^{\frac{1}{|V|-1}} \right).
        \]
        That is, there are $|V|$ sets $T_1, \dots, T_{|V|}$ of size $t_n$ such that
        if $\{v_{i_1}, \dots, v_{i_k}\} \in E$ is an edge of $G$, then
        $\{u_{i_1}, \dots, u_{i_k}\} \in F_n$ is an edge of $H_n$ for all $u_{i_j} \in T_j$.
        This means that $G(t)$ is a subgraph of $H_n$.

    \end{proof}
\end{theorem}

The following corollary highlights that degenerate Turán problems can be applied to solve non-degenerate ones.
\begin{corollary}
    Let $G$ be a $k$-graph and $t$ be a positive integer.
    Then, $\pi(G(t)) = \pi(G)$.
\end{corollary}

In particular, this directly proves the Erdős-Stone-Simonovits theorem for $2$-graphs.

\begin{delayedproof}{thm:erdos_stone_simonovits}
    Let $G$ be a $2$-graph with chromatic number $r$.
    For some $t \geq 1$, $G$ is a subgraph of $\compdotssuperindexoverset{2}{r}{t} \cong \completesuperindex{2}{r}(t)$.
    Therefore,
    \[
        \pi(G) \leq \pi\left(\completesuperindex{2}{r}(t)\right) = \pi\left(\completesuperindex{2}{r}\right) = 1 - \frac{r-2}{r-1}.
    \]
    The reverse inequality follows from the same construction in the \proofref{thm:turan} of Turán's theorem.
    Indeed, this construction has the desired density and avoids not only \completesuperindex{2}{r} but also
    any $r$-partite $2$-graph, and in particular $G$.
\end{delayedproof}

Theorem~\ref{thm:quant-blowup} is known not to be optimal.
Erdős and Bollobás~\cite{bollobas1973structure} in fact proved that the optimal growth
rate of a guaranteed $t_n$-blow-up of a $2$-graph $G$ in $2$-graphs of constant density greater than $\pi(G)$ is
\[
    t_n = \Theta (\log n).
\]
A still open question raised by the authors is whether this can be extended to $k$-graphs.
That is, is it true that $k$-graphs with $n$ vertices and  $\left( 1 + \Omega(1) \right) \pi(G) \binom{n}{k}$ edges
contain $G(t_n)$ for some $t_n = \Omega \left((\log n)^{\frac{1}{k-1}}\right)$?
An even more general yet unresolved question is whether this is true for
$k$-graphs with $n$ vertices and $\Omega (n^{|V(G)|})$ copies of $G$~\cite{rodl2012complete}.
    \section{Our Algorithm}\label{sec:algorithm}

In this section we present a polynomial-time algorithm to find a balanced partite $k$-graph in a given $k$-graph $G$
with $n$ vertices and $m$ edges with part size in the same order of magnitude as stated in
Theorem~\ref{thm:erdos64}.

\begin{remark}
    If we let $q$ be the size of each part in the partite $k$-graph we are looking for, we need
    \[
        m \geq ex(n, K(t, \overset{k}{\cdots}, t)) = O\left(n^{k - \frac{1}{t^{k-1}}}\right)
    \]
    Defining $d = \frac{m}{n^k}$, and taking logarithms, this is true iff
    \[
        \log d \geq - \frac{\log n}{t^{k-1}} + O(1)
    \]
    which implies
    \[
        t = O\left(\left(\frac{\log n}{\log (1/d)}\right)^{\frac{1}{k-1}}\right)
    \]
\end{remark}

This algorithmic problem has already been solved for $k = 2$ Mubayi and Turán~\cite{MUBAYI2010174}.
The algorithm in that case follows very closely the structure of the proof of Theorem~\ref{thm:kst}.
We outline the algorithm for $k = 2$ here for context and clarity.
The variable names have been altered to match the notation used in this thesis.

\begin{algorithm}
    \caption{Finding a balanced partite graph in a graph $G$ with $n$ vertices and $m = dn^2$ edges}
    \label{alg:bipartite}
        \begin{algorithmic}[1]
        \Require Integers \(a > 0\) and \(b > 0\)
        \Ensure \( \gcd(a, b) \), the greatest common divisor of \(a\) and \(b\)
        \While{$b \neq 0$}
            \State $r \gets a \mod b$ \Comment{Compute the remainder of \(a\) divided by \(b\)}
            \State $a \gets b$
            \State $b \gets r$
        \EndWhile
        \State \Return $a$ \Comment{\(a\) now holds the GCD of the original inputs}
        \end{algorithmic}
\end{algorithm}
    \section{Conclusions and Future Work}\label{sec:conclusions} % TODO write this myself

This thesis has focused on the algorithmic aspects of finding $k$-partite subgraphs in $k$-uniform hypergraphs, a problem central to degenerate Turán theory.
We have presented a deterministic, polynomial-time algorithm (Algorithm~\ref{alg:kpartite}) that, given a $k$-graph $G$ on $n$ vertices with $m$ edges, finds a complete balanced $k$-partite $k$-subgraph $\compoverset{k}{t}$. The part size $t$ achieved, given by Equation~\eqref{eq:t} as $t \approx (\log n / \log(1/d))^{1/(k-1)}$ where $d=m/n^k$, closely matches the parameters implicit in the non-constructive existence proofs of Erd\H{o}s for such structures. This provides an efficient, constructive counterpart to these classical results, demonstrating that these fundamental subgraphs can indeed be located algorithmically within polynomial time. The recursive approach, which reduces the uniformity $k$ by analyzing appropriately defined link graphs, generalizes previous work for the $k=2$ case by Mubayi and Turán.

There are several avenues for future research stemming from this work:

\begin{enumerate}
    \item \textbf{Tightening Bounds and Improving Practicality:}
    The proofs of correctness for Algorithm~\ref{alg:kpartite}, particularly Lemmas~\ref{lm:sound} through~\ref{lm:t_prime}, involve several inequalities. While sufficient to establish the polynomial runtime and the asymptotic nature of $t$, some of these bounds are quite loose (e.g., approximations of binomial coefficients, conditions for $w \leq n/2$, or the constants in the density arguments). A more meticulous analysis could potentially yield sharper constants in the definition of $t(n,d,k)$ or relax the minimum density requirements (Remark~\ref{rm:min_d}). This could make the algorithm applicable to sparser hypergraphs or guarantee larger $k$-partite structures for a given density, enhancing its practical significance for analyzing real-world hypergraphs which might not meet the currently proven, somewhat high, minimum vertex or density thresholds.

    \item \textbf{Finding Blow-ups of General $k$-Graphs:}
    The presented algorithm is tailored to find blow-ups of a single edge, i.e., $\compoverset{k}{t}$. A natural extension would be to adapt this algorithmic framework to find $t_n$-blowups $H(t_n)$ of an arbitrary fixed $k$-graph $H$. As discussed in Section~\ref{subsec:degenerate} (Theorem~\ref{thm:quant-blowup}), $k$-graphs with density $\pi(G) + \epsilon$ are known to contain $G(t_n)$ where $t_n = \delta (\log n)^{1/(|V(G)|-1)}$. Our current algorithm, if applied iteratively or adapted, might yield a constructive proof for finding such blow-ups.
    However, for $k=2$, it is known from Bollobás and Erd\H{o}s~\cite{bollobas1973structure} that the optimal growth for $t_n$ is $\delta \log n$, which is better than the $(\log n)^{1/(|V(G)|-1)}$ if $|V(G)| > 2$. The general question for $k > 2$ of whether $t_n = \delta (\log n)^{1/(k-1)}$ (or perhaps even better, related to $|E(G)|$) can be achieved for blow-ups of general $k$-graphs $G$ (not just an edge) via an efficient algorithm remains open. Investigating new algorithmic techniques, possibly deviating from the direct link-graph recursion if it proves suboptimal for general blow-ups, could be a fruitful direction.

    \item \textbf{Average-Case Analysis and Randomized Algorithms:}
    The current algorithm is deterministic. Exploring randomized versions might lead to simpler algorithms or improved part sizes $t$ on average, or with high probability. Furthermore, analyzing the performance of this algorithm on random hypergraph models (e.g., $G_{n,p}^{(k)}$) could provide insights into its typical behavior.

    \item \textbf{Implementation and Experimental Evaluation:}
    Implementing Algorithm~\ref{alg:kpartite} and evaluating its performance on various synthetic and real-world hypergraph datasets would be valuable. This could help identify practical bottlenecks and compare its findings with theoretical guarantees, especially concerning the constants involved in the calculation of $t$.
\end{enumerate}

In summary, while this thesis provides a constructive step forward in finding specific partite structures, the broader landscape of algorithmic extremal hypergraph theory, especially concerning tighter bounds and more general forbidden subgraphs, remains rich with open questions.


    \section{Bibliography}\label{sec:bibliography}
    \bibliography{main-mrefed}
    \bibliographystyle{myplain}

    \appendix
    \vfill\newpage \section{Properties of Hypergraph Embeddings}\label{apx:embeddings-properties}

\begin{proposition}\label{prop:embedding_properties}
    Graph inclusion $(\subset)$ and induced graph inclusion $\left(\subset_{\text{ind}}\right)$
    are preorder relations on $k$-graphs.
    \begin{proof}
        We need to show that the relations are reflexive and transitive.
        Reflexivity is clear, as the identity map is an induced embedding of a $k$-graph in itself.
        Let $G, H,$ and $K$ be $k$-graphs with vertex sets $X, Y,$ and $Z$ respectively.
        If $G \subset H$ via $f: X \to Y$ and $H \subset K$ via $g: Y \to Z$,
        then $g \circ f: X \to Z$ is injective and satisfies that for each edge $e \in E(G)$,
        \[
            g \circ f(e) =
            \{g(f(x))\mid x \in e\} =
            \{g(y) \mid y \in f(e)\} \in E(K),
        \]
        because $f(e) \in E(H)$.
        Therefore, $G \subset K$ via $g \circ f$.
        If the embeddings are induced,
        and $e$ is an edge in
        $E(K[g \circ f(X)])$,
        then $e$ is also an edge in $E(K[g (Y)]) = g(E(H))$.
        Therefore, $e' = g^{-1}(e)$ is an edge in $H$.
        Furthermore, because $e = g(e') \subset g \circ f(X)$,
        we have that $e' \in E(H[f(X)]) = f(E(G))$,
        so $e \in g(f(E(G)))$.
    \end{proof}
\end{proposition}

\begin{remark} \label{rem:inverse_embedding}
    In Definition~\ref{def:embedding}, given that a map $f: V \to W$ is an embedding
    (and therefore injective),
    a different way to state that it is an induced embedding is to say that
    $f^{-1}: H[f(V)] \to G$ is also an embedding.
\end{remark}

\begin{proposition}\label{prop:isomorphism_equivalence}
    The relation of isomorphism $(\cong)$ is an equivalence relation on $k$-graphs.
    \begin{proof}
        The relation is reflexive via the identity map.
        If $f: G \to H$ is an isomorphism, then $f^{-1}: H \to G$ is also an isomorphism,
        so the relation is symmetric.
        Finally, if $f: G \to H$ and $g: H \to K$ are isomorphisms,
        then $g \circ f: G \to K$ is also an isomorphism, because it is bijective,
        and by Proposition~\ref{prop:embedding_properties},
        it is an embedding
        and $(g \circ f)^{-1} = f^{-1} \circ g^{-1}$ is also an embedding.
        By remark~\ref{rem:inverse_embedding}, we are done.
    \end{proof}
\end{proposition}

\begin{proposition} \label{prop:isomorphism_preserves_embedding}
    Let $G, G', H, H'$ be $k$-graphs such that $G \cong G'$ and $H \cong H'$.
    Then,
    \begin{enumerate}
        \item $G \subseteq H$ if and only if $G' \subseteq H'$. \label{item:embedding}
        \item $G \subseteq_{\text{ind}} H$ if and only if $G' \subseteq_{\text{ind}} H'$. \label{item:induced_embedding}
    \end{enumerate}
    \begin{proof}
        Because the isomorphism relation is symmetric, we only need to show one direction of each implication.
        let $f: V(G) \to V(H)$ be an embedding of $G$ in $H$, and let
        $g: V(G) \to V(G')$ and $h: V(H) \to V(H')$ be isomorphisms between the respective graphs.
        We claim that the composition
        \[
            f' = h \circ f \circ g^{-1}: V(G') \to V(H')
        \]
        is an embedding of $G'$ in $H'$.
        Injectivity is given by the injectivity of $h, f$, and $g^{-1}$.
        By Proposition~\ref{prop:isomorphism_equivalence}, we have that $g^{-1}$ is an isomorphism of $G'$ in $G$,
        and in particular an embedding.
        Therefore, by Proposition~\ref{prop:embedding_properties},
        $f'$ is an embedding of $G'$ in $H'$, proving part~\eqref{item:embedding}.
        Suppose now that the embedding $f$ is induced.
        consider the maps
        \[
            (f')^{-1}: f'(V(G')) \to V(G')
        \]
        and
        \[
            \varphi = g \circ f^{-1} \circ h^{-1}: h \circ f(V(G)) \to V(G'),
        \]
        where we restrict the domain of $f^{-1}$ to $f(V(G))$.
        Because $g$ is a bijection, $V(G) = g^{-1}(V(G'))$ so the
        domain of $\varphi$ is $f'(V(G'))$.
        In fact, one can check that the two functions are identical.
        Because $f^{-1}$ is an embedding of $H[f(V(G))]$ in $G$,
        we can argue as in the first case that
        $(f')^{-1}$ is an embedding of $H'[f'(V(G'))]$ in $G'$,
        and therefore $f'$ is an induced embedding of $G'$ in $H'$.
        This concludes the proof of part~\eqref{item:induced_embedding}.
    \end{proof}
\end{proposition}

Propositions~\ref{prop:isomorphism_equivalence},~\ref{prop:embedding_properties},
and~\ref{prop:isomorphism_preserves_embedding}
establish that the properties of containing a $k$-graph $G$ as a (induced)
subgraph within a $k$-graph $H$ depend only on the isomorphism classes of $G$ and $H$.
Therefore, discussions of (induced) subgraph containment can be conducted up to isomorphism.

\end{document}
