\section{Introduction}\label{sec:introduction}

Graph theory provides fundamental tools for modeling relationships and networks across diverse fields.
A natural and powerful extension of graphs is the concept of \emph{hypergraphs},
where edges can connect more than two vertices.
Specifically, a $k$-uniform hypergraph, or $k$-graph,
consists of a set of vertices and a collection of edges, each being a $k$-element subset of the vertices
For example, a $2$-graph is simply an undirected graph with no loops or parallel edges.
$k$-graphs arise naturally in areas ranging from combinatorics and computer science to data analysis and
computational biology.

A central branch is extremal (hyper)graph theory.
This field seeks to understand the maximum or minimum size of a combinatorial structure satisfying certain properties.
For instance, \emph{Turán-type problems} ask how many edges a $k$-graph can have, as a function of its number of vertices $n$,
without containing a specific subgraph $G$.
The maximum such number of edges is called the \emph{Turán number} of $G$ on $n$ vertices and is denoted by $\ex{n}{G}$.
A key result is Turán's Theorem~\cite{Turan1941},
which determines $\ex{n}{K_r}$ for all $n$ and $r \ge 2$, where $K_r$ is the complete graph on $r$ vertices.
Furthermore, the Erdős--Stone--Simonovits Theorem~\cite{erdos1946structure}
asymptotically estimates $\ex{n}{G}$ for any fixed $2$-graph $G$ as $n \to \infty$,
and as a function of the chromatic number of $G$.

We do not yet understand how to extend these theorems to hypergraphs,
as the combinatorial structures become significantly more complex.
The asymptotic behavior of $\ex{n}{G}$ as $n \to \infty$
is characterized by the \emph{Turán density} of $G$, defined as
\[
    \pi(G) = \lim_{n \to \infty} \frac{\ex{n}{G}}{\binom{n}{k}}.
\]
Determining the exact value of $\pi(G)$ for $k$-graphs when $k > 2$
is a notoriously difficult open problem for many families,
including even small hypergraphs like the complete $3$-graph on $4$ vertices,
$K_4^{(3)}$~\cite{keevash2011hypergraph, razborov20103}.
This thesis focuses on the \emph{degenerate} case, where $\pi(G) = 0$.
A fundamental result states that this is the case if and only if $G$ is $k$-partite
(meaning its vertices can be partitioned into $k$ sets such that no edge has two vertices in the same set).
We are particularly interested in the problem of forbidding complete balanced $k$-partite $k$-graphs,
denoted $\compoverset{k}{t}$, which consist of $k$ disjoint sets of $t$ vertices each,
and all $t^k$ edges formed by selecting one vertex from each set.
The classical Kővari--Sós--Turán Theorem~\cite{Kovari1954, Hylten1958}
provides the following upper bound for $k=2$.
\[
    \ex{n}{K(s, t)} = \bigO{n^{2 - \frac{1}{\min\{s, t\}}}}.
\]
Erdős~\cite{Erods1964} found an analogous bound for complete balanced $k$-partite $k$-graphs for $k \ge 2$,
showing that
\begin{equation} \label{eq:erdos64-intro}
    \ex{n}{\compoverset{k}{t}} = \bigO{n^{k - \frac{1}{t^{(k-1)}}}}.
\end{equation}

Upper bounds for Turán numbers, like~\eqref{eq:erdos64-intro},
often involve counting or probabilistic arguments,
which are inherently non-constructive.
They guarantee the existence of the desired subgraph $\compoverset{k}{t}$
in dense enough hypergraphs but do not typically provide an efficient algorithm to \emph{find} such a subgraph.
If we focus on a fixed guest $k$-graph $G = (V, E)$ and let the number $n$ of vertices of the host $k$-graph $H$ grow,
this is not considered a problem,
as a brute-force search over all ordered sets of $|V|$ vertices in $H$ yields a polynomial-time algorithm
for finding a copy of $G$ in $H$.
However, the situation becomes more complex when we consider the case where $G$ is not fixed, but rather grows with $n$.
We focus on the case where the edge density of $H$ is fixed (that is, $H$ has at least $\epsilon \binom{n}{k}$ edges),
and $G = \compoverset{k}{t}$ for some $t$ that can grow with $n$.
Careful analysis of the proof of Erdős' bound shows that this guarantees that $G$ is a subgraph of $H$ for some
\begin{equation} \label{eq:t-lower-intro}
    t = \delta(\epsilon) (\log n)^{\frac{1}{k-1}}.
\end{equation}
Running a brute-force search checking all $\binom{n}{kt}$ sets of $kt$
vertices in $H$ would then \textbf{not} yield a polynomial-time algorithm,
because $\binom{n}{kt}$ grows superpolynomially with $n$.

The main contribution of this thesis is bridging this gap by providing an efficient algorithmic solution.
We develop and analyze a deterministic, polynomial-time algorithm that,
given a $k$-graph $G$ with $n$ vertices and at least $dn^k$ edges,
finds a complete balanced $k$-partite subgraph $\compoverset{k}{t}$ within $G$, where
\[
    \left\lfloor \left(  \frac{\log \left(n/2^{(k-1)}\right)}{\log (3/d)} \right)^{\frac{1}{k-1}} \right\rfloor,
\]
matching the order of magnitude of~\eqref{eq:t-lower-intro}.
This result not only provides a constructive proof for the upper bounds of the type established by Erdős,
but in fact reaches the best possible value of $t$ up to a constant factor depending on $d$, as
can be shown by probabilistic arguments (see~\Cref{prop:probabilistic-lower-bound} and
the beginning of \Cref{sec:algorithm}, where the algorithm is introduced).
Our algorithm generalizes the approach used by Mubayi and Turán for the bipartite case ($k=2$)~\cite{MUBAYI2010174}.
It employs a recursive strategy that mirrors the inductive proof structure of Erdős' bound,
iteratively reducing the uniformity $k$ by constructing appropriate link graphs.

\textbf{Organization of the Thesis:}
\Cref{sec:preliminaries} formally introduces some families of hypergraphs
(including complete $k$-partite hypergraphs),
as well as basic operations like restrictions, links, and blow-ups.
We also use this section to introduce asymptotic notation, which is used throughout the thesis.
\Cref{sec:extremal} provides an overview of relevant theoretical results for Turán-type problems,
proving central theorems like the Turán Theorem (\Cref{thm:turan}), the Kővari--Sós--Turán Theorem (\Cref{thm:kst}),
Erdős' bound (\Cref{thm:erdos64}) and a more precise version of it (\Cref{thm:erdos64-constant-density}),
thus showing that finding a complete balanced $k$-partite $k$-graph of part sizes in the order of~\eqref{eq:t-lower-intro}
exists in $H$ when it has constant positive density.
\Cref{sec:algorithm} presents our main algorithm (\Cref{alg:kpartite}),
provides a rigorous proof of its correctness and analyzes its polynomial runtime complexity (\Cref{thm:kpartite}).
Finally, \Cref{sec:conclusions} summarizes the main results of this thesis, and discusses some open problems for future research.
