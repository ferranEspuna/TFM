\section{Hypergraph Turán Problems}\label{sec:extremal}

\subsection{Turán-Type Problems}\label{subsec:turan}

Now we can state the \emph{forbidden subgraph problem} for $k$-graphs.
Informally, given a $k$-graph $G$, and an integer $n \geq |V(G)|$,
we want to find the smallest $M_0$ such that all $k$-graphs with $n$ vertices and $m > M_0$ edges
contain $G$ as a subgraph.

\begin{proposition} \label{prop:extremal}
    Let $G = (V, E)$ be a $k$-graph with nonempty edge set and $n \geq |V|$ be an integer.
    Then there exists an integer $M_0 = \ex{n}{G} \in \left[ 0, \binom{n}{k}\right)$ such that
    the condition
    \[
        \text{``All $k$-graphs with $n$ vertices and $m$ edges contain $G$ as a subgraph.''}
    \]
    is true for all $\binom{n}{k} \geq m > M_0$ and false for all $0 \leq m \leq M_0$.

    \begin{proof}
        Note that, if such an $M_0$ exists, clearly it is unique.
        Also, the condition is clearly false for $m = 0$ and
        true for $m = \binom{n}{k}$
        (the only graph $H$ with vertex set $W$, $|W|=n$ and $\binom{n}{k}$ edges
        is the one having all $k$-sets of vertices so any injective map $f: V \to W$
        is an embedding of $G$ in $H$).
        We only need to show that if the condition is true for $m$ then it is true for
        all $m' \geq m$.
        Suppose it is true for $m$ and let $m' \geq m$.
        Let $H = (W, F)$ be a $k$-graph with $n$ vertices and $m'$ edges.
        We can take $F' \subset F$ with $|F'| = m$.
        By hypothesis, the graph $H' = (W, F')$ contains $G$ as a subgraph,
        and the identity map in $W$ is an embedding of $H'$ in $H$.
        Then, $G \subset H' \subset H$ implies $G \subset H$ by transitivity of the embedding
        relation (Proposition~\ref{prop:embedding_properties}).

    \end{proof}

\end{proposition}

We call the integer $\ex{n}{G}$ the \emph{Turán number} of $G$ on $n$ vertices.
The Turán number is increasing both in $n$ and under graph inclusion. % TODO Gemini says this is wrong
The first property can be seen by taking a $G$-free $k$-graph on $H$ with $n$ vertices
and $\ex{n}{G}$ vertices; and adding a vertex $v$ and no edges to it, obtaining $H'$.
If $f: G \to H$ is an embedding and $v$ has a preimage $x$, it must be a vertex in $H$ with degree $0$,
so it can be replaced by any other vertex in $H$ outside the image of $f$.
Restricting this new mapping to $H$, we get that $G \subset H$, in contradiction to our assumption.
For the second, suppose that $G \subset G'$.
Because $H$ is $G$-free, it is also $G'$-free, which means that it has at most $\ex{n}{G'}$ edges.
Therefore, $\ex{n}{G} \leq \ex{n}{G'}$.
As a consequence of this, we also get that the Turán number is invariant under isomorphism of $G$.

There are very few $k$-graphs $G$ for which an exact formula for $\ex{n}{G}$ is known.
Of these, the most famous family of examples are the complete $2$-graphs $\completesuperindex{2}{r}$,
for which Turán numbers were first studied by Turán~\cite{Turan1941} in 1941.
The result is the following.

\begin{theorem}[Turán's Theorem]
    \label{thm:turan}
    Let $r > 2$ be an integer and let $n \geq r$.
    Let $a_1, \dots, a_{r-1}$ be integers such that $a_1 + \dots + a_{r-1} = n$
    and $\lfloor n / (r-1) \rfloor \leq a_i \leq \lceil n / (r-1) \rceil$ for all $i$.
    Then
    \begin{equation} \label{eq:turan}
        \ex{n}{\completesuperindex{2}{r}} = \sum_{\{x, y\} \in \binom{[r-1]}{2}} a_x \cdot a_y
    \end{equation}
    Furthermore, if $G$ is a $2$-graph with $\ex{n}{\completesuperindex{2}{r}}$ edges
    and $G$ does not contain $K_r^{(2)}$ as a subgraph, then
    \[
        G\cong \compdotssuperindex{2}{a_1}{a_{r-1}}.
    \]

\end{theorem}

Before the \proofref{thm:turan} of Turán's theorem, we introduce two lemmas.

\begin{lemma}\label{lem:same_degree}
    Let $G = (V, E)$ be a $2$-graph with $n$ vertices and
    $\ex{n}{\completesuperindex{2}{r}}$ edges.
    If $x, y \in V$ are different vertices and $\{x, y\} \notin E$, then $d_G(x) = d_G(y)$.
    \begin{proof}
        We argue by contradiction.
        Suppose, without loss of generality, that $d_G(x) > d_G(y)$.
        We argue that we can construct a $2$-graph $G'$ with $n$ vertices
        and more edges than $G$ that does not contain $K_r^{(2)}$ as a subgraph,
        contradicting the definition of
        the Turán number.

        The new graph $G' = (E', V')$ is constructed from $G$ by removing from $V$ the vertex $y$
        (and all edges containing it)
        and adding a copy $x'$ of $x$, connected to the same vertices (that is, $\{x', v\} \in E'$
        if and only if $\{x, v\} \in E$).
        Clearly, $|V'| = |V|$ and $|E'| = E - d_G(y) + d_G(x) > |E|$.
        To see that $G'$ does not contain $K_r^{(2)}$ as a subgraph,
        suppose that $G'[T']$ is complete for some $T' \subset V'$ of size $r$.
        Because $\{x, x'\}$ is not an edge in $G'$, $T'$ cannot contain both $x$ and $x'$.
        Because the edges not containing $x'$ are the same as in $G$, which contains no $K_r^{(2)}$,
        we deduce that $T'$ contains $x'$ and therefore does not contain $x$.
        Now, let $T = (T' \setminus \{x'\}) \cup \{x\} \subset V$, also of size $r$.
        We argue that the graph $G[T] = G'[T]$ must be complete, reaching a contradiction.
        If it were not, then there would exist $v \in T, v \neq x$, such that $\{x, v\} \notin E$.
        This implies that $\{x', v\} \notin E'$, but $v \in T' \setminus \{x'\}$,
        which contradicts our assumption that $G'[T']$ is complete.
    \end{proof}
\end{lemma}

\begin{lemma} \label{lem:turan_complete_partite}
    Let $G = (V, E)$ be a $2$-graph with $n$ vertices and
    $\ex{n}{\completesuperindex{2}{r}}$ edges.
    Then, $G$ is a complete $p$-partite graph for some $p \geq 2$.
    \begin{proof}
        Equivalently, we show that the relation defined by non-adjacency on $V$ (that is, $x \sim y$ when
        $\{x, y\} \notin E$) is an equivalence relation, so we can divide $V$ into equivalence classes
        by this relation, which means that $\{x, y\} \in E$ if and only if they are in different parts.

        The reflexivity and symmetry of the relation are clear.
        Suppose, by way of contradiction, that there exist $x, y, z \in V$ such that
        $x \sim z$ and $y \sim z$, but $x \nsim y$.
        We now construct a different graph $G'$ with the same number of vertices as $G$
        that also does not contain $K_r^{(2)}$ as a subgraph, reaching a contradiction.
        $G'$ is constructed from $G$ by removing the vertices
        $x$ and $y$ (and all the associated edges) and adding the two new vertices
        $z_1$ and $z_2$ and the edges $\{\{v, z_i\} \mid \{v, z\} \in E, i \in \{1, 2\}\}$.

        First, we show that $G'$ contains no embedding of $K_r^{(2)}$.
        We make a similar argument as in the proof of Lemma~\ref{lem:same_degree}.
        By way of contradiction, suppose that $G'[T']$ is complete for some $T' \subset V'$ of size $r$.
        Because $z$, $z_1$ and $z_2$ pairwise non-edges of $G'$, only one of them can be
        an image of a vertex in $K_r^{(2)}$.
        However, $G'[V \setminus \{x, y\}] \subset G$ has no embedding of $K_r^{(2)}$,
        so at least one of the vertices in $K_r^{(2)}$ must be mapped to $z_1$ or $z_2$.
        Without loss of generality, we can write $T' = \{x_1, x_2, x_3, \dots, x_{r-1}, z_1\}$,
        with $x_i \notin \{z_2, z\}$ for all $i$.
        However, $\{z_1, x_i\}$ is an edge in $G'$ if an only if $\{z, x_i\}$ is an edge in $G$,
        which means that $G'[\{x_1, x_2, x_3 \dots, x_{r-1}, z\}] = G[\{x_1, x_2, x_3 \dots, x_{r-1}, z\}]$ is complete,
        contradicting our assumption.

        Now, we show that $G'$ has more edges than $G$.
        By Lemma~\ref{lem:same_degree}, $d_G(x) = d_G(z)$ and $d_G(y) = d_G(z)$,
        so the three vertices $x, y, z$ have the same degree $d$ in $G$.
        The edges containing $x$ and the edges containing $y$ intersect at exactly the edge $\{x, y\} \in E$.
        Therefore, by removing all of them from $G$ we are removing $2d - 1$ edges.
        Furthermore, for each edge containing $z$ we are adding two edges,
        and these sets of edges do not intersect because $z$ is not adjacent to $x$ or $y$ (so $\{z_1, z_2\} \notin E'$).
        We conclude that $G'$ has $|E'| = |E| - (2d - 1) + 2d = |E| + 1 > |E| $ edges, as desired.
    \end{proof}
\end{lemma}

Now we are ready to prove Turán's theorem.
\begin{delayedproof}{thm:turan}
    We have shown in Lemma~\ref{lem:turan_complete_partite} that $G = (V_1, \dots V_p; E)$ is complete.
    In fact, we can set $p = r - 1$:
    If $p < r - 1$, we can always add empty parts to $G$; and if it has more than $r - 1$ nonempty parts
    (without loss of generality, $x_1 \in V_1, \dots, x_r \in V_r$), then $G[\{x_1, \dots, x_r\}]$ is complete,
    which is a contradiction.
    Furthermore, any complete $(r-1)$-partite $2$-graph is $\completesuperindex{2}{r}$-free,
    because $\completesuperindex{2}{r}$ is not $(r-1)$-partite.

    This means that we only need to show that the choice of the part sizes $a_1, \dots, a_{r-1}$ summing to $n$
    in the statement maximizes the expression~\eqref{eq:turan}.
    The condition that $\lfloor n / (r-1) \rfloor \leq a_i \leq \lceil n / (r-1) \rceil$ for all $i$
    is equivalent to requiring that the part sizes are as equal as possible, that is,
    $|a_i - a_j| \leq 1$ for all $i, j$.
    Suppose, by way of contradiction and without loss of generality, that $a_1 > a_2 + 1$.
    Let $a_1' = a_1 - 1$, $a_2' = a_2 + 1$ and $a_i' = a_i$ for all $i \geq 3$.
    Then,
    \begin{align*}
        \sum_{\{x, y\} \in \binom{[r-1]}{2}} a_x' \cdot a_y'
        =& \, (a_1 - 1)(a_2 + 1) + (a_1 - 1) \sum_{i \geq 3} a_i + (a_2 + 1) \sum_{i \geq 3} a_i + \sum_{3 \leq x < y} a_x a_y \\
        =& \sum_{\{x, y\} \in \binom{[r-1]}{2}} a_x \cdot a_y - a_2 + a_1 - 1
        > \sum_{\{x, y\} \in \binom{[r-1]}{2}} a_x \cdot a_y,
    \end{align*}
    in contradiction to the number of edges in $G$ being maximal.
\end{delayedproof}

Because of the difficulty of finding exact Turán numbers for $k$-graphs,
we usually look for asymptotic approximations of them.
In particular, we are interested in how the expression
$\ex{n}{G}$ grows with $n$ for any fixed $k$-graph $G$.
This is known as the \emph{Turán problem} for the graph $G$.
For an example, we turn to the complete $2$-graph $\completesuperindex{2}{r}$,
for which we already have an exact formula.
In expression~\eqref{eq:turan}, we can see that
$a_i = n/(r-1) + \bigO{1}$ for all $i$.
Therefore,
\begin{equation} \label{eq:turan_asymptotic}
    \ex{n}{\completesuperindex{2}{r}} = \sum_{\{x, y\} \in \binom{[r-1]}{2}} a_x \cdot a_y
    = \binom{r-1}{2} \cdot \left( \frac{n}{r-1} + \mathcal{O}(1) \right)^2
    = \frac{(r-2)}{2(r-1)} n^2 + \mathcal{O}(n).
\end{equation}
Note that the maximum number of edges in a $2$-graph on $n$ vertices is
\[
    \binom{n}{2} = \frac{1}{2} n^2 + \mathcal{O}(n).
\]
The two quantities are comparable as they are both quadratic in $n$.
This allows us to restate equation~\eqref{eq:turan_asymptotic} as
\begin{equation} \label{eq:turan_asymptotic_density}
    \ex{n}{\completesuperindex{2}{r}} =
    \frac{r-2}{r-1} \binom{n}{2} + \mathcal{O}(n) =
    \left(1 - \frac{1}{r-1} + o(1)\right)\binom{n}{2},
\end{equation}
which means that, asymptotically, the maximum \emph{edge density} of a $2$-graph on $n$ vertices
without $K_r^{(2)}$ as a subgraph is $(r-2)/(r-1) < 1$, so we must exclude a nontrivial fraction of edges
to avoid any particular complete $2$-graph.
The following general theorem greatly restricts the growth of Turán numbers
for all $k$-graphs.

\begin{theorem}
    Let $G = (V, E)$ be a $k$-graph.
    The limit
    \begin{equation} \label{eq:turan_density}
        \pi(G) = \lim_{n \to \infty} \frac{\ex{n}{G}}{\binom{n}{k}}
    \end{equation}
    exists and is between $0$ and $1$.
    It is called it the \emph{Turán density} of $G$.
    \begin{proof}
        The sequence
        \[
            a_n = \frac{\ex{n}{G}}{\binom{n}{k}}
        \]
        is bounded between $0$ and $1$ for all $n \geq |V(G)|$, by Proposition~\ref{prop:extremal}.
        Furthermore, it is less than $1$ for all $n \geq |V(G)| + 1$.
        To see this, consider a graph $H = (W, F)$ with $n$ vertices and $\binom{n}{k} - 1$ edges.
        Its edge density is less than $1$.
        Without loss of generality, we can suppose that $F = \binom{W}{k} \setminus \{\{x, y\}\}$.
        This means that $H[W \setminus \{x\}]$ is a complete $k$-graph on $n - 1$ vertices,
        which must contain $G$ as a subgraph.

        We show that the sequence $(a_n)$ is non-increasing, so it must converge to a value $0 \leq \pi(G) < 1$.
        Let $n \geq |V(G)|$.
        There exists a graph $H = (W, F)$ with $n+1$ vertices and $\ex{n+1}{G}$ edges that does not contain
        $G$ as a subgraph.
        For each vertex $w \in W$, the graph $H_w = H[W \setminus \{w\}]$ has $n$ vertices
        and does not contain $G$ as a subgraph.
        Therefore, it must contain at most $\ex{n}{G}$ edges.
        Consider the set
        \[
            \mathcal{P} = \left\{ (w, e) \in W \times F \mid e \in E(H_w) \right\}.
        \]
        Counting on the first coordinate, we get
        \begin{equation} \label{eq:densityUpperBound}
            |\mathcal{P}| = \sum_{w \in W} |E(H_w)| \leq (n+1)\, \ex{n}{G}.
        \end{equation}
        On the other hand, for every edge $e \in F$, $e \in E(H_w)$
        for all $w \in W \setminus e$.
        Therefore, counting on the second coordinate, we get
        \begin{equation} \label{eq:densityLowerBound}
            |\mathcal{P}| = (n + 1 - k) |F| = (n + 1 - k)\, \ex{n+1}{G}.
        \end{equation}
        Combining equations~\eqref{eq:densityUpperBound} and~\eqref{eq:densityLowerBound},
        we get
        \[
            (n + 1 - k)\, \ex{n+1}{G} \leq (n + 1)\, \ex{n}{G}.
        \]
        Going back to the sequence $a_n$, we can write
        \[
            a_{n+1} = \frac{\ex{n+1}{G}}{\binom{n+1}{k}} \leq
            \frac{(n + 1)\, \ex{n}{G}}{(n + 1 - k) \binom{n+1}{k}} =
            \frac{\ex{n}{G}}{\binom{n}{k}} = a_n. \qedhere
        \]
    \end{proof}
\end{theorem}

We can now summarize expression~\eqref{eq:turan_asymptotic_density} as follows.
\begin{corollary} \label{cor:turan_density_kr}
    The Turán density of the complete $2$-graph $\completesuperindex{2}{r}$ is
    \[
        \pi\left(\completesuperindex{2}{r}\right) = \frac{r-2}{r-1} = 1 - \frac{1}{r-1}.
    \]
\end{corollary}

The first natural question that arises is for what graphs $G$ the Turán density $\pi(G)$ is positive
(in which case, we call the corresponding Turán problem \emph{non-degenerate}
and consider it solved if we can calculate $\pi(G)$).
The following gives a complete characterization.

\begin{proposition} \label{prop:degenerate}
    Let $G = (V, E)$ be a $k$-graph.
    Then $\pi(G) = 0$ if and only if $G$ is $k$-partite.
    \begin{proof}
        If $G$ is not $k$-partite, a construction similar to the one in the proof of Theorem~\ref{thm:turan}
        directly shows $\pi (G) > 0$.
        Indeed, for all $m$ the graph $\compoverset{k}{m}$ is $k$-partite so it cannot contain $G$ as a subgraph.
        Furthermore, its edge density is
        \[
             \frac{m^k}{\binom{km}{k}} \geq \frac{1}{k^k}.
        \]
        Because we can make $n = km = |V\left( \compoverset{k}{m} \right)|$ as large as we want,
        the limit~\eqref{eq:turan_density} bounded below by a positive constant.
        We defer the proof of the other direction to subsection~\ref{subsec:degenerate},
        where we study $k$-partite $k$-graph Turán problems
        in more depth (in particular, see Theorem~\ref{thm:erdos64}).
    \end{proof}
\end{proposition}

In fact, non-degenerate Turán problems for $2$-graphs are considered solved in this regard.
The following theorem gives the Turán density of all $2$-graphs.

\begin{theorem}[Erdős--Stone--Simonovits Theorem]
    \label{thm:erdos_stone_simonovits}
    Let $G = (V, E)$ be a $2$-graph and let $r = \chi(G)$.
    Then,
    \[
        \pi(G) = 1 - \frac{1}{r - 1}.
    \]
\end{theorem}

We defer the \proofref{thm:erdos_stone_simonovits} of this theorem to the next subsection,
where we will have more powerful tools at our disposal.
Note that letting $G = \completesuperindex{2}{r}$ we recover
Corollary~\ref{cor:turan_density_kr} of Theorem~\ref{thm:turan}.

We know that $k$-graphs asymptotically below the Turán number of a $k$-graph $G$
may not contain $G$ as a subgraph.
We may also ask how many copies (different embeddings) of $G$ can be found in a $k$-graph $H$
exceeding the Turán density.
The following surprising result~\cite{erdHos1983supersaturated} shows that the number of copies of $G$ in $H$
is asymptotically guaranteed to be very large.
In essence, if a $k$-graph $H_n$ has $n$ vertices and $\left(\pi(G) + \Omega(1) \right) \binom{n}{k}$ edges,
then it must contain $\Omega\left(n^{|V|}\right)$ copies of $G$ as a subgraph,
which correspond to a positive fraction of all the functions from $V(G)$ to $V(H)$.

\begin{theorem}[Supersaturation] \label{thm:supersaturation}
    Let $G = (V, E)$ be a $k$-graph and let $\epsilon > 0$.
    There exists a positive integer $t = t(G, \epsilon)$ and a constant $\delta = \delta(G, \epsilon) > 0$
    such that any $k$-graph $H$ with $n \geq t$ vertices and at least $\left(\pi(G) + \epsilon \right) \binom{n}{k}$ edges
    contains at least $\delta n^{|V|}$ copies of $G$ as a subgraph.

    \begin{proof}
        Pick $t(G, \epsilon)$ large enough so that
        $\ex{t}{G} \leq \left(\pi(G) + \frac{\epsilon}{2}\right) \binom{t}{k}$.
        Let $m \geq \left(\pi(G) + \epsilon \right) \binom{n}{k}$
        be the number of edges of $H$.
        Notice that
        \[
            \binom{n-k}{t-k}m = \sum_{T \in \binom{V}{t}} |E(H[T])|.
        \]
        This is because, for each edge in $H$, we can choose a set $T \subset V$ containing it in
        $\binom{n-k}{t-k}$ ways.
        We define
        \[
            P = \left\{ T \in \binom{V}{t} \middle| E(H[T]) > \left(\pi(G) + \frac{\epsilon}{2}\right) \binom{t}{k} \right\}.
        \]
        If $T \in \binom{V}{t}$, the number of edges in $H[T]$ is at most $\binom{t}{k}$.
        Therefore,
        \[
            \binom{n-k}{t-k}(\pi(G) + \epsilon)\binom{n}{k}
            \leq \binom{n-k}{t-k}m
            \leq |P| \binom{t}{k} + \left(\binom{n}{t} - |P|\right) \left(\pi(G) + \frac{\epsilon}{2}\right) \binom{t}{k}.
        \]
        Rearranging and applying standard binomial coefficient identities, we can bound $|P|$ as
        \[
            |P| \geq \frac{\epsilon}{2(1 - \pi(G)- \epsilon /2)} \binom{n}{t} \geq \frac{\epsilon}{2} \binom{n}{t}.
        \]
        Now, for each $T \in P$, $H[T]$ contains $G$ as a subgraph.
        Furthermore, each copy of $G$ is in at most $\binom{n - |V|}{t - |V|}$
        such sets.
        Therefore, the number of copies of $G$ in $H$ is at least
        \[
            \frac{\epsilon}{2} \binom{n}{t} \frac{1}{\binom{n - |V|}{t - |V|}}
            = \frac{\epsilon}{2 \binom{t}{|V|}} \binom{n}{|V|}
            \geq \frac{\epsilon}{2 \binom{t}{|V|} |V|^{|V|}} n^{|V|}.
        \]
        Picking $\delta = \frac{\epsilon}{2 \binom{t}{|V|} |V|^{|V|}}$ gives us the desired result.
    \end{proof}
\end{theorem}

\subsection{Degenerate Turán-Type Problems}\label{subsec:degenerate}

We now turn our attention to Turán problems for $k$-partite $k$-graphs,
which are the ones that have Turán density $0$ (we will prove so in this section).
All $k$-partite $k$-graphs with part sizes $b_1 \leq a_1, \dots, b_k \leq a_k$
are contained in $\compdots{a_1}{a_k}$ as subgraphs.
This allows us to follow the same argument as in Proposition~\ref{prop:extremal}
to define the following.

\begin{definition}\label{def:zarankiewicz}
    Let $1 < t_1 \leq v_1, \dots, 1 < t_k \leq v_k$ be integers.
    Then the \emph{generalized Zarankiewicz number} $z(v_1, \dots, v_k; t_1, \dots, t_k)$
    is the largest integer $0 \leq z < \prod_i{ v_i}$ for which there exists a $k$-partite $k$-graph
    $H$ with part sizes $ |V_1| = v_1, \dots, |V_k| = v_k$ and $z$ edges
    such that no embedding $f$ of $\compdots{T_1}{T_k}$ with $|T_i| = t_i$ into $H$ exists
    satisfying $f(T_i) \subset V_i$ for all $i$.
\end{definition}

From now on, every time we talk about embeddings from one $k$-partite $k$-graph
$G = (T_1, \dots, T_k; E)$ to another $k$-partite $k$-graph $H = (V_1, \dots, V_k; F)$,
we assume the condition $f(T_i) \subset V_i$.
Similarly to the case of complete graphs,
$H$ contains $\compdots{t_1}{t_k}$ as a subgraph if and only if
for some sets $S_i \subset V_i$ of size $t_i$ for all $i$,
$H[S_1 \cup \dots \cup S_k]$ = $\compdots{S_1}{S_K}$,
and such an embedding is always induced.
Definition~\ref{def:zarankiewicz} is useful for studying the Turán problem for $k$-partite $k$-graphs
in the following way.

\begin{remark}\label{rem:zar_vs_turan}
    Finding Zarankiewicz numbers can help us upper bound the Turán number of $\compdots{t_1}{t_k}$.
    Assume that $H$ is a $\compdots{t_1}{t_k}$-free $n$-vertex $k$-graph with $m$ edges.
    pick $v_1, \dots, v_k$ such that $\sum_{i} v_i = n $ and $v_i \sim n/k $
    (for example, $\lfloor n/k \rfloor \leq v_i \leq \lceil n/k \rceil$).
    Let $V_1, \dots, V_k$ be a uniform random partition of $V(H)$ with $|V_i| = v_i$.
    Assuming, for example, that $n \geq 2k$,
    for any edge $e \in E(H)$, the probability that $e$ is an edge in $\compdots{V_1}{V_k}$ is
    at least
    \[
        k! \prod_i \frac{v_i}{n} \geq \frac{k!}{(2k)^k}
    \]
    which is independent of $n$.
    Therefore, the expected number of edges satisfying this condition is a positive fraction of $m$.
    Applying the first moment method, there is at least one partition retaining $ \frac{k!}{(2k)^k} m$ edges.
    This means that, if the size of each part is greater than $t_i$ (that is, $n \geq k t_i$ for all $i$)
    and $m$ is greater than $z(v_1, \dots, v_k; t_1, \dots, t_k) \cdot \frac{(2k)^k}{k!}$, then
    $H$ must contain $\compdots{t_1}{t_k}$ as a subgraph.
    All in all,
    \[
        \ex{n}{\compdots{t_1}{t_k}} \leq \frac{(2k)^k}{k!} \cdot \zaroversetdots{k}{\lceil n / k \rceil}{t_1}{t_k}.
    \]

\end{remark}

The problem of finding the Zarankiewicz number was first posed by K. Zarankiewicz in 1951 for the
case of bipartite 2-graphs (that is, finding $z(u, w; s, t)$),
in terms of finding all-1 sub-matrices in a $0$-$1$ matrix.
An upper bound for it in the case $u=w, s=t$ was found by Kővari, Sós and Turán~\cite{Kovari1954} in 1954.
This was generalized to arbitrary complete
bipartite 2-graphs by Hyltén--Cavallius~\cite{Hylten1958} in 1958.
The result is stated and proved here for completeness.

\begin{theorem}[Kővari--Sós--Turán Theorem] \label{thm:kst} % TODO G --> H
    Let $0 < s \leq u$ and $0 < t \leq w$ be integers.
    Then
    \[z(u, w; s, t) \leq (s - 1)^{1 / t}(w - t + 1)u^{1 - 1 / t} + (t - 1)u\]
    \begin{proof}
        Suppose, by way of contradiction, that we have a $K(s, t)$-free bipartite graph $G = (U, W; E)$
        with $|U| = u$, $|W| = w$ and $|E| = z$ exceeding the bound stated above.
        Let us consider the set
        \[
            P = \left\{ (x, T) \in U \times \binom{W}{t}
            \middle\vert\, \{x, y\} \in E \text{ for all } y \in T \right\}.
        \]
        Counting on the first coordinate, we get
        \begin{equation} \label{eq:kst_p_lower}
            |P| =
            \sum_{x \in U} \binom{d_G(x)}{t} =
            \sum_{x \in U} \varphi(d_G(x)) \geq
            u \varphi(z/u) =
            u \binom{z / u}{t},
        \end{equation}
        where we define
        \[
            \varphi(x) =
            \begin{cases}
                \binom{x}{t}, & \text{if } x \geq t - 1; \\
                0, & \text{otherwise.}
            \end{cases}
        \]
        The function $\varphi$ is convex, so we get the inequality in~\eqref{eq:kst_p_lower}
        as a consequence of Jensen's inequality.
        The other equalities come from the fact that $\varphi(d)$ agrees
        with $\binom{d}{t}$ for all integers $d \geq 0$;
        and that by our bound on $z$, $z \geq (t-1)u \implies z/u \geq t - 1$.

        If we had $s$ different elements of $P$ with the same second coordinate $T$,
        they would all necessarily have different first coordinates
        (say $S = \{x_1, \dots, x_s\}$).
        But now, by definition of $P$, for all $a \in S, b \in T$, we have $\{a, b\} \in E$,
        so $G[S \cup T] = K(S, T)$, contradicting the assumption that $G$ is $K(s, t)$-free.
        Therefore, there are at most $s - 1$ different elements of $P$ for each $T \in \binom{W}{t}$:
        \begin{equation} \label{eq:kst_p_upper}
            |P| \leq (s - 1) \binom{w}{t}.
        \end{equation}
        Putting inequalities~\eqref{eq:kst_p_lower} and~\eqref{eq:kst_p_upper}
        together, we get
        \begin{equation} \label{eq:kst_chained}
            u \binom{z / u}{t} \leq (s - 1) \binom{w}{t}.
        \end{equation}
        Now, because we can see $E$ as a subset of $U \times W$,
        we get $z \leq uw \implies z/u \leq w$.
        We claim that this implies that
        \begin{equation} \label{eq:kst_binom}
            \frac{(z/u - (t - 1))^t}{\binom{z/u}{t}} \leq \frac{(w - (t - 1))^t}{\binom{w}{t}},
        \end{equation}
        because the function
        \[
            g(x) = \frac{(x - (t - 1))^t}{\binom{x}{t}}
        \]
        is increasing for $x \geq t - 1$.
        To see this, we expand the denominator into a product and absorb the $(x - (t - 1))^t$ factor.
        \begin{equation} \label{eq:g_expansion}
            g(x) = \prod_{i=0}^{t-1} (x-(t-1)) \frac{i+1}{x-i} = t! \prod_{i=0}^{t-1} \frac{x-(t-1)}{x-i}.
        \end{equation}
        Every factor in the product on the right side of~\eqref{eq:g_expansion} is increasing
        in $x$ for $x \geq t - 1 \geq i$, proving the claim.
        Multiplying inequalities~\eqref{eq:kst_chained} and~\eqref{eq:kst_binom} yields
        \[
            u \, (z/u - (t - 1))^t \leq (s - 1)(w - (t - 1))^t.
        \]
        Then, algebraic manipulation then gives
        \[
            z \leq (s - 1)^{1 / t}(w - t + 1)u^{1 - 1 / t} + (t - 1)u,
        \]
        in contradiction to our assumption. \qedhere
    \end{proof}

\end{theorem}

\begin{remark}
    Following Remark~\ref{rem:zar_vs_turan}, we can use this bound to get an upper bound on the Turán number of $K(s, t)$:
    \[
        \ex{n}{K(s, t)} =
        \bigO{(s - 1)^{1 / t}\left(\left\lceil\frac{n}{2}\right\rceil - t + 1\right)n^{1 - 1 / t} + (t - 1)\left\lceil\frac{n}{2}\right\rceil} =
        \bigO{n^{2 - 1 / t}}.
    \]
    Note that if $s < t$, we get the better bound $\bigO{n^{2 - 1 / s}}$ by interchanging the roles of $s$ and $t$.
\end{remark}

In 1964, Erdős~\cite{Erods1964} generalized this result to arbitrary complete partite $k$-graphs in the following theorem.

\begin{theorem}[Erdős 1964 Theorem]\label{thm:erdos64}
    For $k \geq 2$,
    $\ex{n}{\compoverset{k}{t}} = \bigO{n^{k - \frac{1}{t^{k-1}}}}$.
\end{theorem}

This theorem is a consequence of the following lemma.

\begin{lemma}~\label{lem:erdos64-quant}
    Let $k \geq 2$ be an integer.
    There exists a constant $\delta = \delta_k > 0$
    such that, for all integers $t \leq w$,
    \[
        \zaroverset{k}{w}{t} < \delta w^{k - \frac{1}{t^{k-1}}}.
    \]
    \begin{proof}
        We proceed by induction on $k$.
        For $k=2$, this is obtained by setting $u = w$ and $s = t$ in Theorem~\ref{thm:kst}.
        This yields
        \[
            z(w, w, t, t)
            \leq (t-1)^{1/t}(w-t+1)w^{1-1/t} + (t-1)w
            < 2w^{2 - 1/t} + tw,
        \]
        where the right inequality comes from $(t-1)^{1/t} < 2$ for all positive $t$.
        We now argue that $tw < 2 w^{2-1/t}$,
        which will conclude the proof for $k=2$ by setting $\delta_2 = 4$.
        Otherwise,
        \[
            t \geq 2 w^{1 - 1/t} \geq 2 t^{1 - 1/t},
        \]
        which is false for all positive $t$.

        For $k > 2$, suppose that the lemma holds for $k - 1$ and that
        a certain $\delta > 0$ does not meet our conditions.
        There exist integers $t \leq w$ and a $k$-partite $k$-graph $H=(W_1, \dots, W_k; F)$
        with part sizes $|W_i| = w$ and $z = |F| \geq \delta w^{k - \frac{1}{t^{k-1}}}$ edges
        such that no embedding of $\compoverset{k}{t}$ into $H$ exists.
        Consider, for each set $T \in \binom{W_k}{t}$, the associated \text{$(k-1)$-link}
        \link{H}{T}{k-1}.
        We claim that it does not contain $\compoverset{k-1}{t}$ as a subgraph.
        If it did (say, $T_1 \times \dots \times T_{k-1} \subset E(\link{H}{T}{k-1})$),
        then $T_1 \times \dots \times T_{k-1} \times T \subset F$
        would contradict the assumption that $H$ does not contain $\compoverset{k}{t}$ as a subgraph.
        This means that
        \begin{equation} \label{eq:conditionlink}
            \link{H}{T}{k-1} \text{ has at most $z'$ edges for all } T \in \binom{W_k}{t},
        \end{equation}
        where
        \[
            z' = \zaroverset{k-1}{w}{t} \leq \delta_{k-1} w^{(k - 1) - \frac{1}{t^{k-2}}}.
        \]
        Now, consider the bipartite graph $H' = (U, W; F')$, where
        \begin{align*}
            U &= W_1 \times \dots \times W_{k-1}, \\
            W &= W_k, \\
            F' &= \{(X, y) \in U \times W \mid X \cup \{y\} \in F\}.
        \end{align*}
        Condition~\eqref{eq:conditionlink} is equivalent to saying that
        there is no embedding of $K(z' + 1, t)$ onto $H'$ respecting the partitions.
        Furthermore, $H'$ has the same number of edges as $H$.
        Finally, we invoke Theorem~\ref{thm:kst} with
        ${u = |U| = w^{k-1}}$ and
        ${s = z' + 1}$ to get
        \begin{equation} \label{eq:erdos64_induction}
            \delta w^{k - \frac{1}{t^{k-1}}} \leq
            |E| = |E'| \leq
            \left(\delta_{k-1} w^{(k - 1) - \frac{1}{t^{k-2}}}\right)^{1 / t}(w - t + 1)w^{(k-1)(1 - 1 / t)} + (t - 1)w^{k-1}.
        \end{equation}
        Approximating, this implies that
        \[
            \delta w^{k - \frac{1}{t^{k-1}}} < \delta_{k-1}^{1 / t} w^{k - \frac{1}{t^{k-1}}} + tw^{k-1} \leq \delta_{k-1}w^{k - \frac{1}{t^{k-1}}} + tw^{k-1}.
        \]
        Similarly as before, one can check that $tw^{k-1} < 2w^{k-\frac{1}{t^{k-1}}}$.
        Therefore, we reach a contradiction whenever $\delta \geq \delta_{k-1} + 2$,
        so setting $\delta_k = \delta_{k-1} + 2$ gives us the desired result.
        In fact, the theorem works for $\delta_k = 2 \cdot k$.

    \end{proof}
\end{lemma}

The proof of the Erdős 1964 Theorem is now straightforward.

\begin{delayedproof}{thm:erdos64}
    If $t=1$, there is nothing to prove,
    so suppose that $t \geq 2$.
    Following Remark~\ref{rem:zar_vs_turan}, if $n \geq tk$
    (and in particular $n \geq 2k$),
    \[
        \ex{n}{\compoverset{k}{t}}
        \leq \frac{(2k)^k}{k!} \cdot \zaroverset{k}{\lceil n / k \rceil}{t}
        \leq 2k \cdot \frac{(2k)^k}{k!} \cdot \left\lceil \frac{n}{k} \right\rceil^{k - \frac{1}{t^{k-1}}}
        \leq \frac{k \cdot 4^k}{(k-1)!} \cdot n^{k - \frac{1}{t^{k-1}}}. \qedhere
    \]

\end{delayedproof}

Because all $k$-partite $k$-graphs can be embedded in a $\compoverset{k}{t}$,
Theorem~\ref{thm:erdos64} shows that the Turán density of all $k$-partite $k$-graphs is $0$,
completing the proof of Proposition~\ref{prop:degenerate}.
Note that the constant factor found in the bound of Theorem~\ref{thm:erdos64}
does not depend on $t$.
This lets us restate the theorem in the following stronger form.

\begin{theorem} \label{thm:erdos64-constant-density}
    Let $k \geq 2$ be an integer.
    There exist an integer $n_k$ and a positive constant $\gamma_k$
    such that, for all
    $0 < \epsilon < 1$,
    all graphs with more than $\epsilon \binom{n}{k}$ edges
    contain $\compoverset{k}{t_n}$ as a subgraph, where
    \[
        t_n = \left\lfloor \left( \frac{\log n}{\log(\gamma_k/\epsilon)} \right)^{\frac{1}{k-1}} \right\rfloor.
    \]
    \begin{proof}
        Again, if $t_n=1$, there is nothing to prove.
        Suppose that $t_n \geq 2$.
        Let us define
        \[
            c_k = \frac{k \cdot 4^k}{(k-1)!}.
        \]
        In the \proofref{thm:erdos64} of Theorem~\ref{thm:erdos64}, we have shown that
        \[
            \ex{n}{\compoverset{k}{t_n}} \leq c_k n^{k - \frac{1}{t_n^{k-1}}}
        \]
        as long as $n \geq t_n k$.
        Suppose that $H$ is a $k$-graph with $n$ vertices and $m \geq \epsilon \binom{n}{k}$ edges.
        Suppose, furthermore, that $n \geq t_n k$
        (we will later show that this condition can be made true independently of $n$ for our chosen $t_n$).
        This condition also implies that the number of edges of $H$ is at least
        \[
            \epsilon \binom{n}{k} \geq \epsilon \frac{(n - k + 1)^k}{k!} \geq \epsilon \left( \frac{1}{2} \right)^k \frac{1}{k!} n^{k} = (e_k  \cdot \epsilon) n^{k},
        \]
        where $e_k = \frac{1}{k! \cdot 2^k}$ does not depend on $n$.
        We pick, for example,
        \[
             \gamma_k = \frac{2c_k}{e_k} = 2 \cdot 8^k \cdot k,
        \]
        so that
        \[
            \ex{n}{\compoverset{k}{t_n}} \leq c_k n^{k - \frac{1}{t_n^{k-1}}} \leq c_k n^k  \frac{\epsilon}{\gamma_k}
            = \frac{e_k \cdot \epsilon}{2} n^{k} \leq  \frac{|E(H)|}{2} < |E(H)|,
        \]
        which guarantees that $H$ contains $\compoverset{k}{t_n}$ as a subgraph.
        The only thing left to prove is that, for this choice of $t_n$, there exists $n_k$
        such that $n \geq n_k$ implies $n \geq t_n k$.
        Indeed, we can pick any
        \[
            n_k \geq \left( k \left(\frac{1}{\log(\gamma_k)} \right)^{\frac{1}{k-1}}\right)^2.
        \]
        Using that $\gamma_k / \epsilon > \gamma_k$ (because $\epsilon < 1$),
        that $\gamma_k > 1$ (which can be easily checked from the definitions)
        and the inequality $\log n < \sqrt{n}$ yields
        \[
            t_n k
            = \left\lfloor \left( \frac{\log n}{\log(\gamma_k / \epsilon)} \right)^{\frac{1}{k-1}} \right\rfloor k
            < \left\lfloor \left( \frac{\log n}{\log\gamma_k} \right)^{\frac{1}{k-1}} \right\rfloor k
            <   \left( \frac{1}{\log \gamma_k} \right)^{\frac{1}{k-1}} k \sqrt{n}
            \leq \left( \frac{1}{\log \gamma_k} \right)^{\frac{1}{k-1}} k \frac{n}{\sqrt{n_k}}
            \leq n. \qedhere
        \]
    \end{proof}
\end{theorem}

This is the result we prove constructively in Section~\ref{sec:algorithm}.
It is stronger than Theorem~\ref{thm:erdos64}
because it bounds the Turán number of partite $k$-graphs uniformly,
while obtaining the same order of magnitude.
Indeed, suppose we have a fixed value for $t$.
We may choose $\epsilon$ such that $t_n \geq t$.
We only need that
\[
    \epsilon \geq \gamma_k \cdot n^{-\frac{1}{t^{k-1}}},
\]
that is, the graph has at least
\[
    \gamma_k \cdot n^{-\frac{1}{t^{k-1}}} \cdot \binom{n}{k} = \bigO{n^{k - \frac{1}{t^{k-1}}}}
\]
edges.
Qualitatively, Theorem~\ref{thm:erdos64-constant-density} states that
we may find a blow-up of an edge as large as we wish, if we let the number of vertices grow
while keeping the density constant.
The following theorem generalizes this notion to blow-ups of arbitrary $k$-graphs.

\begin{theorem} \label{thm:quant-blowup}
    Let $G = (V, E)$ be a $k$-graph and let $\epsilon > 0$.
    There exists a positive integer $n_0 = n_0(G, \epsilon)$ and a constant $\delta = \delta(G, \epsilon) > 0$
    such that for all $n \geq n_0$,
    \[
         \ex{n}{G(t_n)} \leq \left( \pi(G) + \epsilon \right) \binom{n}{k},
    \]
    where
    \[
        t_n = \delta \cdot (\log n)^{\frac{1}{|V|-1}}.
    \]
    \begin{proof}

        We determine the value of $\delta$ later in the proof.
        Suppose that $H$ is a $k$-graph with $n$ vertices and at least $(\pi(G) + \epsilon) \binom{n}{k}$ edges.
        Theorem~\ref{thm:supersaturation} states that if $n \geq n_0 \geq t(G, \epsilon)$,
        there are at least $\delta_1(G, \epsilon) n^{|V|}$ embeddings of $G$ into $H$.

        Consider, as in Remark~\ref{rem:zar_vs_turan},
        a random partition of the vertices of $H$ into $|V|$ parts of size
        $\left\lfloor n / |V| \right\rfloor  \leq |V_i| \leq \left\lceil n / |V|  \right\rceil$.
        Suppose that we have an embedding $f$ of $G$ in $H$.
        Assuming that $n \geq n_0 \geq 2|V|$,
        the probability that for all vertices $v_i \in V$, $f(v_i) \in V_i$ is
        \[
            \prod_{i=1}^{|V|} \frac{|V_i|}{n} \geq \left( \frac{1}{2|V|} \right)^{|V|}.
        \]
        Therefore, for at least one such partition, there are at least
        \[
            \delta_1(G, \epsilon) \cdot \left( \frac{1}{2|V|} \right)^{|V|} \cdot n^{|V|} = \delta_2(G, \epsilon) n^{|V|}
        \]
        embeddings of $G$ in $H$ respecting the partition.
        Furthermore, these embeddings all have different image sets, which have one vertex in each part.
        Consider now the $|V|$-partite $|V|$-graph $H' = (V_1, \dots, V_{|V|}; F)$,
        where
        \[
            F = \{f(V) \mid f \text{ is an embedding of } G \text{ in } H \text{ and } f(V_i) \subset V_i \text{ for all } i\}.
        \]
        By Theorem~\ref{thm:erdos64-constant-density}, making $n_0$ large enough depending only on $\delta_2$ and $|V|$,
        there exists some $\delta_3 = \delta_3(G, \epsilon) = \delta(|V|, \delta_2(G, \epsilon)) > 0$ such that
        $H'$ contains $\compoverset{|V|}{t_n}$ as a subgraph, where
        \[
            t_n = \delta_3(G, \epsilon) \cdot (\log n)^{\frac{1}{|V|-1}}.
        \]
        That is, there are $|V|$ sets $T_1, \dots, T_{|V|}$ of size $t_n$ such that
        if $\{v_{i_1}, \dots, v_{i_k}\} \in E$ is an edge of $G$, then
        $\{u_{i_1}, \dots, u_{i_k}\}$ is an edge of $H_n$ for all $u_{i_j} \in T_j$.
        This means that $G(t_n)$ is a subgraph of $H_n$, so picking $\delta = \delta_3$
        gives the desired result.

    \end{proof}
\end{theorem}

The following corollary highlights that degenerate Turán problems can be applied to solve non-degenerate ones.
\begin{corollary}
    Let $G$ be a $k$-graph and $t$ be a positive integer.
    Then, $\pi(G(t)) = \pi(G)$.
\end{corollary}

In particular, this directly proves the Erdős--Stone--Simonovits theorem for $2$-graphs.

\begin{delayedproof}{thm:erdos_stone_simonovits}
    Let $G$ be a $2$-graph with chromatic number $r$.
    For some $t \geq 1$, $G$ is a subgraph of
    $\compdotssuperindexoverset{2}{r}{t} \cong \completesuperindex{2}{r}(t)$.
    Therefore,
    \[
        \pi(G) \leq \pi\left(\completesuperindex{2}{r}(t)\right)
        = \pi\left(\completesuperindex{2}{r}\right) = 1 - \frac{1}{r-1}.
    \]
    The reverse inequality follows from the same construction in the \proofref{thm:turan} of Turán's theorem.
    Indeed, this construction has the desired density and avoids not only \completesuperindex{2}{r} but also
    any $r$-partite $2$-graph, and in particular $G$.
\end{delayedproof}


Knowing that the Turán density of $k$-partite $k$-graphs is $0$
gives little information on the growth of their Turán numbers.
In this case, we are usually satisfied with determining the growth up to a constant factor.
So far, we have only proven upper bounds for this growth rate.
General lower bounds are usually obtained by probabilistic arguments,
which are often weak and worse that lower bounds obtained by other means (See Subsection~\ref{subsec:open-problems})
for a few examples of this in the literature).
Here we present an example of a general probabilistic argument that applies to any guest graph.

\begin{proposition} \label{prop:probabilistic-lower-bound} % TODO i dont like e for edges
    Let $G = (T_1, \dots, T_k; E)$ be a $k$-graph with
    $t = \sum_{i=1}^{k} t_i = \sum_{i=1}^{k} |T_i|$ vertices
    and $e = |E| > 1$ edges.
    Then, $\ex{n}{G} = \Omega\left(n^{k - \frac{t - k}{e - 1}} \right)$, and the
    constant factor depends only on the number of edges $e$ (and not on the number of vertices $t$).
    \begin{proof}
        Let $n \geq t$.
        We use the so-called \emph{random alteration} method to construct a $k$-graph
        $H_n$ with $n$ vertices that does not contain $G$ as a subgraph.
        We first define $R_n = (V, E)$ to be a random $k$-graph on a vertex set $V$ of size $n$,
        where each edge $e \in \binom{V}{k}$ is included in $E$ independently at random
        with a certain probability $p \in (0, 1)$.
        The expected number of edges in $R_n$ is
        \[
            \mathbb{E}(|E|) = p \binom{n}{k} \geq p \left( \frac{n}{k} \right)^k.
        \]
        Let us now count the number of possible injective functions of $V(G)$ in $V$.
        They are defined by the (ordered) choice of the image of each vertex, so there are
        \[
            \prod_{j=1}^{t} (n - j + 1) \leq n^t
        \]
        of them.
        The probability that any particular injective function $f$ of $V(G)$ in $V$ is an embedding of $G$ in $R_n$
        is calculated as the product of the probabilities that each image of an edge is an edge in $R_n$,
        because the presence of edges in $R_n$ is independent.
        Therefore,
        \[
            \mathbb{P}(f \text{ is an embedding of } G) = p^{|E|} = p^{e}.
        \]
        If we define $X$ to be the number of embeddings of $G$ in $R_n$, by linearity of expectation we get
        \[
            \mathbb{E}(X) = \sum_{f} \mathbb{P}(f \text{ is an embedding of } G) \leq n^t p^{e}.
        \]
        We can now obtain a $G$-free $k$-graph $H_n$ by removing from $R_n$, for each embedding of $G$,
        the image of an edge of $G$.
        The expected number of edges in $H_n$ is
        \[
            \mathbb{E}(|E(H_n)|) = \mathbb{E}(|E|) - \mathbb{E}(X) \geq
            p \left( \frac{n}{k} \right)^k - n^t p^{e}.
        \]
        This quantity is maximized by setting
        \[
            p = \left( \frac{1}{ek^k} n^{k-t} \right)^{\frac{1}{e-1}}.
        \]
        This yields
        \[
            \mathbb{E}(|E(H_n)|) \geq
            m_0(n) =
            \left( \frac{(e-1)^{e-1}}{e^e k^{ek}} \right)^{\frac{1}{e-1}} n^{k - \frac{t-k}{e-1}}
            = \Omega\left(n^{k - \frac{t-k}{e-1}} \right).
        \]
        Therefore, the event that $|E(H_n)| \geq m_0(n)$
        must have positive probability, and in particular,
        there exists one such graph $\widehat{H_n}$, which is $G$-free by construction.
    \end{proof}
\end{proposition}

\subsection{Open Problems}\label{subsec:open-problems} % TODO mention people by name + techniques, remove citations in math environment

In Subsections~\ref{subsec:turan} and~\ref{subsec:degenerate}, we have
seen the solution for non-degenerate Turán problems for $2$-graphs.
Determining the Turán density of $k$-graphs for $k > 2$ is a much harder problem.

Famously, not even the Turán density of the tetrahedron $3$-graph $\completesuperindex{3}{4}$
(pictured in Figure~\ref{fig:complete_kgraph}) or the unique graph $\completesuperindex{3}{4-}$
obtained by removing one edge from it, are known.
The best known bounds are
\[
    0.5555 = \frac{5}{9}
    \leq \pi\left(\completesuperindex{3}{4}\right)
    \leq 0.561666~\cite{keevash2011hypergraph,baber2011hypergraphs} % TODO cites outside math environment?
\]
and
\[
     0.2857 = \frac{2}{7}
     \leq \pi\left(\completesuperindex{3}{4-}\right)
     \leq 0.2871~\cite{frankl1984exact, baber2011hypergraphs}.
\]
The lower bounds are obtained by explicit constructions of $3$-graphs,
and were conjectured to be optimal by Turán~\cite{keevash2011hypergraph},
while the upper bounds are obtained by the method of flag algebras~\cite{razborov2007flag},
which is a powerful tool for studying Turán problems that automates the search for relevant inequalities.

One of the few examples of success in obtaining Turán densities of $k$-graphs with uniformity $k > 2$ is the case of
the Fano plane $F^{(3)}_7$, a $3$-graph with $7$ vertices corresponding to the points
of the projective plane over the field $\mathbb{F}_2$,
and $7$ edges corresponding to the projective lines.
It is known that
\[
    \pi\left(F^{(3)}_7\right) = \frac{3}{4}~\cite{de2000maximum}.
\]

There are even fewer solved cases for degenerate Turán problems than in the non-degenerate case.
In general, there is a very large gap between the upper and lower bounds for the Turán numbers of degenerate $k$-graphs,
even for complete $k$-partite $k$-graphs.
For example, in the balanced case, where all $t_i$ are equal, we get
\begin{equation} \label{eq:balanced_upper_bound}
    \ex{n}{\compoverset{k}{t}} = \bigO{n^{k - \frac{1}{t^{(k-1)}}}}
\end{equation}
from Theorem~\ref{thm:erdos64}, but only
\begin{equation} \label{eq:balanced_lower_bound}
    \ex{n}{\compoverset{k}{t}} = \Omega\left(n^{k - \frac{k(t-1)}{t^k - 1}}\right)
\end{equation}
from Proposition~\ref{prop:probabilistic-lower-bound}.
The exponent in~\eqref{eq:balanced_lower_bound} is always
smaller than the one in~\eqref{eq:balanced_upper_bound},
as long as $t \geq 2$ and $k \geq 2$.

In the case $k=2$, it is known that, for $K(2, t)$ (for $t \geq 2$) and $K(3, 3)$,
Theorem~\ref{thm:kst} is optimal in the sense that
\[
    \ex{n}{K(2, t)}
    = \Theta\left(n^{\frac{3}{2}}\right)~\cite{erdHos1966problem, furedi1996new},
\]
and also
\[
    \ex{n}{K(3, 3)}
    = \Theta\left(n^{\frac{5}{3}}\right)~\cite{brown1966graphs}.
\]
The theorem is also optimal for $K(s, t)$ when $s \geq t! + 1$~\cite{kollar1996norm}.
Some progress has been made in the case of $K(s, t)$ when $s$ and $t$ have similar sizes,
only for small values of $s \geq t \geq 4$.
For example, Theorem~\ref{thm:kst} gives
\[
    \ex{n}{K(5,5)} = \bigO{n^{\frac{9}{5}}} = \bigO{n^{1.8}},
\]
and by Proposition~\ref{prop:probabilistic-lower-bound} we get
\begin{equation} \label{eq:k55-probabilistic}
    \ex{n}{K(5,5)}
= \Omega\left(n^{\frac{5}{3}}\right)
= \Omega\left(n^{1.67}\right),
\end{equation}
but~\eqref{eq:k55-probabilistic} has been improved to
\[
    \ex{n}{K(5,5)}
    = \Omega\left(n^{\frac{7}{4}}\right)
    = \Omega\left(n^{1.75}\right)~\cite{ball2012asymptotic}.
\]
Even less is known about degenerate problems for graphs of higher uniformity.
For example, not even the growth rate of the Turán number for the octahedron 3-graph
($\ex{n}{K(2, 2, 2)}$, pictured in Figure~\ref{fig:222}) is known.
The best upper bound, again, comes from Theorem~\ref{thm:erdos64}, which gives
\[
    \ex{n}{K(2, 2, 2)} = \bigO{n^{\frac{11}{4}}} = \bigO{n^{2.75}},
\]
while the best know lower bound is
\[
    \ex{n}{K(2, 2, 2)}
    = \Omega\left(n^{\frac{8}{3}}\right)
    = \Omega\left(n^{2.67}\right)~\cite{conlon2020random}.
\]

The main difficulty for degenerate problems is that sharp lower bounds for the Turán numbers
often rely on specific geometric or algebraic constructions that work for very few cases,
such as the ones cited for $K(2, 2)$ and $K(3, 3)$.

Theorem~\ref{thm:quant-blowup} is known not to be optimal.
Erdős and Bollobás~\cite{bollobas1973structure} in fact proved that the optimal growth
rate of a guaranteed $t_n$-blow-up of a $2$-graph $G$ in $2$-graphs of constant density greater than $\pi(G) + \epsilon$ is
\[
    t_n = \delta(G, \epsilon) \cdot \log n.
\]
A still open question is whether this can be extended to $k$-graphs.
That is, is it true that $k$-graphs with $n$ vertices and  $\left(\pi(G) + \epsilon \right)  \binom{n}{k}$ edges
contain $G(t_n)$ for some $t_n = \delta(G, \epsilon) (\log n)^{\frac{1}{k-1}}$?
An even more general yet unresolved question is whether this is true for
$k$-graphs with $n$ vertices and $\Omega (n^{|V(G)|})$ copies of $G$~\cite{rodl2012complete}.