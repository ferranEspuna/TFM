%! Author = Ferran

% Preamble
\documentclass[12pt]{article}

% Packages
\usepackage{amsmath}
\usepackage{parskip}
\usepackage[margin=70pt]{geometry}
\usepackage{amsthm}
\usepackage{mathtools}
\usepackage{xcolor}
\author{Ferran Espuña}
\date{} % clear date


\newtheorem{thm}{Theorem}
\newtheorem{lemma}[thm]{Lemma}


\theoremstyle{remark}
\newtheorem{remark}[thm]{Remark}




% Document
\begin{document}

    \title{Extending Mubayi and Turán's Algorithm to $k$-graphs}

    \maketitle

    Let $G$ be a $k$-graph with $n$ vertices and $m = dn^k$ edges.
    A polynomial time algorithm is given to find a $K_{q,\, \ldots\, ,\, q} \subset G$ for

    \begin{align*}
        q (k, d) = \left\lfloor \left(  \frac{\log (n/2^k)}{\log (2^{k+1}/d)} \right)^{\frac{1}{k-1}} \right\rfloor
        \label{eq:q}
    \end{align*}

    As long as $d \geq 4^k n^{-\frac{1}{2^{k-1}}}$.

    \begin{remark}
        This minimum density requirement is not a big restriction.
    \end{remark}

    This result is a generalization of the result in $2$-graphs by~\cite{MUBAYI2010174},
    and algorithm will be analogous to the one given there.
    This algorithm, referred to as \verb|FIND_PARTITE|$(k, \cdot)$,
    is the one described in~\cite{MUBAYI2010174}
    when $k=2$, and for $k \geq 3$ involves the following steps:

    \begin{enumerate}
        \item Choose parameters $q, r, s$ depending on $n, k$ and $d$. \label{step:choose_parameters}

        \item Find the set $R$ of $r$ vertices with the highest degree in $G$. \label{step:find_R}

        \item find a subset $Q \subset R$ with $q$ vertices and a
        $S \subset T \coloneqq \binom{[n] \setminus Q}{k-1}$ with $s$ edges satisfying
        \[\{x_1, x_2, \ldots, x_k\} \in E(G) \; \forall \, \{x_2, \ldots, x_k\} \in S, \, x_1 \in Q\] \label{step:find_Q_S}

        \item The set $S$ induces a $(k-1)$-graph $G'$ on $T$.
        Evaluate \verb|FIND_PARTITE|$(k-1, G')$ to find a $K_{q',\, \ldots\, ,\, q'}$ in $G'$
        (say, $H' = \{U_1, \ldots, U_{k-1}\}$). It will turn out that $q' \geq q$,
        and because of the condition for $S$ the $k$-partite subgraph $H = \{Q, V_1, \ldots V_{k-1}\}$
        is complete in $G$, where $V_i \subset U_i$ is a subset of size $q$.
        Return $H$.

    \end{enumerate}

    For step~\ref{step:choose_parameters}, we will use the following formulas:

    \begin{align*}
                    q (k, d) = \left\lfloor \left(  \frac{\log (n/2^k)}{\log (2^{k+1}/d)} \right)^{\frac{1}{k-1}} \right\rfloor,\,
            r(k, d) = \left \lceil \frac{2q(k, d)}{d} \right \rceil,\,
            s(k, d) = \left\lfloor d^{q(k, d)} n^{k-1} \right\rfloor
    \end{align*}

    The goal is to prove that the algorithm is successful and runs in polynomial time.

    \begin{lemma}
        This selection of parameters, \color{red} for $k \geq 3$, \color{black} is sound in the sense that $q  \leq r \leq n$, $k - 1 \leq n - r$ and $s \leq \binom{n - r}{k - 1}$.

        \begin{proof}
            $q \leq r$ is clear from the definition of $r$.
            We will show that in fact $r < \frac{n}{2}$.
            Suppose not:
            \[
                \frac{n}{2} \leq r \leq 1 + \frac{2q}{d} \leq 1 + \frac{2 \log (n/2)}{2^{k+1}n^{-\frac{1}{2^{k-1}}}}
                \leq 1 + \frac{\log n \cdot \sqrt {n}}{8} \leq 1 + \frac{n}{8} \implies n < 3
            \]
            Which is a contradicion, as there cannot be any edges and thus $d = 0$.
            In particular, $r \leq n$.
     
            We can similarly show that $k \leq \frac{n}{2}$.
            Suppose not:
            
            \[
                1 \geq d \geq 2^{\frac{n}{2}} n^{- \frac{1}{2^{n/2}-1}}
                = n^{\frac{\log(2) n}{2} - \frac{1}{2^{n/2}-1}} \implies
                \frac{\log (2) n}{2} \leq \frac{1}{2^{n/2}-1} \implies n < 3
            \]

            Therefore, $k+r \leq n < n + 1 \implies k - 1 \leq n - r$.

            Finally, suppose that $s > \binom{n - r}{k - 1}$.
            Then,

            \begin{align*}
                \left( \frac{n}{2k} \right)^{k-1} \leq \left( \frac{n-r}{k-1} \right)^{k-1} \leq \binom{n - r}{k - 1} < s \leq d^q n^{k-1}
                \implies  \left( \frac{1}{2k} \right)^{k-1} < d^q \leq \left( \frac{1}{k!} \right)^2
            \end{align*}
            
            Where in the last inequality we use that $q \geq 2$ and there are at most
            $\binom{n}{k} \leq \frac{n^k}{k!}$ edges in $G$.

            We can show that \color{red} $k!^2 \geq (2k)^{k-1}$ for all $k$ % TODO proof by freaking look at it!!
            \color{black}, which means we have reached a contradiction.

        \end{proof}

    \end{lemma}

    \begin{lemma}
        With the above restrictions and choice of parameters, \color{red} for $k \geq 3$, \color{black}
        there are at least $\frac{3}{2}drn^{k-1}$
        edges with exactly one vertex in $R$.

        \begin{proof}

            The degree sum over $V(G)$ is $kdn^{k}$.
            Thus, by the pigeonhole principle, the degree sum over $R$ is at least
            $\frac{r}{n}kdn^{k} = rkdn^{k-1}$.
            For $2 \leq j \leq n$, consider the contribution to this sum by edges with exactly $j$ vertices in $R$.
            Each such edge contributes $j$ to the sum, and there are at most
            $\binom{r}{j}\binom{n-r}{k-j} \leq \frac{r^j n^{k-j}}{j!} \leq \frac{r^j n^{k-j}}{j}$ of them.
            Thus, the total contribution of these edges is at most $r^j n^{k-j} \leq r^{2}n^{k-2}$.
            The number of edges we want is then at least

            \[
                rkdn^{k-1} - (k-1)r^{2}n^{k-2} = drn^{k-1} \left( k - \frac{(k-1)r}{nd}\right)
            \]
            
            Suppose, by way of contradiction,
            that $ k - \frac{(k-1)r}{nd} < \frac{3}{2}$.
            Using that $\frac{k-1}{k-3/2} \leq 2$
            for $k \geq 2$, we arrive at

            \[
                d < 2rn^{-1} \leq \frac{3q}{nd} \implies nd^2 < 3 \log n
            \]
            
            Applying our minimum density, this means
            
            \[
                \sqrt {n} \leq 2^{2k+1} n^{1 - \frac{1}{2^{k-2}}} < 3 \log n
            \]

            which is false for all $n$.
        \end{proof}
    \end{lemma}

    \begin{lemma}\label{lemma:q_s}
        For this selection of parameters,
        there exist sets $Q, S$ as described in step~\ref{step:find_Q_S} of the algorithm.

        \begin{proof}

            Consider the biparite graph with vertex set $\left( R, \binom{T}{k-1} \right)$
            and edges corresponding to edges in $G$ with exactly one vertex in $R$ (and thus all others in $T$).
            The sets $Q$ and $S$ we want to find correspond to a complete bipartite subgraph of this graph
            with parts of size $q$ and $s$ respectively.
            Suppose that such a subgraph does not exist.
            ~\cite{Kovari1954} tells us then that

            \begin{align*}\label{eq:equation}
                \frac{3}{2}drn^{k-1} & < z\left(\binom{n-r}{k-1}, r; s, q  \right) < (s-1)^{1/q}(r-q+1)\binom{n-r}{k-1}^{1-1/q} + (q-1)\binom{n-r}{k-1} \\
                           & \leq s^{1/q} r \binom{n}{k-1}^{1-1/q} + q \binom{n}{k-1} \leq s^{1/q} r \binom{n}{k-1}^{1-1/q} + \frac{1}{2} drn^{k-1}
            \end{align*}

            Where the last inequality follows from our choice of $r$.

            Rearranging and approximating the binomial coefficient, we get
            \begin{align*}
                drn^{k-1} < s^{1/q}r n^{(k-1)(1-1/q)} \iff d < \left( \frac{s}{n^{k-1}} \right)^{1/q}
            \end{align*}

            Which is false for the given choice of $s$.

        \end{proof}

    \end{lemma}

    \begin{lemma} \label{lemma:q_prime}
        For this choice of parameters, the number of edges $s$ in $G'$
        is such that we can apply the algorithm to $G'$.
        Furthermore, the resulting $q'$ satisfies $q' \geq q$.

        \begin{proof}

            First we calculate a lower bond for the corresponding edge density $d'$ in $G'$:

            \begin{equation*}
                d' = \frac{s}{(n-r)^{k-1}} \geq \frac{\left(  n^{k-1} d^{q} - 1\right)}{n^{k-1}}
                \geq d^q - n^{1-k}
            \end{equation*}

            Note however that because $1 \geq d \geq 2^{k+1} n^{-\frac{1}{2^{k-1}}}$ and $q \geq 2$, we have

            \[
                d^q \geq 2^{q(k+1)}n^{-\frac{q}{2^{k-1}}} \geq
                2^{k+1} n^{-\frac{1}{2^{k-1}} \left( \frac{\log n}{\log (2^{k+1}/d)} \right)^{1/(k-1)}}
                \geq 2^{k+1} n^{-\frac{1}{2^{k-1}} \left( \frac{\log n}{2^{1-k} \log (n)} \right)^{1/(k-1)}}
                = 2^{k+1}n^{-\frac{1}{2^{k-2}}}
            \]

            Now, clearly this means that $n^{1-k} \leq \frac{1}{2} d^q \implies d' \geq \frac{1}{2}d^q
            \implies d' \geq 2^{k}n^{-\frac{1}{2^{k-2}}}$,
            satisfying our minimum density requirement for $k-1$.
            Furthermore, we can bound


            \begin{align*}
                q' \geq &
                \color{blue} \left\lfloor \left(  \frac{\log ((n-r)/2^{k-2})}{\log (2 \cdot 2^{(k-1)+1}/d^q)} \right)^{\frac{1}{k-2}} \right\rfloor
                \geq \left\lfloor \left(  \frac{\log (n/2^{k-1})}{\log (2 \cdot 2^{(k-1)+1}/d^q)} \right)^{\frac{1}{k-2}} \right\rfloor =
                \left\lfloor \left(  \frac{\log (n / 2^{k-1})}{(k+1) \log 2 - q \log d} \right)^{\frac{1}{k-2}} \right\rfloor \\
                \color {blue}\geq &
                \color {blue} \left\lfloor \left(  \frac{\log (n / 2^{k-1})}{(k+1) \log 2 - \left(  \frac{\log (n / 2^{k-1})}{\log (2^{k+1}/d)} \right)^{\frac{1}{k-1}}  \log d} \right)^{\frac{1}{k-2}} \right\rfloor
                =
                \left\lfloor \left(  \frac{\log (n / 2^{k-1})^{1-\frac{1}{k-1}}}{\frac{(k+1) \log 2}{\log(n / 2^{k-1})^{\frac{1}{k-1}}} - \frac{\log d}{\log (2^{k+1}/d)^{\frac{1}{k-1}}} } \right)^{\frac{1}{k-2}} \right\rfloor \\
                \geq &
                \left\lfloor  \frac{(\log n)^{\frac{
                    1}{k-1}}}{\left(\frac{\log (2^{k+1}/d)}{\log (2^{k+1}/d)^{\frac{1}{k-1}}} \right)^{\frac{1}{k-2}} } \right\rfloor
                = \left\lfloor  \frac{(\log n)^{\frac{1}{k-1}}}{\left(\log (2^{k+1}/d)\right)^{(1-\frac{1}{k-1})\frac{1}{k-2}}} \right\rfloor
                = \left\lfloor \left(  \frac{\log n}{\log (2^{k+1}/d)} \right)^{\frac{1}{k-1}} \right\rfloor = q
            \end{align*}

            where the last inequality follows from the fact that $n \geq 2^{k+1}/d$,
            which is a consequence of our minimum density requirement.

        \end{proof}

    \end{lemma}

    \bibliography{main}
    \bibliographystyle{plain}

\end{document}
