%! Author = Ferran

% Preamble
\documentclass[12pt]{article}

% Packages
\usepackage{amsmath}
\usepackage{parskip}
\usepackage[margin=70pt]{geometry}
\usepackage{amsthm}
\usepackage{mathtools}
\usepackage{xcolor}
\author{Ferran Espuña}
\date{} % clear date


\newtheorem{thm}{Theorem}
\newtheorem{lemma}[thm]{Lemma}



% Document
\begin{document}

    \title{Extending Mubayi and Turán's Algorithm to $k$-graphs}

    \maketitle

    Let $G$ be an $r$-graph with $n$ vertices and $m = dn^k$ edges.
    A polynomial time algorithm is given to find a $K_{q,\, \ldots\, ,\, q} \subset G$ for

    \begin{align*}
        q (k, d) = \left\lfloor \left(  \frac{\log n}{\log (d/2^k)} \right)^{\frac{1}{k-1}} \right\rfloor
        \label{eq:q}
    \end{align*}

    As long as (insert condition here). % TODO: Insert condition
    
    Note that this result is tight up to the constant $c(k, d)$,
    as proved in~\cite{Erods1964}.
    This result is a generalization of the result in $2$-graphs by~\cite{MUBAYI2010174},
    and algorithm will be analogous to the one given there.
    We will call this algorithm \verb|FIND_PARTITE|$(k, \cdot)$.
    The procedure is as follows:

    \begin{enumerate}
        \item Choose parameters $q, r, s$ depending on $n, k$ and $d$. \label{step:choose_parameters}
        \item Find the set $R$ of $r$ vertices with the highest degree in $G$. \label{step:find_R}

        \item find a subset $Q \subset R$ with $q$ vertices and a
        $S \subset T \coloneqq \binom{[n] \setminus Q}{k-1}$ with $s$ edges satisfying
        \[\{x_1, x_2, \ldots, x_k\} \in E(G) \; \forall \, \{x_2, \ldots, x_k\} \in S, \, x_1 \in Q\] \label{step:find_Q_S}

        \item The set $S$ induces a $(k-1)$-graph $G'$ on $T$.
        Evaluate \verb|FIND_PARTITE|$(k-1, G')$ to find a $K_{q',\, \ldots\, ,\, q'}$ in $G'$
        (say, $\{U_1, \ldots, U_{k-1}\}$). We will prove that $q' \geq q$,
        and because of the condition for $S$ we can return $\{Q, V_1, \ldots V_{k-1}\}$
        where $V_i$ is a subset of $U_i$ of size $q$.

    \end{enumerate}

    For step~\ref{step:choose_parameters}, we will use the following formulas:

    \begin{align*}
            q(k, d) &= \left\lfloor \left(  \frac{\log n}{\log (d/2^k)} \right)^{\frac{1}{k-1}} \right\rfloor \\
            r(k, d) &= \left \lceil \frac{2q(k, d)}{d} \right \rceil \\
            s(k, d) &= \left\lfloor \left( \frac{d}{2} \right)^{q(k, d)} n^{k-1} \right\rfloor
    \end{align*}

    Our goal is to prove that the algorithm is successful and runs in polynomial time.

    \begin{lemma}\label{many_edges}There exist sets $Q, S$ as described in step~\ref{step:find_Q_S} of the algorithm.

        \begin{proof}
            We first show that ther are at least $drn^{k-1}$ edges in $G$ with exactly one vertex in $R$.
            inded, $\color{red}\ldots$ % TODO add minimum density condition and prove this with it

            Now, consider the biparite graph with vertex set $\left( R, \binom{T}{k-1} \right)$
            and edges corresponding to edges in $G$ with exactly one vertex in $R$ (and thus all others in $T$).
            The sets $Q$ and $S$ we want to find correspond to a complete bipartite subgraph of this graph
            with parts of size $q$ and $s$ respectively.
            Suppose that such a subgraph does not exist.
            ~\cite{Kovari1954} tells us then that

            \begin{align*}\label{eq:equation}
                drn^{k-1} & < z\left(\binom{n-r}{k-1}, r; s, q  \right) < (s-1)^{1/q}(r-q+1)\binom{n-r}{k-1}^{1-1/q} + (q-1)\binom{n-r}{k-1} \\
                           & \leq s^{1/q} r \binom{n}{k-1}^{1-1/q} + q \binom{n}{k-1} \leq s^{1/q} r \binom{n}{k-1}^{1-1/q} + \frac{1}{2} drn^{k-1}
            \end{align*}

            Where the last inequality follows from $\color{red} \ldots$

            Rearranging and approximating the binomial coefficient, we get
            \begin{align*}
                drn^{k-1} &< 2s^{1/q}r n^{(k-1)(1-1/q)} \implies d < 2 \left( \frac{s}{n^{k-1}} \right)^{1/q}
            \end{align*}

            Which is false for our election of $s$.



        \end{proof}

    \end{lemma}








    \bibliography{main}
    \bibliographystyle{plain}

\end{document}
