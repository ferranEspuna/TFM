\section{Conclusions and Future Work}\label{sec:conclusions} % TODO write this myself

This thesis has focused on the algorithmic aspects of finding $k$-partite subgraphs in $k$-uniform hypergraphs,
a problem central to degenerate Turán theory.
We have presented a deterministic,
polynomial-time algorithm (Algorithm~\ref{alg:kpartite}) that, given a $k$-graph $G$ on $n$ vertices with $m$ edges,
finds a complete balanced $k$-partite $k$-subgraph $\compoverset{k}{t}$.
The part size $t$ achieved, given by Equation~\eqref{eq:t} as $t \approx (\log n / \log(1/d))^{1/(k-1)}$ where $d=m/n^k$,
closely matches the parameters implicit in the non-constructive existence proofs of Erd\H{o}s for such structures.
This provides an efficient, constructive counterpart to these classical results,
demonstrating that these subgraphs can indeed be located algorithmically within polynomial time.
The recursive approach, which reduces the uniformity $k$ by analyzing appropriately defined link graphs,
generalizes previous work for the $k=2$
case by Mubayi and Turán.
There are several avenues for future research stemming from this work:

\begin{enumerate}
    \item \textbf{Tightening Bounds and Improving Practicality:}
    The proofs of correctness for Algorithm~\ref{alg:kpartite}, particularly Lemmas~\ref{lm:sound} through~\ref{lm:t_prime},
    involve several inequalities.
    While sufficient to establish the polynomial runtime and the asymptotic nature of $t$,
    some of these bounds are quite loose (e.g., approximations of binomial coefficients, conditions for $w \leq n/2$,
    or the constants in the density arguments).
    A more meticulous analysis could potentially yield sharper constants in the definition of
    $t(n,d,k)$ or relax the minimum density requirements (Remark~\ref{rm:min_d}).
    This could make the algorithm applicable to sparser hypergraphs or guarantee larger $k$-partite structures for a given density,
    enhancing its practical significance for analyzing real-world hypergraphs which might not meet the currently proven,
    somewhat high, minimum vertex or density thresholds.

    \item \textbf{Generalizing to unbalanced partite hypergraphs:}
    This approach taken in the proof of Theorem~\ref{thm:erdos64}
    can be generalized to give a lower bound on the number of
    copies (that is, embeddings with different image sets)
    of $\compdots{t_1}{t_k}$ in a $k$-partite $k$-graph $G$
    with different part sizes~\cite{carvajal2024canonical},
    therefore upper bounding all generalized Zarankiewicz numbers.
    Applying the same observations that we have made for the balanced case,
    we arrive at
    \begin{equation} \label{eq:general-partite-bound}
        \ex{n}{\compdots{t_1}{t_k}} = \bigO{n^{k - \frac{1}{\prod_{i=1}^{k-1} t_i}}}.
    \end{equation}
    An algorithm finding these more general $k$-partite hypergraphs
    of growing sizes could be obtained by modifying the current algorithm
    and, for example, fixing the approximate ratio between the part sizes to be found.

    \item \textbf{Exploring what happens when the density is not fixed:}
    There are versions of Theorem~\ref{thm:erdos64-constant-density}
    that make $n_k$ depend on $\epsilon$ and show that the size of the
    guaranteed complete $k$-partite $k$-graph is at least $\delta n^{k-\frac{1}{t^{k-1}}}$.
    This shows Algorithm~\ref{alg:kpartite} may not return a complete balanced $k$-partite
    $k$-graph of the best possible order of magnitude for a family of graphs with density tending to $1$.
    An interesting question is whether the algorithm can be modified to find
    a partite hypergraph of approximately the right size in these circumstances,
    maybe losing the polynomial time complexity when this size is no longer logarithmic in $n$.

    \item \textbf{Finding Blow-ups of General $k$-Graphs:}
    The presented algorithm is tailored to find blow-ups of a single edge, i.e., $\compoverset{k}{t}$.
    A natural extension would be to adapt this algorithmic framework to find $t_n$-blowups $H(t_n)$ of an arbitrary fixed $k$-graph $H$.
    As discussed in Section~\ref{subsec:degenerate} (Theorem~\ref{thm:quant-blowup}), $k$-graphs with density $\pi(G) + \epsilon$
    are known to contain $G(t_n)$ where $t_n = \delta (\log n)^{1/(|V(G)|-1)}$.
    Our current algorithm, if adapted, might yield a constructive proof for finding such blow-ups.
    However, for $k=2$, it is known from Bollobás and Erd\H{o}s~\cite{bollobas1973structure} that the optimal growth for $t_n$
    is $\delta \log n$, which is better than $(\log n)^{1/(|V(G)|-1)}$ if $|V(G)| > 2$.
    It remains open whether there is a way to adapt their proof into a constructive one,
    providing a polynomial-time algorithm.


    \item \textbf{Implementation and Experimental Evaluation:}
    Implementing Algorithm~\ref{alg:kpartite}
    and evaluating its performance on various synthetic and real-world hypergraph datasets would be valuable.
    This could help identify practical bottlenecks and compare its findings with theoretical guarantees,
    especially concerning the constants involved in the calculation of $t$.
\end{enumerate}
