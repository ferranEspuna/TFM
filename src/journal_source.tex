\maketitle

\begin{abstract}
We give a deterministic polynomial-time algorithm that, for a given $k$-uniform hypergraph $H$ with $n$ vertices and edge density $d$,
finds a $\compoverset{k}{t}$ subgraph with parts of size at least $c_d (\log n)^{1/(k-1)}$,
building on work by Mubayi and Tur\'{a}n for the $k=2$ case.
This value for the part size matches the order of magnitude guaranteed by the non-constructive proof due to Erd\H{o}s and is tight up to a constant factor.
\end{abstract}

\section{Introduction}\label{sec:introduction}

Hypergraph Tur\'{a}n problems study how many edges a $k$-uniform hypergraph $H = (V, E)$ with $n$ vertices can have without containing a specific subgraph $G$.
The maximal such number is known as the \emph{Tur\'{a}n number} $\ex{n}{G}$.
It is known~\cite{keevash2011hypergraph}
that $\ex{n}{G} = o\left( \binom{n}{k}\right)$ if and only if $G$ is $k$-partite, i.e.,
if its vertex set can be partitioned into $k$ disjoint sets such that each edge contains exactly one vertex from each part.
Kőv\'{a}ri, S\'{o}s, and Tur\'{a}n~\cite{Kovari1954} (for $k=2$) and
Erd\H{o}s~\cite{Erods1964} (for any $k \geq 2$) established that
\[ \label{eq:erdos64-intro}
    \ex{n}{\compoverset{k}{t}} = \bigO{n^{k - \frac{1}{t^{(k-1)}}}},
\]
where $\compoverset{k}{t}$ is the complete balanced $k$-partite $k$-graph with $k$ parts of size $t$.
Furthermore, if $H$ is a $k$-graph with at least $d \binom{n}{k}$ edges for some constant $d > 0$, then it contains a $\compoverset{k}{t}$ with
$t = c_d \log(n)^{1/(k-1)}$.

This result is non-constructive, meaning it guarantees the existence of such a subgraph but does not provide an efficient way to find it.
Note that a simple brute-force search for a $\compoverset{k}{t}$ would involve checking all $\binom{n}{kt}$ vertex subsets, which is superpolynomial in $n$ for $t = \Theta((\log n)^{1/(k-1)})$.
Mubayi and Tur\'{a}n~\cite{MUBAYI2010174} developed a polynomial-time algorithm for the case $k=2$, which reaches the stated order of magnitude for the subgraph part size.
This work extends their approach to the general case of $k$-uniform hypergraphs, reaching analogous results for $k \ge 3$.
More concretely, we prove the following.

\begin{theorem} \label{thm:main_theorem}
There is a deterministic algorithm that, given a $k$-graph $H$ with $n$ vertices and $m=d \binom{n}{k}$ edges, finds a complete balanced $k$-partite subgraph $\compoverset{k}{t}$ in polynomial time, where
\[
    t = t(n, d, k) = \left\lfloor \left( \frac{\log n}{\log (16/d)}  \right)^{\frac{1}{k-1}} \right \rfloor
\]
\end{theorem}
This value of $t$ matches the order of magnitude from existence proofs.
In fact, a probabilistic argument shows that it is the best possible up to a constant factor.

\section{The algorithm}\label{sec:finding-a-balanced-$k$-partite-subgraph}

We present a recursive algorithm, \texttt{FindPartite}, that finds a $\compoverset{k}{t}$ in a given $k$-graph $H$.
The core idea is to reduce the uniformity of the problem from $k$ to $k-1$ in each recursive step.
The algorithm takes a $k$-graph $H$ with $n$ vertices and $m$ edges as input.
It first defines the target part size $t$, a small set size $w$,
and a threshold edge count $s$ for the recursive call, based on the input graph's parameters:

\begin{align*}
    t &= t(n, d, k) = \left\lfloor \left( \frac{\log n}{\log (16/d)}  \right)^{\frac{1}{k-1}} \right \rfloor, \\
    w &= w(n, d, k) = \left\lceil \frac{4 t}{d} \right\rceil, \text{ and } \\
    s &= s(n, d, k) = \left\lceil \left( \frac{d}{4} \right)^t \binom{n}{k-1} \right\rceil,
\end{align*}
where $d = \frac{m}{\binom{n}{k}}$ is the edge density of $H$.
The main steps are:
\begin{enumerate}
    \item \textbf{Base Case ($k=1$):} The edge set of a 1-graph is just a collection of vertices.
    Return the set of all vertices that are ``edges''.

    \item \textbf{Select High-Degree Vertices:} Choose a set $W \subset V$ of $w$ vertices with the highest degrees in $H$. \label{W}

    \item \textbf{Find a Dense Link Graph:} Iterate through all $t$-subsets $T \subset W$.
    For each $T$, consider the set $S$ of all $(k-1)$-subsets of $V$ that form a hyperedge with \emph{every} vertex in $T$. \label{link}

    \item \textbf{Recurse:} As we prove further along using the Kőv\'{a}ri–S\'{o}s–Tur\'{a}n theorem, for at least one choice of $T$,
    the resulting set $S$ will be large ($|S| \ge s$). We form a new $(k-1)$-graph $H'=(V, S)$ and make a recursive call: \texttt{FindPartite($H'$, $k-1$)}.

    \item \textbf{Construct Solution:} The recursive call returns $k-1$ parts $V_1, \dots, V_{k-1}$ of size at least $t$.
    By construction, every choice of vertices from these parts forms an edge in $H'$ with every vertex of $T$.
    Thus, $(T_1, \dots, T_{k-1}, T)$ form the desired $\compoverset{k}{t}$ in the original graph $H$.

\end{enumerate}

The pseudocode is given in Algorithm~\ref{alg:kpartite}.

\begin{algorithm}[H]
    \caption{Finding a balanced partite $k$-graph}
    \label{alg:kpartite}
    \begin{algorithmic}[1]
        \Function{FindPartite}{$H, k$}
            \If {$k = 1$}
                \State \Return $(\{x \colon \{x\} \in E(H)\})$
            \EndIf

            \State $n \gets |V(H)|$, $m \gets |E(H)|$, $d \gets \frac{m}{\binom{n}{k}}$
            \State $t \gets t(n, d, k)$, $w \gets w(n, d, k)$, $s \gets s(n, d, k)$
            \State \textbf{assert} $t \ge 2$

            \State $W \gets$ a set of $w$ vertices with highest degree in $H$
            \ForAll{$T \in \binom{W}{t}$}
                \State $S \gets \{\,y \in \binom{V}{k-1} \colon \forall x \in T, \{x\} \cup y \in E(H)\,\}$
                \If{$|S| \ge s$}
                    \State $H' \gets (V, S)$  \Comment{$H'$ is a $(k-1)$-graph}
                    \State $(V_1, \dots, V_{k-1}) \gets$ \Call{FindPartite}{$H', k-1$} \label{recurse}
                    \State \Return $(V_1, \dots, V_{k-1}, T)$
                \EndIf
            \EndFor
        \EndFunction
    \end{algorithmic}
\end{algorithm}

We now present the proof of correctness and polynomial runtime for our algorithm.
We assume $t \ge 2$ for our estimates to be easier.
If $t < 2$, we may just return the vertices of any single edge in $H$.

\section{Proof of correctness}\label{sec:correctness}

It is not immediately clear that the set $W$ defined in step~\ref{W} of the algorithm
is well-defined, as for this it is necessary that $w \leq n$.
To show this, we first observe that our assumption $t \geq 2$
implies that $ 1 \geq d \geq \frac{16}{\sqrt {n}}$.
Suppose, by way of contradiction, that $w > n$.
Then, we have
\[
    n \leq w - 1 \leq \frac{4t}{d} \leq \frac{4\log n}{d\log(16/d)} \leq \frac{4 \sqrt {n} \log n }{16 \log(16/d)}.
\]
Taking, for example, the logarithms to be in base $e$, we note that $\log x \leq \sqrt {x}$ for all positive $x$,
and that $\log(16/d) \geq \log(16) > 1$.
Therefore, we get $n \leq \frac{n}{4}$, which is a contradiction.

Next, we will prove that in step~\ref{link} of the algorithm
we indeed find a set $T \in \binom{W}{t}$ such that the associated set
$S \subset \binom{V}{k-1}$ has size at least $s$.
That is, Algorithm~\ref{alg:kpartite} reaches line~\ref{recurse} at some point in the for loop.
For this, consider the bipartite graph $B$ with parts $\binom{V}{k-1}$ and $W$ with edge set
\[
  \left\{(x, y) \in \binom{V}{k-1} \times W \middle| \, x \cup \{y\} \in E \right\}.
\]
The edges of $B$ correspond to the edges containing each vertex in $W$, so there are
\[
    z = \sum_{y \in W} d_H(y) \geq k \cdot m \cdot \frac{w}{n} = \frac{k \cdot w \cdot d \cdot \binom{n}{k}}{n} = w \cdot d \cdot \binom{n - 1}{k-1}
\]
of them, where the inequality follows from the fact that we have picked a set of $w$ vertices with highest degree in $H$.
The existence of a set $T \subset W$ as desired is equivalent to there being $T \subset W$ of size $t$ and a set $S \subset \binom{V}{k-1}$ of size $s$
such that the induced bipartite subgraph $B[S, T]$ is complete.
To prove that this is the case,
we use a version the Kőv\'{a}ri–S\'{o}s–Tur\'{a}n theorem~\cite{Kovari1954},
which we state and prove here for completeness.

\begin{lemma}\label{thm:kst}
    Let $u, w, s, t$ be positive integers with $u \geq s$, $w \geq t$, and let $B$ be a bipartite graph with parts $W$ and $U$ such that
    $|U| = u, |W| = w$.
    If $B$ has more than \[(s - 1)^{1 / t}(w - t + 1)u^{1 - 1 / t} + (t - 1)u\] edges, then there are
    $T \subset W$ of size $t$ and $S \subset U$ of size $s$ such that the induced bipartite subgraph $B[T, S]$ is complete.
\end{lemma}

We apply this lemma with $u = \binom{n}{k-1}$.
It is clear from the definitions that our parameter satisfy the requirements $u \geq s$ and $w \geq t$.
Suppose, by way of contradiction, that
\[
    w \cdot d \cdot \binom{n - 1}{k-1} \leq (s - 1)^{1 / t}(w - t + 1)\binom{n}{k-1}^{1 - 1 / t} + (t - 1)\binom{n}{k-1}.
\]
Algebraic manipulation then shows that
\[
    \frac{1}{2} \cdot w \cdot d
    \leq w \cdot d \cdot \left( 1 - \frac{k}{n} \right)
    \leq w \left( \frac{s-1}{\binom{n}{k-1} } \right)^{1 / t} + (t - 1),
\]
where the first inequality follows from $n \geq 2k$, which follows from $t \geq 2$. % TODO: check this
Finally, since $t \leq \frac{w \cdot d}{4}$ by the definition of $w$, we obtain
\[
    \left( \frac{d}{4}\right)^t \binom{n}{k-1} < s-1,
\]
against the definition of $s$.
We are now ready to prove that the algorithm returns a $\compoverset{k}{t}$.
More precisely, we show the following.

\begin{theorem}
    For $k \geq 2$, if $t \geq 2$, Algorithm~\ref{alg:kpartite} returns a tuple $(V_1, \dots, V_k)$ of disjoint sets $V_i \subset V(H)$ such that
    $|V_i| \geq t$ and $H[V_1, \dots, V_k]$ is complete.

    \begin{proof}
        We proceed by induction on $k$.
        For $k=2$, the recursive call returns the common neighborhood $V_1$ of the vertices in $T$,
        which is obviously disjoint from $T$, so it only remains to check that $|V_1| \geq t$.
        Now, since by construction $|V_1| = |S| \geq s$, it is enough that
        \[
            s
            = \left\lceil \left( \frac{d}{4} \right)^t n \right\rceil
            \geq \left( \frac{d}{4} \right)^{\frac{\log n}{\log(16/d)}} n
            = \frac{1}{n} \cdot 4^{\frac{\log n}{\log(16/d)}} \cdot n
            \geq 4^t
            > t.
        \]

        For $k \geq 3$, we assume the inductive hypothesis holds for $k-1$.
        If $d'$ is the edge density of the $(k-1)$-graph $H'$
        and $t' = t(n, d', k-1)$, as long as $t' \geq 2$, the recursive call returns
        a tuple $(V_1, \dots, V_{k-1})$ of disjoint sets $V_i \subset V(H)$ such that
        $|V_i| \geq t'$ and $H'[V_1, \dots, V_{k-1}]$ is complete.

        We claim that $t' \geq t$.
        This implies that $t' \geq 2$ so we get to apply the inductive hypothesis to $H'$.
        Furthermore, the sets $V_i$ are at least as large as desired.
        By construction of $H'$, for all $(x_1, \dots x_{k-1}, y) \in V_1 \times \dots \times V_{k-1} \times T$,
        we have that $\{x_1, \dots, x_{k-1}, y\} \in E(H)$.
        In particular, because all $V_i$ are nonempty, this implies that $T$ is disjoint from each of them.
        Furthermore, $H[V_1, \dots, V_{k-1}, T]$ is complete, finishing the proof.

        Let us now prove the claim that $t' \geq t$.
        By the definition of $s$, we have $d' \geq \left( \frac{d}{4} \right)^t$.
        Therefore,
        \[
            t' \geq
            %
            \left\lfloor \left(  \frac{\log n}{\log \left(\frac{16}{(d/4)^t}\right)} \right)^
            {\frac{1}{k-2}} \right\rfloor \geq
            %
            \left\lfloor \left(  \frac{\log n}{\log 16 - t \log (d/4)} \right)^
            {\frac{1}{k-2}} \right\rfloor.
        \]
        Then, we substitute the definition of $t$, where removing the floor function
        maintains the inequality because the right hand side is decreasing in $t$ (recall $d \leq 1$):
        \begin{equation*} \label{eq:t_prime}
            t' \geq
            %
            \left\lfloor \left(  \frac{\log n}
            {\log 16 - \left(  \frac{\log n}{\log (16/d)} \right)^{\frac{1}{k-1}}  \log (d/4)} \right)^
            {\frac{1}{k-2}} \right\rfloor
            =
            %
            \left\lfloor \left(  \frac{(\log n)^{\left(1-\frac{1}{k-1}\right)}}
            {\frac{\log 16}{(\log n)^{\frac{1}{k-1}}} - \frac{\log (d/4)}{\log (16/d)^{\frac{1}{k-1}}} }
            \right)^{\frac{1}{k-2}} \right\rfloor.
        \end{equation*}
        If we bound the denominator by showing that
        \begin{equation} \label{eq:denominator_bound}
            \frac{\log 16}{(\log n)^{\frac{1}{k-1}}} - \frac{\log (d/4)}{\log (16/d)^{\frac{1}{k-1}}}
            \leq \left( \log (16/d) \right)^{\left( 1 - \frac{1}{k-1} \right)},
        \end{equation}
        then the expression simplifies to
        \[
            t'
            %
            \geq \left\lfloor \left(  \frac{(\log n)^{\left(1-\frac{1}{k-1}\right)}}
            {\left( \log (16/d) \right)^{\left( 1 - \frac{1}{k-1} \right)}}
            \right)^{\frac{1}{k-2}} \right\rfloor
            %
            = \left\lfloor \left(  \frac{\log n}
            {\log (16/d)}
            \right)^{\frac{1}{k-2}\left( 1 - \frac{1}{k-1} \right)} \right\rfloor
            = \left\lfloor \left(  \frac{\log n}{\log (16/d)} \right)^{\frac{1}{k-1}} \right\rfloor
            = t,
        \]
        as desired.
        Suppose, by way of contradiction, that Inequality~\eqref{eq:denominator_bound} does not hold.
        We can rewrite
        \[
            (\log (16/d))^{\left( 1 - \frac{1}{k-1} \right)}
            = \frac{\log (16/d)}{\log (16/d)^{\frac{1}{k-1}}}
        \]
        and rearrange the inequality to obtain
        \[
            \frac{\log 16}{(\log n)^{\frac{1}{k-1}}}
            > \frac{\log (16/d) + \log (d/4)}{\log (16/d)^{\frac{1}{k-1}}}
            = \frac{\log 4}{(\log (16/d))^{\frac{1}{k-1}}}.
        \]
        This implies that
        \[
            t
            \leq \left( \frac{\log n}{\log (16/d)} \right)^{\frac{1}{k-1}}
            < \frac{\log 16}{\log 4} = 2,
        \]
        which contradicts the assumption that $t \geq 2$.
    \end{proof}
\end{theorem}

\section{Proof of Polynomial Complexity}\label{sec:complexity}
TODO re-evaluate the complexity analysis. % TODO

\section{Conclusion and Future Work}\label{sec:conclusion-and-future-work}

We have presented a deterministic,
polynomial-time algorithm to find a large complete balanced $k$-partite subgraph in any sufficiently dense $k$-uniform hypergraph.
This provides a constructive counterpart to a classical existence result by Erd\H{o}s in extremal hypergraph theory.

Several avenues for future research remain open.
\begin{itemize}
    \item \textbf{General Blow-ups:} Our algorithm finds a blow-up of a single edge, $\compoverset{k}{t}$.
    Can this framework be adapted to find a $t_n$-blowup of an arbitrary fixed $k$-graph $G$? Existence theorems guarantee such structures, but efficient algorithms are lacking.
    \item \textbf{Unbalanced Partite Graphs:} The algorithm could be modified to search for unbalanced complete partite graphs $\compdots{t_1}{t_k}$, where the part sizes may grow at different rates.
    \item \textbf{Optimality:} The bounds on $t$ are asymptotically tight, but the constants can likely be improved with a more refined analysis.
    For $k=2$, it is known that in dense graphs one can find a $t=\Theta(\log n)$ blow-up of any bipartite graph.
    It is an open question if a constructive proof for this stronger result exists for $k \ge 2$.
\end{itemize}

\subsection*{Acknowledgements}

The ideas in this work stem from the author's master's thesis of the same name,
Universitat Polit\`{e}cnica de Catalunya,
under the supervision of Richard Lang.
The author thanks Dr. Lang for suggesting the problem and for his guidance and support throughout the project.
