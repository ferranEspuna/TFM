\section{Our Algorithm}\label{sec:algorithm}

In this section we present a polynomial-time algorithm to find a balanced $k$-partite
$k$-graph in a given $k$-graph $G$
with $n$ vertices and $m$ edges with part size in the same order of magnitude as stated in
Theorem~\ref{thm:erdos64}.
In fact, we will prove the following:

\begin{theorem}\label{thm:kpartite}
    Let $G$ be a $k$-graph with $n$ vertices and $m = dn^k$ edges.
    There is a polynomial-time algorithm
    to find a balanced partite $k$-graph embedded in $G$ with part size
    \[
        t (n, d, k) \coloneqq \left\lfloor
        \left(  \frac{\log (n/2^{k-1})}{\log (2^{k}/d)} \right)
        ^{\frac{1}{k-1}} \right\rfloor
    \]
    as long as $t \geq 2$.
\end{theorem}

\begin{remark}
    If we let $t$ be the size of each part in the partite $k$-graph we are looking for, we need
    \[
        m \geq ex(n, K(t, \overset{k}{\cdots}, t)) = O\left(n^{k - \frac{1}{t^{k-1}}}\right)
    \]
    Defining $d = \frac{m}{n^k}$, and taking logarithms, this is true when
    \[
        \log d \geq - \frac{\log n}{t^{k-1}} + O(1)
    \]
    which implies
    \[
        t = O\left(\left(\frac{\log n}{\log (1/d)}\right)^{\frac{1}{k-1}}\right)
    \]
    Therefore, Theorem~\ref{thm:kpartite} gives a part size in the same order of magnitude as the bound
    given by Theorem~\ref{thm:erdos64}.
\end{remark}

This algorithmic problem has already been solved for $k = 2$ by Mubayi and Turán~\cite{MUBAYI2010174}.
The algorithm in that case follows very closely the structure of the proof of Theorem~\ref{thm:kst}.
We outline the algorithm for $k = 2$ here for context and clarity.
The variable names have been altered to match the notation used in this thesis.

\begin{algorithm}[H]
    \caption{Finding a balanced bipartite graph in a $2$-graph}
    \label{alg:bipartite}
        \begin{algorithmic}[1]
        \Require A graph G = $(V, E)$ with $|V| = n$, $E = m$
        \State $d \gets m / n^2 $
        \State \textbf{assert} $d \geq 3 n^{-\frac{1}{2}}$

        \State $t \gets \left\lfloor\frac{\log (n/2) }
        {\log (2e/d)}\right\rfloor,\, w \gets \lfloor t/d \rfloor$

        \State $W \gets $ The set of $w$ vertices with highest degree in $G$
        \State $U \gets V \setminus W$
        \ForAll{$T \in \binom{W}{t} $}
            \State $S \gets \{x \in U : {x, y} \in E \text{ for all } y \in T\}$
            \If{$|S| \geq t$}
                \State \Return $(S, T)$
            \EndIf
        \EndFor
        \end{algorithmic}
\end{algorithm}

If the set $S$ is too large we can simply take a subset of it of size $t$.
The algorithm is correct if at some point it returns a pair of sets $(S, T)$.
The argument of why this is the case boils down to showing that there is a
large number of edges between $U$ and $W$ and then applying Theorem~\ref{thm:kst}
with $u = |U| = n - w$ and $s = t$ .
Finally, the algorithm runs in polynomial time because 
the  number of iterations of the loop is
\[
    \binom{w}{t} \leq
    \left(\frac{ew}{t}\right)^t \leq
    \left(\frac{1}{d}\right)^t e^t < e^{t \log (1/d) + \log n} <
    e^{2\log n} = n^2
\]

\begin{remark}
    The requirement for a minimum density is because if $d = o\left(n^{-1/2}\right)$ then
    there may not even be a $K(2, 2)$ embedded in $G$. % TODO: Explain this better
\end{remark}

We now present the algorithm for the general case mentioned in Theorem~\ref{thm:kpartite}.
It follows the same structure as Algorithm~\ref{alg:bipartite},
but it is defined recursively.

\begin{algorithm}
    \caption{Finding a balanced partite $k$-graph in a $k$-graph}
    \label{alg:kpartite}
    \begin{algorithmic}[1]
        \Function{FIND\_PARTITE}{$G, k$}
            \State \textbf{assert} G is a $k$-graph
            \If {$k = 1$}
                \State \Return $\{x : \{x\} \in E(G)\}$
            \EndIf

            \State $n \gets |V(G)|,\, m \gets |E(G)|,\, d \gets m/n^k$
            \State $t \gets t(n, d, k),\, w \gets \lceil 2t/d \rceil,\, s \gets \lfloor d^t n^{k-1} \rfloor$ \label{line:tws}
            \State \textbf{assert} $t \geq 2$ \label{line:min_t}
            \State $W \gets$ the set of $w$ vertices with highest degree in $G$ \label{line:W}
            \State $U \gets \binom{V\setminus W}{k-1}$

            \ForAll{$T \in \binom{W}{t}$} \label{line:for}
                \State $S \gets \{\,y \in U : \{x\} \cup y \in E \text{ for all } x \in T\,\}$
                \If{$|S| \geq s$}
                    \State \Return \Call{FIND\_PARTITE}
                    {$G' \coloneqq (V \setminus W,\, S),\, k-1$} $\circ \, (T)$\label{line:return}
                \EndIf
            \EndFor
        \EndFunction
    \end{algorithmic}
\end{algorithm}

Where the operator $\circ$ denotes the concatenation of tuples, and we
understand a $1$-graph to be a subset of a set.
The aim of the rest of this section is to prove that this algorithm is correct
(as long as the  condition $t \geq 2$ in line~\ref{line:min_t} is met on the first call)
and runs in polynomial time.
That is, to prove it meets the requirements of Theorem~\ref{thm:kpartite}.
The following lemmas are stated assuming that $k \geq 2$ and
$t \geq 2$.

\begin{remark}\label{rm:min_d}
    The $t=1$ case is trivial, as we can simply select one edge of $G$.
    The requirement $t \geq 2$ is met whenever
    \[
        d \geq 2^{k} 2^{\frac{k - 1}{2^{k-1}}} n^{-\frac{1}{2^{k-1}}}
    \]
    Note that this implies that
    \[
        1 \geq d \geq \frac{4\sqrt{2}}{\sqrt{n}}
    \]
    and in particular $n \geq 32$.
\end{remark}

\begin{lemma}\label{lm:sound}
    The selection of $t, w, s$ in line~\ref{line:tws} is sound in the sense that
    $t  \leq w \leq n$, $k - 1 \leq n - w$ and $s \leq \binom{n - w}{k - 1}$.
    \begin{proof}
        $t \leq w$ is clear.
        We will in fact show that $w < \frac{n}{2}$.
        If not,
        \[
            \frac{n}{2} \leq
            w \leq 1 + \frac{2t}{d} <
            1 + \frac{2 \log (n/2) \sqrt{n}}{4\sqrt{2}} =
            1 + \frac{\log(n/2) \sqrt{n}}{2\sqrt{2}} <
            1 + \frac{n}{4}
        \]

        This implies that $n < 4$, in contradiction with Remark~\ref{rm:min_d}.

        We also show that $k < \frac{n}{2}$.
        If not,
        \[
            1 \geq
            d >
            2^{\frac{n}{2}} n^{-\frac{1}{2^{n/2-1}}} \geq
            e^{\frac{n}{2} \log 2 - \frac{\log n}{2^{n/2-1}}}
        \]
        which implies
        \[
            \frac{n}{2} \log 2 < \frac{\log n}{2^{n/2-1}}
        \]
        This is false for all $n \geq 2$, and in particular for $n \geq 32$.
        Therefore, $ k + w < n$, which implies $k - 1 < n - w$,
        as we wanted to show.

        Finally, suppose $s > \binom{n - w}{k - 1}$.
        Then, using the fact that $w < \frac{n}{2}$,
        \[
            \left( \frac{n}{2k} \right)^{k-1} \leq
            \left( \frac{n-w}{k-1} \right)^{k-1} \leq
            \binom{n - w}{k - 1} < s \leq d^t n^{k-1}
        \]
        which implies
        \[
            \left( \frac{1}{2k} \right)^{k-1} < d^t \leq
            \left( \frac{1}{k!} \right)^2
        \]
        Where in the last inequality we use that $t \geq 2$ and there are at most
        $\binom{n}{k} \leq \frac{n^k}{k!}$ edges in $G$.
        Since $k!^2 \geq (2k)^{k-1}$ for all $k$,
        we have reached a contradiction. \qedhere

    \end{proof}
\end{lemma}

\begin{lemma}\label{lm:many_edges}
    With $W  \subset V$ as defined in line~\ref{line:W},
    There are at least $\frac{3}{2}dwn^{k-1}$ edges of $G$ with exactly one vertex in $W$.
    \begin{proof}
        The degree sum over $V$ is $kdn^{k}$.
        Thus, by the pigeonhole principle, the degree sum over $W$ is at least
        $\frac{w}{n}kdn^{k} = wkdn^{k-1}$.
        For $2 \leq j \leq n$,
        consider the contribution to this sum by edges with exactly $j$ vertices in $W$.
        Each such edge contributes $j$ to the sum, and there are at most
        $\binom{w}{j}\binom{n-w}{k-j} \leq
        \frac{w^j n^{k-j}}{j!} \leq
        \frac{w^j n^{k-j}}{j}$ of them.
        Thus, the total contribution of these edges is at most $w^j n^{k-j} \leq w^{2}n^{k-2}$.
        The number of edges with only one vertex in $W$ is then at least

        \[
            wkdn^{k-1} - (k-1)w^{2}n^{k-2} = dwn^{k-1} \left( k - \frac{(k-1)w}{nd}\right)
        \]

        Suppose, by way of contradiction,
        that $ k - \frac{(k-1)w}{nd} < \frac{3}{2}$.
        Using that $\frac{k-1}{k-3/2} \leq 2$
        for $k \geq 2$, we arrive at
        \[
             2 \geq  \frac{nd}{w}
        \]
        which implies
        \[
            d \leq \frac{2w}{n} = \frac{2 \left\lceil\frac{2t}{d} \right\rceil}{n}
            < \frac{6t}{dn}
        \]

        Where the last inequality follows from the fact that $t > 1$ and $d \leq 1$.
        Rearranging:
        \[
            nd^2 < 6t
        \]

        If $k \geq 3$, applying the minimum density requirement from Remark~\ref{rm:min_d} yields:
        \[
            128 \sqrt {n} \leq 4^{k-1 + \frac{k-1}{2^{k-1}}} n \cdot n^{-\frac{2}{2^{k-1}}} \leq nd^2 < 6t < 6 \log n
        \]
        Which is false for all $n$.

        We have to be more careful in the $k = 2$ case.
        We closely follow the steps of~\cite{MUBAYI2010174}:
        \begin{itemize}
            \item If $2 \sqrt{\frac{2}{n}} \leq d < 4 \sqrt{\frac{\log n}{n}}$, we get
            \[
                32 \leq nd^2
                < 6t \leq
                6 \frac{\log n/2}{\log(2/d) } \leq
                6 \frac{\log n}{\log\left(\sqrt{\frac{n}{\log n}}\right)} =
                12 \frac{\log n}{{\log \left( \frac{n}{\log n} \right)}} <
                12 \frac{\log n}{{\log \left( \frac{n}{\log n} \right)}} <
                12 \frac{\log n}{{\log \left( n^{2/3} \right)}} =
                18
            \]
            where last inequality follows from $\log n < n^{1/3}$.
            This is a contradiction.

            \item If $2 \sqrt{\frac{\log n}{n}} \leq d \leq \frac{1}{2}$, then
            \[
                4 \, \frac{\log n}{n} \leq nd^2 < 6t \leq 6 \log n
            \]
            so that
            \[
                n < 4 n \leq 6
            \]
            in contradiction with Remark~\ref{rm:min_d}. \qedhere
        \end{itemize}
    \end{proof}
\end{lemma}

\begin{lemma}\label{lm:return}
    Line~\ref{line:return} of Algorithm~\ref{alg:kpartite} is reached at some point in the for
    loop in line~\ref{line:for}.
    \begin{proof}
        We will apply Theorem~\ref{thm:kst} to the $2$-partite $2$-graph
        \[
            \mathcal{P} = (U, W, \{(x, y) \in U \times W) | \{x\} \cup y \in E \})
        \]
        By Lemma~\ref{lm:many_edges}, $\mathcal{P}$ has at least
        $\frac{3}{2}dwn^{k-1}$ edges.

        Suppose that what we want to show is false.
        This means that for no sets $S \in \binom{U}{s}, T \in \binom{W}{t}$
        such that $(x, y) \in E (\mathcal{P})$ for all $x \in S, y \in T $.
        In other words, there is no embedding of $K(s, t)$ on $\mathcal{P}$.
        This means that

        % TODO: check the math works, just copied

        \begin{align*}
            \frac{3}{2}dwn^{k-1} \leq &
            \, z \left(\binom{n - w}{k-1}, w, s, t  \right) \leq
            (s-1)^{1/t}(w-t+1)\binom{n-w}{k-1}^{1-1/t} + (t-1)\binom{n-w}{k-1} \leq \\
            \leq & \, s^{1/t} w \binom{n}{k-1}^{1-1/t} + t \binom{n}{k-1} \leq
             s^{1/t} wn^{(k-1)(1-1/t)} + tn^{k-1} \leq \\
            \leq & \, s^{1/t} wn^{(k-1)(1-1/t)} + \frac{1}{2} dwn^{k-1}
        \end{align*}

        Where the last inequality follows from our definition of $w$.
        Rearranging, we get

        \[
            dwn^{k-1} \leq s^{1/t} wn^{(k-1)(1-1/t)}
        \]
        which implies
        \[
            d \leq \left(\frac{s}{n^{k-1}}\right)^{1/t}
        \]
        which is false by the definition of $s$.
    \end{proof}
\end{lemma}

Now, for the base case, we need:

\begin{lemma}\label{lm:base_case}
    For $k=2$, Algorithm~\ref{alg:kpartite} finds $s \geq t$.
    \begin{proof}
        Suppose $t < s$.
        Substituting $k=2$, we get $t > \left\lfloor d^t n \right\rfloor$ which implies
        \[
            t >
            d^t n \geq
            d^{\frac{\log n}{\log (2/d)}} n =
            2^{\frac{\log n}{\log (2/d)}}(d/2)^{\frac{\log n}{\log (2/d)}} n \geq
            \frac{2^t}{n} n =
            2^t
        \]
        Which is false for all $t \geq 0$.
    \end{proof} % TODO review
\end{lemma}

For the recursive step, we need:

\begin{lemma}\label{lm:d_t}
    For $k \geq 3$, in the recursive call in line~\ref{line:return} of Algorithm~\ref{alg:kpartite},
    we have
    \[
        d' \coloneqq \frac{s}{(n-w)^{k-1}} \geq \frac{d^t}{2}
    \]

    \begin{proof}
        By the definition of $s$,
        \[
            d' =
            \frac{s}{(n-w)^{k-1}} \geq
            \frac{s}{n^{k-1}} \geq
            \frac{d^t n^{k-1} - 1 }{n^{k-1}} =
            d^t - n^{1-k}
        \]
        so in fact we only need to show that $n^{1-k} \leq d^t/2$.
        However,
        \[
            d^t \geq
            d^{\left( \frac{\log \left(\frac{n}{2^{k-1}}\right)}{\log \left(\frac{2^{k-1}}{d}\right)} \right)^{1/(k-1)}}
        \]
        where the right hand side is increasing in $d$.
        Therefore, we may substitute the minimum density requirement from Remark~\ref{rm:min_d}.
        That is, our statement is true if
        \[
            n^{1-k} \leq \left(2^{k} 2^{\frac{k - 1}{2^{k-1}}} n^{-\frac{1}{2^{k-1}}}\right)^2
        \]
        which is clearly true for all $n > 0$.
    \end{proof}

\end{lemma}

\begin{lemma}\label{lm:t_prime}
    For $k \geq 3$, in the recursive call in line~\ref{line:return} of Algorithm~\ref{alg:kpartite},
    the resulting part size $t'$ satisfies
    \[
        t' \coloneqq t(n - w, d', k - 1) \geq t
    \]
    
    \begin{proof}

        Substituting the new parameters into the definition of $t$, we get
        \[
            t' = \left\lfloor \left(  \frac{\log ((n-w)/2^{k-2})}{\log (2 \cdot 2^{k-1}/d')} \right)^
            {\frac{1}{k-2}} \right\rfloor
        \]

        We start by using Lemma~\ref{lm:d_t} and the fact that $w \leq n/2$:

        \[
            t' \geq
            %
            \left\lfloor \left(  \frac{\log ((n-w)/2^{k-2})}{\log (2 \cdot 2^{k-1}/d^t)} \right)^
            {\frac{1}{k-2}} \right\rfloor \geq
            %
            \left\lfloor \left(  \frac{\log (n/2^{k-1})}{\log (2^{k}/d^t)} \right)^{\frac{1}{k-2}} \right\rfloor =
            %
            \left\lfloor \left(  \frac{\log (n / 2^{k-1})}{k \log 2 - t \log d} \right)^
            {\frac{1}{k-2}} \right\rfloor
        \]
        Then, we substitute the definition of $t$, where removing the floor function
        maintains the inequality because the right hand side is decreasing in $t$ (recall $d \leq 1$):

        \[
            t' \geq
            %
            \left\lfloor \left(  \frac{\log (n / 2^{k-1})}
            {k \log 2 - \left(  \frac{\log (n / 2^{k-1})}{\log (2^{k}/d)} \right)^{\frac{1}{k-1}}  \log d} \right)^
            {\frac{1}{k-2}} \right\rfloor
            =
            %
            \left\lfloor \left(  \frac{\log (n / 2^{k-1})^{1-\frac{1}{k-1}}}
            {\frac{k \log 2}{\log(n / 2^{k-1})^{\frac{1}{k-1}}} - \frac{\log d}{\log (2^{k}/d)^{\frac{1}{k-1}}} }
            \right)^{\frac{1}{k-2}} \right\rfloor \\
        \]

        Now we argue that $n/2^{k-1} \geq 2^{k}/d$.
        Otherwise,
        \[
            2^{k} 2^{\frac{k - 1}{2^{k-1}}} n^{-\frac{1}{2^{k-1}}} \leq d < \frac{2^{2k-1}}{n}
        \]
        which implies
        \[
            n^{1 - \frac{1}{2^{k-1}}} \leq 2^{k-1}
        \]
        so that
        \[
            n \leq 2^{\frac{k-1}{1-\frac{1}{2^{k-1}}}}
        \]
        Substituting this expression into the minimum density requirement, we get
        \[
            d >
            2^{k} \left( 2^{\frac{k-1}{1-\frac{1}{2^{k-1}}}} \right)^{-\frac{1}{2^{k-1}}}  \geq
            2^{k-(k-1)} = 2
        \]
        which is a contradiction as $d \leq 1$.
        This allows us to find a common denominator on the
        right hand side of the previous inequality:
        \[
            t' \geq
            %
            \left\lfloor \left(  \frac{\log (n / 2^{k-1})^{1-\frac{1}{k-1}}}
            {\frac{\log (2^{k} / d)}{\log(2^{k} / d)^{\frac{1}{k-1}}} } \right)^
            {\frac{1}{k-2}} \right\rfloor =
            %
            \left\lfloor \left(  \frac{\log (n / 2^{k-1})}
            {\log (2^{k} / d)} \right)^
            {\frac{1}{k-2}\left( 1 - \frac{1}{k-1} \right)} \right\rfloor =
            %
            \left\lfloor \left(  \frac{\log (n / 2^{k-1})}
            {\log (2^{k} / d)} \right)^
            {\frac{1}{k-1}} \right\rfloor =
            %
            t \qedhere
        \]
    \end{proof}
    

\end{lemma}

All in all, we can now state the following theorem:

\begin{theorem}
    Algorithm~\ref{alg:kpartite} finds a balanced partite $k$-graph in a $k$-graph $G$ with
    $n$ vertices and $m = d n^k$ with part size $t(n, d, k)$ in polynomial time.
    \begin{proof}
        To prove the correctness of the algorithm, we will proceed by induction on $k$:
        If $k=2$, it follows from Lemmas~\ref{lm:return} and~\ref{lm:base_case}.
        Likewise, if $k \geq 3$, Lemma~\ref{lm:return} tells us that the algorithm will
        reach line~\ref{line:return} at some point.
        Furthermore, Lemma~\ref{lm:t_prime} tells us that the recursive call in line~\ref{line:return}
        will have a part size $t'$ that is at least $t$.
        In particular, this means that $t' \geq 2$.
        Using the induction hypothesis for $k-1$, this recursive call will be successful and
        return a tuple of pairwise disjoint sets
        $(X_1, X_2, \ldots, X_{k-1}) \in \mathcal{P}(V \setminus W)^{k-1}$ such that:
        \begin{itemize}
            \item $|X_i| \geq t(n-w, d', k-1) \geq t$

            \item $X_1 \times \dots \times X_{k-1} \subseteq E(G') = S =
            \left\{x \in \binom{V \setminus W}{k-1} : \{x\} \cup y \in E \text{ for all } y \in T\right\}$
        \end{itemize}

        That is, (making the sizes of the $X_i$ smaller if necessary) the returned tuple $(X_1, \dots, X_{k-1}, T)$
        satisfies $X_1 \times \dots \times X_{k-1} \times T \subseteq E = E(G)$, making the algorithm correct.

        For the time complexity, note that all operations in the algorithm are in polynomial time, % TODO: check this
        except for perhaps the for loop in line~\ref{line:for} and the recursive call in line~\ref{line:return}.
        Because there is only one recursive call, we can prove that it runs in polynomial time
        by induction on $k$.
        The only thing left to show is that the for loop runs in polynomial time.
        This is argued in~\cite{MUBAYI2010174}, but we reproduce the argument here for completeness:
        As seen in~\cite{reingold1977combinatorial}, the $t$-sets of $W$ can be enumerated in
        $O\left( \binom{w}{t} \right)$ steps.
        However, we can bound
        \[
            \binom{w}{t} \leq
            \binom{2t/d + 1}{t} <
            \left( \frac{3et/d}{t} \right)^{t} =
            \left( \frac{3e}{d} \right)^{t} <
            e^{3 t + t \log (1/d)} <
            e^{4 \log n} = n^4 \qedhere
        \]

    \end{proof}
\end{theorem}





