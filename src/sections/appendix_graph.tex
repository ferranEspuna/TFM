\vfill\newpage \section{Properties of Hypergraph Embeddings}\label{apx:embeddings-properties}

\begin{proposition}\label{prop:embedding_properties}
    Graph inclusion $(\subset)$ and induced graph inclusion $\left(\subset_{\text{ind}}\right)$
    are preorder relations on $k$-graphs.
    \begin{proof}
        We need to show that the relations are reflexive and transitive.
        Reflexivity is clear, as the identity map is an induced embedding of a $k$-graph in itself.
        Let $G, H,$ and $K$ be $k$-graphs with vertex sets $X, Y,$ and $Z$ respectively.
        If $G \subset H$ via $f: X \to Y$ and $H \subset K$ via $g: Y \to Z$,
        then $g \circ f: X \to Z$ is injective and satisfies that for each edge $e \in E(G)$,
        \[
            g \circ f(e) =
            \{g(f(x))\mid x \in e\} =
            \{g(y) \mid y \in f(e)\} \in E(K),
        \]
        because $f(e) \in E(H)$.
        Therefore, $G \subset K$ via $g \circ f$.
        If the embeddings are induced,
        and $e$ is an edge in
        $E(K[g \circ f(X)])$,
        then $e$ is also an edge in $E(K[g (Y)]) = g(E(H))$.
        Therefore, $e' = g^{-1}(e)$ is an edge in $H$.
        Furthermore, because $e = g(e') \subset g \circ f(X)$,
        we have that $e' \in E(H[f(X)]) = f(E(G))$,
        so $e \in g(f(E(G)))$.
    \end{proof}
\end{proposition}

\begin{remark} \label{rem:inverse_embedding}
    In~\Cref{def:embedding}, given that a map $f: V \to W$ is an embedding
    (and therefore injective),
    a different way to state that it is an induced embedding is to say that
    $f^{-1}: H[f(V)] \to G$ is also an embedding.
\end{remark}

\begin{proposition}\label{prop:isomorphism_equivalence}
    The relation of isomorphism $(\cong)$ is an equivalence relation on $k$-graphs.
    \begin{proof}
        The relation is reflexive via the identity map.
        If $f: G \to H$ is an isomorphism, then $f^{-1}: H \to G$ is also an isomorphism,
        so the relation is symmetric.
        Finally, if $f: G \to H$ and $g: H \to K$ are isomorphisms,
        then $g \circ f: G \to K$ is also an isomorphism, because it is bijective,
        and by~\Cref{prop:embedding_properties},
        it is an embedding
        and $(g \circ f)^{-1} = f^{-1} \circ g^{-1}$ is also an embedding.
        By~\Cref{rem:inverse_embedding}, we are done.
    \end{proof}
\end{proposition}

\begin{proposition} \label{prop:isomorphism_preserves_embedding}
    Let $G, G', H, H'$ be $k$-graphs such that $G \cong G'$ and $H \cong H'$.
    Then,
    \begin{enumerate}
        \item $G \subseteq H$ if and only if $G' \subseteq H'$. \label{item:embedding}
        \item $G \subseteq_{\text{ind}} H$ if and only if $G' \subseteq_{\text{ind}} H'$. \label{item:induced_embedding}
    \end{enumerate}
    \begin{proof}
        Because the isomorphism relation is symmetric, we only need to show one direction of each implication.
        let $f: V(G) \to V(H)$ be an embedding of $G$ in $H$, and let
        $g: V(G) \to V(G')$ and $h: V(H) \to V(H')$ be isomorphisms between the respective graphs.
        We claim that the composition
        \[
            f' = h \circ f \circ g^{-1}: V(G') \to V(H')
        \]
        is an embedding of $G'$ in $H'$.
        Injectivity is given by the injectivity of $h, f$, and $g^{-1}$.
        By~\Cref{prop:isomorphism_equivalence}, we have that $g^{-1}$ is an isomorphism of $G'$ in $G$,
        and in particular an embedding.
        Therefore, by~\Cref{prop:embedding_properties},
        $f'$ is an embedding of $G'$ in $H'$, proving part~\eqref{item:embedding}.
        Suppose now that the embedding $f$ is induced.
        consider the maps
        \[
            (f')^{-1}: f'(V(G')) \to V(G')
        \]
        and
        \[
            \varphi = g \circ f^{-1} \circ h^{-1}: h \circ f(V(G)) \to V(G'),
        \]
        where we restrict the domain of $f^{-1}$ to $f(V(G))$.
        Because $g$ is a bijection, $V(G) = g^{-1}(V(G'))$ so the
        domain of $\varphi$ is $f'(V(G'))$.
        In fact, one can check that the two functions are identical.
        Because $f^{-1}$ is an embedding of $H[f(V(G))]$ in $G$,
        we can argue as in the first case that
        $(f')^{-1}$ is an embedding of $H'[f'(V(G'))]$ in $G'$,
        and therefore $f'$ is an induced embedding of $G'$ in $H'$.
        This concludes the proof of part~\eqref{item:induced_embedding}.
    \end{proof}
\end{proposition}

\Cref{prop:isomorphism_equivalence,prop:embedding_properties,prop:isomorphism_preserves_embedding},
establish that the properties of containing a $k$-graph $G$ as a (induced)
subgraph within a $k$-graph $H$ depend only on the isomorphism classes of $G$ and $H$.
Therefore, discussions of (induced) subgraph containment can be conducted up to isomorphism.