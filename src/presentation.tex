\documentclass{beamer}
\usepackage{amsmath,amssymb,amsfonts}
\usepackage{tikz}
\usetikzlibrary{calc}
\usepackage{booktabs}
\usepackage{blkarray} % For tables if needed, though likely not for a 20-min talk

% Thesis specific commands (simplified or as needed)
\newcommand{\N}{\ensuremath{\mathbb{N}}}
\newcommand{\ex}[2]{\ensuremath{\text{ex} \left( #1, #2 \right)}}
\newcommand{\compoverset}[2]{\ensuremath{K\left(#2, \overset{#1}{\dots}, #2\right)}} % k-partite, k parts of size #2
\newcommand{\compdots}[2]{\ensuremath{K\left(#1, \dots, #2\right)}} % k-partite, part sizes #1 to #2
\newcommand{\bigO}[1]{\ensuremath{\mathcal{O}\left(#1\right)}}
\newcommand{\OmegaBig}[1]{\ensuremath{\Omega\left(#1\right)}} % For asymptotic lower bound
\newcommand{\completesuperindex}[2]{\ensuremath{K^{(#1)}_{#2}}} % Complete k-graph on #2 vertices
\newcommand{\link}[3]{\ensuremath{L_{#1}\left(#2; #3\right)}} % Link graph

\makeatletter
\def\th@mystyle{%
    \normalfont % body font
    \setbeamercolor{block title example}{bg=orange,fg=white}
    \setbeamercolor{block body example}{bg=orange!20,fg=black}
    \def\inserttheoremblockenv{exampleblock}
  }
\makeatother
\theoremstyle{mystyle}
\newtheorem*{remark}{Remark}

\title[Finding Partite Hypergraphs Efficiently]{Finding Partite Hypergraphs Efficiently}
\author{Ferran Espuña Bertomeu}
\institute{Supervisor: Richard Lang}
\date{June 2025}

\usetheme{Copenhagen} % A common Beamer theme; choose any you like
\usecolortheme{beaver}
\setbeamertemplate{navigation symbols}{} % Remove navigation symbols

\begin{document}

    \frame{\titlepage}

    \section{Hypergraphs}\label{sec:notation}

    \begin{frame}
        \frametitle{$k$-Graphs}
        \begin{definition}
            A \emph{$k$-graph} is a pair $G = (V, E)$
            where $V$ is a finite set of \emph{vertices} and
            $E \subseteq \binom{V}{k}$ is a set of \emph{edges}.
        \end{definition}
        \begin{figure}[htbp]
            \centering
            \scalebox{0.7}{
                % TikZ code for K_4^(3) using predefined edge colors
\begin{tikzpicture}[scale=1]
\coordinate (V1) at (0, 5);
\coordinate (V2) at (5, 5);
\coordinate (V3) at (5, 0);
\coordinate (V4) at (0, 0);
\coordinate (R0) at (3.500, 3.500);
\draw[line width=1.5pt, color=magenta!70!white] (R0) -- (V1);
\draw[line width=1.5pt, color=magenta!70!white] (R0) -- (V2);
\draw[line width=1.5pt, color=magenta!70!white] (R0) -- (V3);
\fill[color=magenta!70!white] (R0) circle (2.5pt);
\coordinate (R1) at (1.500, 3.500);
\draw[line width=1.5pt, color=teal!70!white] (R1) -- (V1);
\draw[line width=1.5pt, color=teal!70!white] (R1) -- (V2);
\draw[line width=1.5pt, color=teal!70!white] (R1) -- (V4);
\fill[color=teal!70!white] (R1) circle (2.5pt);
\coordinate (R2) at (1.500, 1.500);
\draw[line width=1.5pt, color=blue!70!white] (R2) -- (V1);
\draw[line width=1.5pt, color=blue!70!white] (R2) -- (V3);
\draw[line width=1.5pt, color=blue!70!white] (R2) -- (V4);
\fill[color=blue!70!white] (R2) circle (2.5pt);
\coordinate (R3) at (3.500, 1.500);
\draw[line width=1.5pt, color=orange!70!white] (R3) -- (V2);
\draw[line width=1.5pt, color=orange!70!white] (R3) -- (V3);
\draw[line width=1.5pt, color=orange!70!white] (R3) -- (V4);
\fill[color=orange!70!white] (R3) circle (2.5pt);
\fill[black] (V1) circle (4.0pt);
\fill[black] (V2) circle (4.0pt);
\fill[black] (V3) circle (4.0pt);
\fill[black] (V4) circle (4.0pt);
\end{tikzpicture}
            }
            \caption{A complete $3$-graph on $4$ vertices: $K_4^{(3)}$.}
            \label{fig:complete_kgraph}
        \end{figure}
    \end{frame}

    \begin{frame}
        \frametitle{Partite $k$-Graphs}
        \begin{definition}
            A $k$-graph $G = (V, E)$ is \emph{$r$-partite}
            if there exists a partition $V = V_1 \cup \dots \cup V_r$
            such that every edge of $G$ intersects every part $V_i$ in at most one vertex.
            We write $G = (V_1, \dots, V_r; E)$.
        \end{definition}

        \begin{figure}[htbp]
            \centering
            \scalebox{0.6}{
                % TikZ code for K^(2)(2, 2, 2) using predefined edge colors
\pgfdeclarelayer{background}
\pgfdeclarelayer{main}
\pgfsetlayers{background,main}
\begin{tikzpicture}[scale=0.8]
\begin{pgfonlayer}{background}
  \draw[fill=gray!20, rounded corners] (0.30, 0.30) rectangle (1.70, 3.70);
\end{pgfonlayer}
\node at (1.00, 2.00) [align=center] {$V_1$};
\begin{pgfonlayer}{background}
  \draw[fill=gray!20, rounded corners] (8.30, 0.30) rectangle (9.70, 3.70);
\end{pgfonlayer}
\node at (9.00, 2.00) [align=center] {$V_2$};
\begin{pgfonlayer}{background}
  \draw[fill=gray!20, rounded corners] (4.30, 5.30) rectangle (5.70, 8.70);
\end{pgfonlayer}
\node at (5.00, 7.00) [align=center] {$V_3$};
\coordinate (A1) at (1, 1);
\coordinate (A2) at (1, 3);
\coordinate (B1) at (9, 1);
\coordinate (B2) at (9, 3);
\coordinate (C1) at (5, 6);
\coordinate (C2) at (5, 8);
\draw[line width=1.5pt, color=magenta!60!white] (A1) -- (B1);
\draw[line width=1.5pt, color=magenta!60!white] (A1) -- (B2);
\draw[line width=1.5pt, color=magenta!60!white] (A2) -- (B1);
\draw[line width=1.5pt, color=magenta!60!white] (A2) -- (B2);
\draw[line width=1.5pt, color=teal!60!white] (A1) -- (C1);
\draw[line width=1.5pt, color=teal!60!white] (A1) -- (C2);
\draw[line width=1.5pt, color=teal!60!white] (A2) -- (C1);
\draw[line width=1.5pt, color=teal!60!white] (A2) -- (C2);
\draw[line width=1.5pt, color=orange!60!white] (B1) -- (C1);
\draw[line width=1.5pt, color=orange!60!white] (B1) -- (C2);
\draw[line width=1.5pt, color=orange!60!white] (B2) -- (C1);
\draw[line width=1.5pt, color=orange!60!white] (B2) -- (C2);
\begin{pgfonlayer}{main}
  \fill[black] (A1) circle (4.0pt);
  \fill[black] (A2) circle (4.0pt);
  \fill[black] (B1) circle (4.0pt);
  \fill[black] (B2) circle (4.0pt);
  \fill[black] (C1) circle (4.0pt);
  \fill[black] (C2) circle (4.0pt);
\end{pgfonlayer}
\end{tikzpicture}
            }
            \caption{A complete $3$-partite $2$-graph: $K^{(3)}(2, 2, 2)$.}
            \label{fig:complete_3partite_2graph}
        \end{figure}
    \end{frame}

    \begin{frame}
        \frametitle{Partite $k$-Graphs}
        \begin{remark}
            We may identify $E$ as a subset of $\mathcal{C} = \bigcup_{\{i_1, \dots, i_k \} \in \binom{[r]}{k}} V_{i_1} \times \dots \times V_{i_k}$.
            If $E = \mathcal{C}$, we say that $G$ is a \emph{complete} $r$-partite $k$-graph.
        \end{remark}

        \begin{figure}[htbp]
            \centering
            \scalebox{0.55}{
                % TikZ code for K^(3)(2, 2, 2) using predefined edge colors
\begin{tikzpicture}[scale=1]
\draw[fill=gray!20, rounded corners] (-0.50, -0.50) rectangle (2.50, 2.50);
\node at (1.00, 1.00) [align=center] {$V_1$};
\draw[fill=gray!20, rounded corners] (7.50, -0.50) rectangle (10.50, 2.50);
\node at (9.00, 1.00) [align=center] {$V_2$};
\draw[fill=gray!20, rounded corners] (1.50, 5.50) rectangle (8.50, 6.50);
\node at (5.00, 6.00) [align=center] {$V_3$};
\coordinate (A1) at (2, 0);
\coordinate (A2) at (0, 2);
\coordinate (B1) at (8, 0);
\coordinate (B2) at (10, 2);
\coordinate (C1) at (2, 6);
\coordinate (C2) at (8, 6);
\coordinate (R0) at (3.333, 2.148);
\draw[line width=1.5pt, color=blue!70!white] (R0) -- (A1);
\draw[line width=1.5pt, color=blue!70!white] (R0) -- (B1);
\draw[line width=1.5pt, color=blue!70!white] (R0) -- (C1);
\fill[color=blue!70!white] (R0) circle (2.5pt);
\coordinate (R1) at (6.667, 2.148);
\draw[line width=1.5pt, color=blue!70!white] (R1) -- (A1);
\draw[line width=1.5pt, color=blue!70!white] (R1) -- (B1);
\draw[line width=1.5pt, color=blue!70!white] (R1) -- (C2);
\fill[color=blue!70!white] (R1) circle (2.5pt);
\coordinate (R2) at (4.222, 3.037);
\draw[line width=1.5pt, color=blue!70!white] (R2) -- (A1);
\draw[line width=1.5pt, color=blue!70!white] (R2) -- (B2);
\draw[line width=1.5pt, color=blue!70!white] (R2) -- (C1);
\fill[color=blue!70!white] (R2) circle (2.5pt);
\coordinate (R3) at (7.556, 3.037);
\draw[line width=1.5pt, color=blue!70!white] (R3) -- (A1);
\draw[line width=1.5pt, color=blue!70!white] (R3) -- (B2);
\draw[line width=1.5pt, color=blue!70!white] (R3) -- (C2);
\fill[color=blue!70!white] (R3) circle (2.5pt);
\coordinate (R4) at (2.444, 3.037);
\draw[line width=1.5pt, color=blue!70!white] (R4) -- (A2);
\draw[line width=1.5pt, color=blue!70!white] (R4) -- (B1);
\draw[line width=1.5pt, color=blue!70!white] (R4) -- (C1);
\fill[color=blue!70!white] (R4) circle (2.5pt);
\coordinate (R5) at (5.778, 3.037);
\draw[line width=1.5pt, color=blue!70!white] (R5) -- (A2);
\draw[line width=1.5pt, color=blue!70!white] (R5) -- (B1);
\draw[line width=1.5pt, color=blue!70!white] (R5) -- (C2);
\fill[color=blue!70!white] (R5) circle (2.5pt);
\coordinate (R6) at (3.333, 3.926);
\draw[line width=1.5pt, color=blue!70!white] (R6) -- (A2);
\draw[line width=1.5pt, color=blue!70!white] (R6) -- (B2);
\draw[line width=1.5pt, color=blue!70!white] (R6) -- (C1);
\fill[color=blue!70!white] (R6) circle (2.5pt);
\coordinate (R7) at (6.667, 3.926);
\draw[line width=1.5pt, color=blue!70!white] (R7) -- (A2);
\draw[line width=1.5pt, color=blue!70!white] (R7) -- (B2);
\draw[line width=1.5pt, color=blue!70!white] (R7) -- (C2);
\fill[color=blue!70!white] (R7) circle (2.5pt);
\fill[black] (A1) circle (4.0pt);
\fill[black] (A2) circle (4.0pt);
\fill[black] (B1) circle (4.0pt);
\fill[black] (B2) circle (4.0pt);
\fill[black] (C1) circle (4.0pt);
\fill[black] (C2) circle (4.0pt);
\end{tikzpicture}
            }
            \caption{A complete $3$-partite $3$-graph: $K^{(2)}(2, 2, 2)$.}
            \label{fig:complete_3partite_3graph}
        \end{figure}
    \end{frame}

    \section{Turán-Type Problems}\label{sec:turan-type}

    \begin{frame}
        \frametitle{Turán-Type Problems}
        \begin{definition}
            Let $G = (V, E)$ be a $k$-graph and $n \geq |V|$ an integer.
            The \emph{Turán number} $\ex{G}{n}$ is the maximum number of edges in a $k$-graph on $n$ vertices
            that does not contain a copy of $G$ as a subgraph.
        \end{definition}
        Determining $\ex{G}{n}$ or estimating it as $n \to \infty$ is known as the \emph{Turán problem} for $G$.
        \begin{theorem}
            For all $k$-graphs $G$ there exists a constant $\alpha (G) \in [0, 1)$ such that
            \[
                \ex{G}{n} = (\alpha (G) + o(1)) \cdot \binom{n}{k} \quad \text{as } n \to \infty.
            \]
            Furthermore, $\alpha (G) = 0$ if and only if $G$ is $k$-partite.
        \end{theorem}
    \end{frame}

    \begin{frame}
        \frametitle{The Kővari--Sós--Turán Theorem}
        The bound $\ex{G}{n} = o(n^k)$ can be improved by a lot.
        \begin{definition}
         Let $1 < t_1 \leq v_1, \dots, 1 < t_k \leq v_k$ be integers.
            Then the \emph{generalized Zarankiewicz number} $z(v_1, \dots, v_k; t_1, \dots, t_k)$
            is the largest integer $z$ for which there exists a $k$-partite $k$-graph
            $H = (V_1, \dots V_k, F)$ with part sizes $ |V_i| = v_i$ and $|F| = z$ edges
            such that for all choices of $W_i \subset V_i$ of sizes $|W_i| = t_i$,
            $W_1 \times \dots \times W_k \not \subset F$.
        \end{definition}
        \begin{theorem}[Kővari--Sós--Turán]
            Let $0 < s \leq u$ and $0 < t \leq w$ be integers.
            Then
            \[z(u, w; s, t) \leq (s - 1)^{1 / t}(w - t + 1)u^{1 - 1 / t} + (t - 1)u\]
        \end{theorem}
        Standard arguments then show that
        $\ex{n}{K(s, t)} = \bigO{n^{2 - 1 / t}}$.
    \end{frame}





\end{document}