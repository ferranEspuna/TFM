\documentclass[11pt,a4paper]{article}
\usepackage{amsmath,amssymb,amsthm}
\usepackage{algorithm,algpseudocode}
\usepackage{geometry}
\geometry{a4paper, margin=1in}

% --- Custom Commands from Thesis ---
\newcommand{\ex}[2]{\ensuremath{\mathrm{ex}(#1, #2)}}
\newcommand{\compoverset}[2]{\ensuremath{K(#2, \overset{#1}{\dots}, #2)}}
\newcommand{\compdots}[2]{\ensuremath{K(#1, \dots, #2)}}
\newcommand{\bigO}[1]{\ensuremath{\mathcal{O}(#1)}}
\newtheorem{theorem}{Theorem}
\newtheorem{lemma}[theorem]{Lemma}
\newtheorem{corollary}[theorem]{Corollary}
\theoremstyle{definition}
\newtheorem{definition}{Definition}

% --- Article Information ---
\title{Finding Partite Hypergraphs Efficiently}
\author{Ferran Espuña Bertomeu}
\date{June 2025}

\begin{document}

\maketitle

\begin{abstract}
TODO % TODO
\end{abstract}

\section{Introduction}\label{sec:introduction}

Hypergraph Turán problems study how many edges a $k$-uniform hypergraph $H = (V, E)$ with $n$ vertices can have without containing a specific subgraph $G$.
The maximal such number is known as the \emph{Turán number} $\ex{n}{G}$.
It is known % todo reference
that $\ex{n}{G} = o\left( \binom{n}{k}\right)$ if and only if $G$ is $k$-partite, i.e.,
if its vertex set can be partitioned into $k$ disjoint sets such that each edge contains exactly one vertex from each part.
Kővári, Sós, and Turán~\cite{Kovari1954} (for $k=2$) and
Erdős~\cite{Erods1964} (for any $k \geq 2$) established that
\begin{equation} \label{eq:erdos64-intro}
    \ex{n}{\compoverset{k}{t}} = \bigO{n^{k - \frac{1}{t^{(k-1)}}}},
\end{equation}
where $\compoverset{k}{t}$ is the complete balanced $k$-partite $k$-graph with $k$ parts of size $t$.
Furthermore, if $H$ is a $k$-graph with at least $d \binom{n}{k}$ edges for some constant $d > 0$, then it contains a $\compoverset{k}{t}$ with
$t = c_d \log(n)^{1/(k-1)}$.

This result is non-constructive, meaning it guarantees the existence of such a subgraph but does not provide an efficient way to find it.
Note that a simple brute-force search for a $\compoverset{k}{t}$ would involve checking all $\binom{n}{kt}$ vertex subsets, which is superpolynomial in $n$ for $t = \Theta((\log n)^{1/(k-1)})$.
Mubayi and Turán~\cite{MUBAYI2010174} developed a polynomial-time algorithm for the case $k=2$, which reaches the stated order of magnitude for the subgraph part size.
This paper extends their approach to the general case of $k$-uniform hypergraphs, reaching analogous results for $k \ge 3$.
More concretely, we prove the following.

\begin{theorem} \label{thm:main_theorem}
There is a deterministic algorithm that, given a $k$-graph $H$ with $n$ vertices and $m=dn^k$ edges, finds a complete balanced $k$-partite subgraph $\compoverset{k}{t}$ in polynomial time, where
\[
    t = \left\lfloor \left( \frac{\log(n/2^{k-1})}{\log(3/d)} \right)^{\frac{1}{k-1}} \right\rfloor.
\]
\end{theorem}
This value of $t$ matches the order of magnitude from existence proofs.
In fact, a probabilistic argument shows that it is the best possible up to a constant factor.

\section{Finding a Balanced $k$-Partite Subgraph}\label{sec:finding-a-balanced-$k$-partite-subgraph}

We present a recursive algorithm, \texttt{Find\_Partite}, that finds a $\compoverset{k}{t}$ in a given $k$-graph $H$. The core idea is to reduce the uniformity of the problem from $k$ to $k-1$ in each recursive step.

The algorithm takes a $k$-graph $H$ as input. It first defines the target part size $t$, a small set size $w$, and a threshold edge count $s$ for the recursive call, based on the input graph's parameters ($n$, $d=m/n^k$, $k$):
\begin{align*}
    t(n, d, k) &= \left\lfloor \left(  \frac{\log \left(n/2^{k-1}\right)}{\log (3/d)} \right)^{\frac{1}{k-1}} \right\rfloor \\
    w(n, d, k) &= \left\lceil \frac{2t(n, d, k)}{d} \right\rceil \\
    s(n, d, k) &= \left\lceil d^{t(n, d, k)} n^{k-1} \right\rceil
\end{align*}

The main steps are:
\begin{enumerate}
    \item \textbf{Base Case ($k=1$):} A 1-graph is just a collection of vertices. Return the set of all vertices that are "edges".
    \item \textbf{Select High-Degree Vertices:} Choose a set $W \subset V$ of $w$ vertices with the highest degrees in $H$. Let $U = V \setminus W$.
    \item \textbf{Find a Dense Link Graph:} Iterate through all $t$-subsets $T \subset W$. For each $T$, consider the set $S$ of all $(k-1)$-subsets of $U$ that form a hyperedge with \emph{every} vertex in $T$.
    \item \textbf{Recurse:} The pigeonhole principle, formalized by the Kővári–Sós–Turán theorem, guarantees that for at least one choice of $T$, the resulting set $S$ will be large ($|S| \ge s$). We form a new $(k-1)$-graph $H'=(U, S)$ and make a recursive call: \texttt{Find\_Partite($H'$, $k-1$)}.
    \item \textbf{Construct Solution:} The recursive call returns $k-1$ parts $V_1, \dots, V_{k-1}$ of size at least $t$. By construction, every choice of vertices from these parts forms an edge in $H'$ with $T$. Thus, $(V_1, \dots, V_{k-1}, T)$ form the desired $\compoverset{k}{t}$ in the original graph $H$.
\end{enumerate}

The pseudocode is given in Algorithm \ref{alg:kpartite}.

\begin{algorithm}[H]
    \caption{Finding a balanced partite $k$-graph}
    \label{alg:kpartite}
    \begin{algorithmic}[1]
        \Function{Find\_Partite}{$H, k$}
            \If {$k = 1$}
                \State \Return $(\{x \colon \{x\} \in E(H)\})$
            \EndIf

            \State $n \gets |V(H)|$, $m \gets |E(H)|$, $d \gets m/n^k$
            \State $t \gets t(n, d, k)$, $w \gets w(n, d, k)$, $s \gets s(n, d, k)$
            \State \textbf{assert} $t \ge 2$

            \State $W \gets$ a set of $w$ vertices with highest degree in $H$
            \ForAll{$T \in \binom{W}{t}$}
                \State $S \gets \{\,y \in \binom{V\setminus W}{k-1} \colon \forall x \in T, \{x\} \cup y \in E(H)\,\}$
                \If{$|S| \ge s$}
                    \State $H' \gets (V \setminus W, S)$  \Comment{$H'$ is a $(k-1)$-graph}
                    \State $(V_1, \dots, V_{k-1}) \gets$ \Call{Find\_Partite}{$H', k-1$}
                    \State \Return $(V_1, \dots, V_{k-1}, T)$
                \EndIf
            \EndFor
        \EndFunction
    \end{algorithmic}
\end{algorithm}

\section{Analysis}

We briefly sketch the proof of correctness and polynomial runtime for our algorithm. The full, detailed proofs can be found in the first author's master's thesis \cite{Espuna2025}. We assume $t \ge 2$, which holds if $d$ is not too small (e.g., $d = \Omega(n^{-1/2^{(k-1)}})$).

\subsection{Correctness}

The correctness of the algorithm hinges on two key lemmas, which we state here without proof.
\begin{enumerate}
    \item There always exists a $t$-subset $T \subseteq W$ for which the set $S$ of common neighbors satisfies $|S| \ge s$.
    \item The recursive call \texttt{Find\_Partite($H'$, $k-1$)} is guaranteed to find parts of size at least $t$.
\end{enumerate}

The first claim follows from a double-counting argument akin to the proof of the Kővári-Sós-Turán theorem. We define a bipartite graph between the $(k-1)$-subsets of $V \setminus W$ and the vertices in $W$. By showing this graph is dense, we can guarantee the existence of a complete bipartite subgraph $K(s,t)$, which corresponds to finding the desired set $T$ and a large corresponding set $S$.

For the second claim, we show that the new hypergraph $H'=(V \setminus W, S)$ is sufficiently dense. Its density $d'$ satisfies $d' \ge d^t$. We then prove that the target part size for the recursive call, $t' = t(n-w, d', k-1)$, satisfies $t' \ge t$. This ensures that the part sizes do not shrink during recursion. For the base case $k=2 \to k=1$, we show directly that $|S| \ge t$.

\subsection{Complexity}

The algorithm runs in polynomial time in $n$. The dominant cost at each recursive level comes from the loop over all $t$-subsets of $W$. The number of iterations is $\binom{w}{t}$. Since $w = \lceil 2t/d \rceil$ and $t = \Theta((\log n)^{1/(k-1)})$, the number of iterations can be bounded:
\[
    \binom{w}{t} \le \left(\frac{ew}{t}\right)^t \le \left(\frac{e(2t/d + 1)}{t}\right)^t \approx \left(\frac{3e}{d}\right)^t.
\]
Substituting the definition of $t$, this expression is polynomial in $n$ (e.g., $\bigO{n^c}$ for some constant $c$). Inside the loop, constructing the set $S$ and the new hypergraph $H'$ takes polynomial time. Since the recursion depth is fixed at $k-1$, the total runtime is a nested polynomial, which remains polynomial in $n$. A detailed analysis shows the complexity is roughly $\bigO{n^{2k+3}}$.

\section{Conclusion and Future Work}

We have presented a deterministic, polynomial-time algorithm to find a large complete balanced $k$-partite subgraph in any sufficiently dense $k$-uniform hypergraph. This provides a constructive counterpart to a classical existence result by Erdős in extremal hypergraph theory.

Several avenues for future research remain open.
\begin{itemize}
    \item \textbf{General Blow-ups:} Our algorithm finds a blow-up of a single edge, $\compoverset{k}{t}$. Can this framework be adapted to find a $t_n$-blowup of an arbitrary fixed $k$-graph $G$? Existence theorems guarantee such structures, but efficient algorithms are lacking.
    \item \textbf{Unbalanced Partite Graphs:} The algorithm could be modified to search for unbalanced complete partite graphs $\compdots{t_1}{t_k}$, where the part sizes may grow at different rates.
    \item \textbf{Optimality:} The bounds on $t$ are asymptotically tight, but the constants can likely be improved with a more refined analysis. For $k=2$, it is known that in dense graphs one can find a $t=\Theta(\log n)$ blow-up of any bipartite graph. It is an open question if a constructive proof for this stronger result exists for $k \ge 2$.
\end{itemize}

\begin{thebibliography}{99}

\bibitem{baber2011hypergraphs}
R. Baber and J. Talbot.
Hypergraphs do jump.
\textit{Combinatorics, Probability and Computing}, 20(2):161--171, 2011.

\bibitem{ball2012asymptotic}
S. Ball and V. Pepe.
Asymptotic improvements to the lower bound of certain bipartite Turán numbers.
\textit{Combinatorics, Probability and Computing}, 21(3):323--329, 2012.

\bibitem{bollobas1973structure}
B. Bollobás and P. Erdős.
On the structure of edge graphs.
\textit{Bulletin of the London Mathematical Society}, 5(3):317–321, 1973.

\bibitem{brown1966graphs}
W. G. Brown.
On graphs that do not contain a Thomsen graph.
\textit{Canadian Mathematical Bulletin}, 9:281--285, 1966.

\bibitem{carvajal2024canonical}
M. A. Carvajal, G. Santos, and M. Schacht.
Canonical Ramsey numbers for partite hypergraphs.
\textit{arXiv preprint arXiv:2411.16218}, 2024.

\bibitem{conlon2020random}
D. Conlon, C. Pohoata, and D. Zakharov.
Random multilinear maps and the Erdős box problem.
\textit{Discrete Analysis}, Paper No. 17, 8, 2021.

\bibitem{de2000maximum}
D. De Caen and Z. Füredi.
The maximum size of 3-uniform hypergraphs not containing a Fano plane.
\textit{Journal of Combinatorial Theory, Series B}, 78(2):274--276, 2000.

\bibitem{Erods1964}
P. Erdős.
On extremal problems of graphs and generalized graphs.
\textit{Israel Journal of Mathematics}, 2:183–190, 1964.

\bibitem{erdos1946structure}
P. Erdős and A. H. Stone.
On the structure of linear graphs.
\textit{Bulletin of the American Mathematical Society}, 52:1087--1091, 1946.

\bibitem{erdHos1966problem}
P. Erdős, A. Rényi, and V. T. Sós.
On a problem of graph theory.
\textit{Studia Scientiarum Mathematicarum Hungarica}, 1:215--235, 1966.

\bibitem{erdHos1983supersaturated}
P. Erdős and M. Simonovits.
Supersaturated graphs and hypergraphs.
\textit{Combinatorica}, 3(2):181--192, 1983.

\bibitem{frankl1984exact}
P. Frankl and Z. Füredi.
An exact result for 3-graphs.
\textit{Discrete Mathematics}, 50(2-3):323--328, 1984.

\bibitem{furedi1996new}
Z. Füredi.
New asymptotics for bipartite Turán numbers.
\textit{Journal of Combinatorial Theory, Series A}, 75(1):141--144, 1996.

\bibitem{Hylten1958}
C. Hyltén-Cavallius.
On a combinatorical problem.
\textit{Colloquium Mathematicum}, 6:59--65, 1958.

\bibitem{keevash2011hypergraph}
P. Keevash.
Hypergraph Turán problems.
In \textit{Surveys in combinatorics 2011}, volume 392 of \textit{London Mathematical Society Lecture Note Series}, pages 83--139. Cambridge University Press, 2011.

\bibitem{kollar1996norm}
J. Kollár, L. Rónyai, and T. Szabó.
Norm-graphs and bipartite Turán numbers.
\textit{Combinatorica}, 16(3):399--406, 1996.

\bibitem{Kovari1954}
T. Kővári, V. T. Sós, and P. Turán.
On a problem of K. Zarankiewicz.
\textit{Colloquium Mathematicum}, 3:50–57, 1954.

\bibitem{krivelevich1998chromatic}
M. Krivelevich and B. Sudakov.
The chromatic numbers of random hypergraphs.
\textit{Random Structures \& Algorithms}, 12(4):381--403, 1998.

\bibitem{MUBAYI2010174}
D. Mubayi and G. Turán.
Finding bipartite subgraphs efficiently.
\textit{arXiv preprint arXiv:0905.2527}, 2009.

\bibitem{razborov2007flag}
A. A. Razborov.
Flag algebras.
\textit{The Journal of Symbolic Logic}, 72(4):1239--1282, 2007.

\bibitem{razborov20103}
A. A. Razborov.
On 3-hypergraphs with forbidden 4-vertex configurations.
\textit{SIAM Journal on Discrete Mathematics}, 24(3):946--963, 2010.

\bibitem{reingold1977combinatorial}
E. M. Reingold, J. Nievergelt, and N. Deo.
\textit{Combinatorial algorithms: theory and practice}.
Prentice-Hall, Inc., 1977.

\bibitem{rodl2012complete}
V. R"odl and M. Schacht.
Complete partite subgraphs in dense hypergraphs.
\textit{Random Structures \& Algorithms}, 41(4):557--573, 2012.

\bibitem{Turan1941}
P. Turán.
Eine Extremalaufgabe aus der Graphentheorie.
\textit{Matematikai és Fizikai Lapok}, 48:436--452, 1941.

\bibitem{zarankiewicz1951problemes}
K. Zarankiewicz.
Problem 101.
\textit{Colloquium Mathematicae}, 2(3-4):301, 1951.

\end{thebibliography}

\end{document}