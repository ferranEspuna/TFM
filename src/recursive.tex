%! Author = Ferran

% Preamble
\documentclass[12pt]{article}

% Packages
\usepackage{amsmath}
\usepackage{parskip}
\usepackage[margin=70pt]{geometry}
\usepackage{amsthm}
\usepackage{mathtools}
\author{Ferran Espuña}
\date{} % clear date


\newtheorem{thm}{Theorem}
\newtheorem{lemma}[thm]{Lemma}



% Document
\begin{document}

    \title{Extending Mubayi and Turán's Algorithm to $k$-graphs}

    \maketitle

    Let $G$ be an $r$-graph with $n$ vertices and $m = dn^k$ edges.
    A polynomial time algorithm is given to find a $K_{q,\, \ldots\, ,\, q} \subset G$ for

    \begin{align*}
        q (k, d) = % TODO: Insert formula
        \label{eq:q}
    \end{align*}

    As long as (insert condition here). % TODO: Insert condition
    
    Note that this result is tight up to the constant $c(k, d)$,
    as proved in~\cite{Erods1964}.
    This result is a generalization of the result in $2$-graphs by~\cite{MUBAYI2010174},
    and algorithm will be analogous to the one given there.
    We will call this algorithm \verb|FIND_PARTITE|$(k, \cdot)$.
    The procedure is as follows:

    \begin{enumerate}
        \item Choose parameters $q, r, s$ depending on $n, k$ and $d$.
        \item Find the set $R$ of $r$ vertices with the highest degree in $G$.

        \item find a subset $Q \subset R$ with $q$ vertices and a
        $S \subset T \coloneqq \binom{[n] \setminus Q}{k-1}$ with $s$ edges satisfying
        \[\{x_1, x_2, \ldots, x_k\} \in E(G) \; \forall \, \{x_2, \ldots, x_k\} \in S, \, x_1 \in Q\]

        \item The set $S$ induces a $(k-1)$-graph $G'$ on $T$.
        Evaluate \verb|FIND_PARTITE|$(k-1, G')$ to find a $K_{q',\, \ldots\, ,\, q'}$ in $G'$
        (say, $\{U_1, \ldots, U_{k-1}\}$). We will prove that $q' \geq q$,
        and because of the condition for $S$ we can return $\{Q, V_1, \ldots V_{k-1}\}$
        where $V_i$ is a subset of $U_i$ of size $q$.

    \end{enumerate}

    Our goal is now to verify the correctness of the algorithm and to find the parameters $q, r, s$
    such that the above procedure is successful, and to prove that it runs in polynomial time.

    \begin{lemma}\label{many_edges}

    \end{lemma}








    \bibliography{main}
    \bibliographystyle{plain}

\end{document}
