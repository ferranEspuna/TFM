\section{Our Algorithm}\label{sec:algorithm}

In this section we present a polynomial-time algorithm to find a balanced partite $k$-graph in a given $k$-graph $G$
with $n$ vertices and $m$ edges with part size in the same order of magnitude as stated in
Theorem~\ref{thm:erdos64}.

\begin{remark}
    If we let $q$ be the size of each part in the partite $k$-graph we are looking for, we need
    \[
        m \geq ex(n, K(t, \overset{k}{\cdots}, t)) = O\left(n^{k - \frac{1}{t^{k-1}}}\right)
    \]
    Defining $d = \frac{m}{n^k}$, and taking logarithms, this is true iff
    \[
        \log d \geq - \frac{\log n}{t^{k-1}} + O(1)
    \]
    which implies
    \[
        t = O\left(\left(\frac{\log n}{\log (1/d)}\right)^{\frac{1}{k-1}}\right)
    \]
\end{remark}

This algorithmic problem has already been solved for $k = 2$ by Mubayi and Turán~\cite{MUBAYI2010174}.
The algorithm in that case follows very closely the structure of the proof of Theorem~\ref{thm:kst}.
We outline the algorithm for $k = 2$ here for context and clarity.
The variable names have been altered to match the notation used in this thesis.

\begin{algorithm}
    \caption{Finding a balanced bipartite graph in a $2$-graph}
    \label{alg:bipartite}
        \begin{algorithmic}[1]
        \Require A graph G = $(V, E)$ with $|V| = n$, $E = m$
        \State $d \gets m / n^2 $
        \State \textbf{assert} $d \geq 3 n^{1/2}$
        \State $t \gets \left\lfloor\frac{\log (n/2) }{\log (2e/d)}\right\rfloor,\, w \gets \lfloor t/d \rfloor$
        \State $W \gets $ The set of $t$ vertices with highest degree in $G$
        \State $U \gets V \setminus W$
        \ForAll{$T \in \binom{W}{t} $}
            \State $S \gets \{x \in U : {x, y} \in E \text{ for all } y \in T\}$
            \If{$|S| \geq t$}
                \State \Return $(S, T)$
            \EndIf
        \EndFor
        \end{algorithmic}
\end{algorithm}

If the set $S$ is too large we can simply take a subset of it of size $t$.
The algorithm is correct if at some point it returns a pair of sets $(S, T)$.
The argument of why this is the case boils down to showing that there is a
large number of edges between $U$ and $W$ and then applying Theorem~\ref{thm:kst}
with $u = |U| = n - w$ and $s = t$ .
Finally, the algorithm runs in polynomial time because 
the  number of iterations of the loop is
\[
    \binom{w}{t} \leq \left(\frac{ew}{t}\right)^t \leq \left(\frac{1}{d}\right)^t e^t < e^{t \log (1/d) + \log n} < e^{2\log n} = n^2
\]

\begin{remark}
    The requirement for a minimum density is because if $d = o\left(n^{1/2}\right)$ then
    there may not even be a $K(2, 2)$ in $G$. % TODO: Explain this better
\end{remark}

We will now describe an extended algorithm which will find a balanced partite $k$-graph in a $k$-graph $G$ with
$n$ vertices and $m = d n^k$ with part size

\[
    t (n, d, k) = \left\lfloor \left(  \frac{\log (n/2^k)}{\log (2^{k+1}/d)} \right)^{\frac{1}{k-1}} \right\rfloor
\]

\begin{algorithm}
    \caption{Finding a balanced partite $k$-graph in a $k$-graph}
    \label{alg:kpartite}
    \begin{algorithmic}[1]
        \Function{FIND\_PARTITE}{$G, k$}
            \State \textbf{assert} G is a $k$-graph
            \If {$k = 1$}
                \State \Return $\{x : \{x\} \in E(G)\}$
            \EndIf

            \State \textbf{assert} $d \geq 4^k n^{-\frac{1}{2^{k-1}}}$ \label{line:min_density}
            \State $n \gets |V(G)|,\, m \gets |E(G)|,\, d \gets m/n^k$
            \State $t \gets t(n, d, k),\, w \gets \lceil 2t/d \rceil,\, s \gets \lfloor d^t n^{k-1} \rfloor$ \label{line:tws}
            \State $W \gets$ the set of $t$ vertices with highest degree in $G$ \label{line:W}
            \State $U \gets \binom{V\setminus W}{k-1}$

            \ForAll{$T \in \binom{W}{t}$} \label{line:for}
                \State $S \gets \{\,y \in U : \{x\} \cup y \in E \text{ for all } x \in T\,\}$
                \If{$|S| \geq s$}
                    \State \Return $(S)\, \circ $\Call{FIND\_PARTITE\,}{$G' \coloneqq (V \setminus W,\, S),\, k-1$} \label{line:return}
                \EndIf
            \EndFor
        \EndFunction
    \end{algorithmic}
\end{algorithm}

Where the operator $\circ$ denotes the concatenation of tuples, and we
understand a $1$-graph to be a subset of a set.
We will now argue that this algorithm is correct (as long as the the density condition is met on the first call)
and runs in polynomial time.
The following lemmas are stated assuming that $k \geq 2$ and
the minimum density requirement of line~\ref{line:min_density} of Algorithm~\ref{alg:kpartite} is met.

\begin{remark} % TODO explain better
    The $t=1$ case is trivial, as we can simply select one edge of $G$.
    The minimum density requirement implies $t \geq 2$, although the converse is not true for any $n$.
    However, it \emph{is} true that, for all $k$, $t \geq 3$ implies the minimum density requirement
    for $n \geq n_0(k)$.
    Therefore, we can simply solve the problem for $t = 2$ or $n < n_0(k)$
    by checking all possible embeddings (which clearly can be done in polynomial time for every $k$) and apply~\ref{alg:kpartite} otherwise.
\end{remark}

\begin{lemma}
    The selection of $t, w, s$ in line~\ref{line:tws} is sound in the sense that
    $t  \leq w \leq n$, $k - 1 \leq n - w$ and $s \leq \binom{n - w}{k - 1}$.
    \begin{proof}
        $t \leq w$ is clear.
        We will in fact show that $w < \frac{n}{2}$.
        If not,
        \[
            \frac{n}{2} \leq
            w \leq 1 + \frac{2t}{d} <
            1 + \frac{2 \log (n/2^k)}{4^k n^{-\frac{1}{2^{k-1}}}} <
            + \frac{\log(n)}{2^{2k-1} n^{-\frac{1}{2}}} =
            1 + \frac{\log n \cdot \sqrt {n}}{8} <
            1 + \frac{n}{8} \implies n < 3
        \]
        But because we have positive density this implies $n = k = 2$, with at most one edge, which clearly
        does not satisfy the density condition.

        We can also show that $k < \frac{n}{2}$.
        If not,
        \[
            1 \geq
            d \geq
            4^{\frac{n}{2}} n^{-\frac{1}{2^{n/2-1}}} \geq
            e^{\frac{n}{2} \log 4 - \frac{\log n}{2^{n/2-1}}} \implies
            \frac{n}{2} \log 4 \leq \frac{\log n}{2^{n/2-1}}
        \]
        which is false for all $n > 0$. % TODO show this

        Therefore,
        \[
            k + w < n \implies k - 1 < n - w
        \]

        Finally, suppose $s > \binom{n - w}{k - 1}$.
        Then, using the fact that $w < \frac{n}{2}$,
        \[
            \left( \frac{n}{2k} \right)^{k-1} \leq \left( \frac{n-w}{k-1} \right)^{k-1} \leq \binom{n - w}{k - 1} < s \leq d^t n^{k-1}
            \implies  \left( \frac{1}{2k} \right)^{k-1} < d^t \leq \left( \frac{1}{k!} \right)^2
        \]
        Where in the last inequality we use that $t \geq 2$ and there are at most
        $\binom{n}{k} \leq \frac{n^k}{k!}$ edges in $G$.

        We can show that $k!^2 \geq (2k)^{k-1}$ for all $k$ % TODO proof by freaking look at it!!
        , which means we have reached a contradiction. \qedhere

    \end{proof}
\end{lemma}

\begin{lemma}
    With $W  \subset V$ as defined in line~\ref{line:W},
    There are at least $\frac{3}{2}dwn^{k-1}$ edges of $G$ with exactly one vertex in $W$.
    \begin{proof}
        The degree sum over $V$ is $kdn^{k}$.
        Thus, by the pigeonhole principle, the degree sum over $W$ is at least
        $\frac{w}{n}kdn^{k} = wkdn^{k-1}$.
        For $2 \leq j \leq n$, consider the contribution to this sum by edges with exactly $j$ vertices in $W$.
        Each such edge contributes $j$ to the sum, and there are at most
        $\binom{w}{j}\binom{n-w}{k-j} \leq \frac{w^j n^{k-j}}{j!} \leq \frac{w^j n^{k-j}}{j}$ of them.
        Thus, the total contribution of these edges is at most $w^j n^{k-j} \leq w^{2}n^{k-2}$.
        The number of edges with only one vertex in $W$ is then at least

        \[
            wkdn^{k-1} - (k-1)w^{2}n^{k-2} = dwn^{k-1} \left( k - \frac{(k-1)w}{nd}\right)
        \]

        Suppose, by way of contradiction,
        that $ k - \frac{(k-1)w}{nd} < \frac{3}{2}$.
        Using that $\frac{k-1}{k-3/2} \leq 2$
        for $k \geq 2$, we arrive at

        \[
            2 \geq  \frac{nd}{w} \implies d \leq \frac{2w}{n} = \frac{2 \left\lceil\frac{2t}{d} \right\rceil}{n}
            < \frac{6t}{dn}
        \]

        Where the last inequality follows from the fact that $t > 1$ and $d \leq 1$.
        Rearranging, and applying the density requirement, if $k \geq 3$ we have
        \[
            16 \sqrt {n} \leq 4^{2k} n \cdot n^{-\frac{2}{2^{k-1}}} \leq nd^2 < 6t < 6 \log n
        \]
        Which is false for all $n$.
        We have to be more careful in the $k = 2$ case.

        TODO % TODO

    \end{proof}
\end{lemma}

\begin{lemma}
    Line~\ref{line:return} of Algorithm~\ref{alg:kpartite} is reached at some point in the for
    loop in line~\ref{line:for}.
    \begin{proof}
        TODO % TODO
    \end{proof}
\end{lemma}

Now, for the base case, we need:

\begin{lemma}
    For $k=2$, Algorithm~\ref{alg:kpartite} finds $s \geq t$.
    \begin{proof}
        TODO % TODO
    \end{proof}
\end{lemma}

For the recursive step, we need:

\begin{lemma}
    For $k \geq 3$, in the recursive call in line~\ref{line:return} of Algorithm~\ref{alg:kpartite},
    the density condition of line~\ref{line:min_density} is met for $k-1$.
    That is,
    \[
        d' \coloneqq \frac{s}{(n-w)^{k-1}} \geq 4^{k-1} (n-w)^{-\frac{1}{2^{k-2}}}
    \]
    Furthermore,
    the resulting part size $t'$ satisfies
    \[
        t' \coloneqq t(n - w, d', k - 1) \geq t
    \]
    \begin{proof}
        TODO % TODO
    \end{proof}

\end{lemma}

All in all, we can now state the following theorem:

\begin{theorem}
    Algorithm~\ref{alg:kpartite} finds a balanced partite $k$-graph in a $k$-graph $G$ with
    $n$ vertices and $m = d n^k$ with part size $t(n, d, k)$ in polynomial time.
    \begin{proof}
        TODO % TODO
    \end{proof}
\end{theorem}





