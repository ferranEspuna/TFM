\documentclass{beamer}
\usepackage{amsmath,amssymb,amsfonts}
\usepackage{tikz}
\usetikzlibrary{calc}
\usepackage{booktabs}
\usepackage{blkarray} % For tables if needed, though likely not for a 20-min talk
\usetikzlibrary{fit}

% Thesis specific commands (simplified or as needed)
\newcommand{\N}{\ensuremath{\mathbb{N}}}
\newcommand{\ex}[2]{\ensuremath{\text{ex} \left( #1, #2 \right)}}
\newcommand{\compoverset}[2]{\ensuremath{K\left(#2, \overset{#1}{\dots}, #2\right)}} % k-partite, k parts of size #2
\newcommand{\compdots}[2]{\ensuremath{K\left(#1, \dots, #2\right)}} % k-partite, part sizes #1 to #2
\newcommand{\bigO}[1]{\ensuremath{\mathcal{O}\left(#1\right)}}
\newcommand{\OmegaBig}[1]{\ensuremath{\Omega\left(#1\right)}} % For asymptotic lower bound
\newcommand{\completesuperindex}[2]{\ensuremath{K^{(#1)}_{#2}}} % Complete k-graph on #2 vertices
\newcommand{\link}[3]{\ensuremath{L_{#1}\left(#2; #3\right)}} % Link graph

\makeatletter
\def\th@mystyle{%
    \normalfont % body font
    \setbeamercolor{block title example}{bg=orange,fg=white}
    \setbeamercolor{block body example}{bg=orange!20,fg=black}
    \def\inserttheoremblockenv{exampleblock}
  }
\makeatother
\theoremstyle{mystyle}
\newtheorem*{remark}{Remark}

\title[Finding Partite Hypergraphs Efficiently]{Finding Partite Hypergraphs Efficiently}
\author{Ferran Espuña Bertomeu}
\institute{Supervisor: Richard Lang}
\date{June 2025}

\usetheme{Copenhagen} % A common Beamer theme; choose any you like
\usecolortheme{beaver}
\setbeamertemplate{navigation symbols}{} % Remove navigation symbols

\begin{document}

\frame{\titlepage}

\section{Hypergraphs}\label{sec:notation}

\begin{frame}
    \frametitle{$k$-Graphs}
    \begin{definition}
        A \emph{$k$-graph} is a pair $G = (V, E)$
        where $V$ is a finite set of \emph{vertices} and
        $E \subseteq \binom{V}{k}$ is a set of \emph{edges}.
    \end{definition}
    \begin{figure}[htbp]
        \centering
        \scalebox{0.7}{
            % TikZ code for K_4^(3) using predefined edge colors
\begin{tikzpicture}[scale=1]
\coordinate (V1) at (0, 5);
\coordinate (V2) at (5, 5);
\coordinate (V3) at (5, 0);
\coordinate (V4) at (0, 0);
\coordinate (R0) at (3.500, 3.500);
\draw[line width=1.5pt, color=magenta!70!white] (R0) -- (V1);
\draw[line width=1.5pt, color=magenta!70!white] (R0) -- (V2);
\draw[line width=1.5pt, color=magenta!70!white] (R0) -- (V3);
\fill[color=magenta!70!white] (R0) circle (2.5pt);
\coordinate (R1) at (1.500, 3.500);
\draw[line width=1.5pt, color=teal!70!white] (R1) -- (V1);
\draw[line width=1.5pt, color=teal!70!white] (R1) -- (V2);
\draw[line width=1.5pt, color=teal!70!white] (R1) -- (V4);
\fill[color=teal!70!white] (R1) circle (2.5pt);
\coordinate (R2) at (1.500, 1.500);
\draw[line width=1.5pt, color=blue!70!white] (R2) -- (V1);
\draw[line width=1.5pt, color=blue!70!white] (R2) -- (V3);
\draw[line width=1.5pt, color=blue!70!white] (R2) -- (V4);
\fill[color=blue!70!white] (R2) circle (2.5pt);
\coordinate (R3) at (3.500, 1.500);
\draw[line width=1.5pt, color=orange!70!white] (R3) -- (V2);
\draw[line width=1.5pt, color=orange!70!white] (R3) -- (V3);
\draw[line width=1.5pt, color=orange!70!white] (R3) -- (V4);
\fill[color=orange!70!white] (R3) circle (2.5pt);
\fill[black] (V1) circle (4.0pt);
\fill[black] (V2) circle (4.0pt);
\fill[black] (V3) circle (4.0pt);
\fill[black] (V4) circle (4.0pt);
\end{tikzpicture}
        }
        \caption{A complete $3$-graph on $4$ vertices: $K_4^{(3)}$.}
        \label{fig:complete_kgraph}
    \end{figure}
\end{frame}

\begin{frame}
    \frametitle{Partite $k$-Graphs}
    \begin{definition}
        A $k$-graph $G = (V, E)$ is \emph{$r$-partite}
        if there exists a partition $V = V_1 \cup \dots \cup V_r$
        such that every edge of $G$ intersects every part $V_i$ in at most one vertex.
        We write $G = (V_1, \dots, V_r; E)$.
    \end{definition}

    \begin{figure}[htbp]
        \centering
        \scalebox{0.6}{
            % TikZ code for K^(2)(2, 2, 2) using predefined edge colors
\pgfdeclarelayer{background}
\pgfdeclarelayer{main}
\pgfsetlayers{background,main}
\begin{tikzpicture}[scale=0.8]
\begin{pgfonlayer}{background}
  \draw[fill=gray!20, rounded corners] (0.30, 0.30) rectangle (1.70, 3.70);
\end{pgfonlayer}
\node at (1.00, 2.00) [align=center] {$V_1$};
\begin{pgfonlayer}{background}
  \draw[fill=gray!20, rounded corners] (8.30, 0.30) rectangle (9.70, 3.70);
\end{pgfonlayer}
\node at (9.00, 2.00) [align=center] {$V_2$};
\begin{pgfonlayer}{background}
  \draw[fill=gray!20, rounded corners] (4.30, 5.30) rectangle (5.70, 8.70);
\end{pgfonlayer}
\node at (5.00, 7.00) [align=center] {$V_3$};
\coordinate (A1) at (1, 1);
\coordinate (A2) at (1, 3);
\coordinate (B1) at (9, 1);
\coordinate (B2) at (9, 3);
\coordinate (C1) at (5, 6);
\coordinate (C2) at (5, 8);
\draw[line width=1.5pt, color=magenta!60!white] (A1) -- (B1);
\draw[line width=1.5pt, color=magenta!60!white] (A1) -- (B2);
\draw[line width=1.5pt, color=magenta!60!white] (A2) -- (B1);
\draw[line width=1.5pt, color=magenta!60!white] (A2) -- (B2);
\draw[line width=1.5pt, color=teal!60!white] (A1) -- (C1);
\draw[line width=1.5pt, color=teal!60!white] (A1) -- (C2);
\draw[line width=1.5pt, color=teal!60!white] (A2) -- (C1);
\draw[line width=1.5pt, color=teal!60!white] (A2) -- (C2);
\draw[line width=1.5pt, color=orange!60!white] (B1) -- (C1);
\draw[line width=1.5pt, color=orange!60!white] (B1) -- (C2);
\draw[line width=1.5pt, color=orange!60!white] (B2) -- (C1);
\draw[line width=1.5pt, color=orange!60!white] (B2) -- (C2);
\begin{pgfonlayer}{main}
  \fill[black] (A1) circle (4.0pt);
  \fill[black] (A2) circle (4.0pt);
  \fill[black] (B1) circle (4.0pt);
  \fill[black] (B2) circle (4.0pt);
  \fill[black] (C1) circle (4.0pt);
  \fill[black] (C2) circle (4.0pt);
\end{pgfonlayer}
\end{tikzpicture}
        }
        \caption{A complete $3$-partite $2$-graph: $K^{(3)}(2, 2, 2)$.}
        \label{fig:complete_3partite_2graph}
    \end{figure}
\end{frame}

\begin{frame}
    \frametitle{Partite $k$-Graphs}
    \begin{remark}
        We may identify $E$ as a subset of $\mathcal{C} = \bigcup_{\{i_1, \dots, i_k \} \in \binom{[r]}{k}} V_{i_1} \times \dots \times V_{i_k}$.
        If $E = \mathcal{C}$, we say that $G$ is a \emph{complete} $r$-partite $k$-graph.
    \end{remark}

    \begin{figure}[htbp]
        \centering
        \scalebox{0.55}{
            % TikZ code for K^(3)(2, 2, 2) using predefined edge colors
\begin{tikzpicture}[scale=1]
\draw[fill=gray!20, rounded corners] (-0.50, -0.50) rectangle (2.50, 2.50);
\node at (1.00, 1.00) [align=center] {$V_1$};
\draw[fill=gray!20, rounded corners] (7.50, -0.50) rectangle (10.50, 2.50);
\node at (9.00, 1.00) [align=center] {$V_2$};
\draw[fill=gray!20, rounded corners] (1.50, 5.50) rectangle (8.50, 6.50);
\node at (5.00, 6.00) [align=center] {$V_3$};
\coordinate (A1) at (2, 0);
\coordinate (A2) at (0, 2);
\coordinate (B1) at (8, 0);
\coordinate (B2) at (10, 2);
\coordinate (C1) at (2, 6);
\coordinate (C2) at (8, 6);
\coordinate (R0) at (3.333, 2.148);
\draw[line width=1.5pt, color=blue!70!white] (R0) -- (A1);
\draw[line width=1.5pt, color=blue!70!white] (R0) -- (B1);
\draw[line width=1.5pt, color=blue!70!white] (R0) -- (C1);
\fill[color=blue!70!white] (R0) circle (2.5pt);
\coordinate (R1) at (6.667, 2.148);
\draw[line width=1.5pt, color=blue!70!white] (R1) -- (A1);
\draw[line width=1.5pt, color=blue!70!white] (R1) -- (B1);
\draw[line width=1.5pt, color=blue!70!white] (R1) -- (C2);
\fill[color=blue!70!white] (R1) circle (2.5pt);
\coordinate (R2) at (4.222, 3.037);
\draw[line width=1.5pt, color=blue!70!white] (R2) -- (A1);
\draw[line width=1.5pt, color=blue!70!white] (R2) -- (B2);
\draw[line width=1.5pt, color=blue!70!white] (R2) -- (C1);
\fill[color=blue!70!white] (R2) circle (2.5pt);
\coordinate (R3) at (7.556, 3.037);
\draw[line width=1.5pt, color=blue!70!white] (R3) -- (A1);
\draw[line width=1.5pt, color=blue!70!white] (R3) -- (B2);
\draw[line width=1.5pt, color=blue!70!white] (R3) -- (C2);
\fill[color=blue!70!white] (R3) circle (2.5pt);
\coordinate (R4) at (2.444, 3.037);
\draw[line width=1.5pt, color=blue!70!white] (R4) -- (A2);
\draw[line width=1.5pt, color=blue!70!white] (R4) -- (B1);
\draw[line width=1.5pt, color=blue!70!white] (R4) -- (C1);
\fill[color=blue!70!white] (R4) circle (2.5pt);
\coordinate (R5) at (5.778, 3.037);
\draw[line width=1.5pt, color=blue!70!white] (R5) -- (A2);
\draw[line width=1.5pt, color=blue!70!white] (R5) -- (B1);
\draw[line width=1.5pt, color=blue!70!white] (R5) -- (C2);
\fill[color=blue!70!white] (R5) circle (2.5pt);
\coordinate (R6) at (3.333, 3.926);
\draw[line width=1.5pt, color=blue!70!white] (R6) -- (A2);
\draw[line width=1.5pt, color=blue!70!white] (R6) -- (B2);
\draw[line width=1.5pt, color=blue!70!white] (R6) -- (C1);
\fill[color=blue!70!white] (R6) circle (2.5pt);
\coordinate (R7) at (6.667, 3.926);
\draw[line width=1.5pt, color=blue!70!white] (R7) -- (A2);
\draw[line width=1.5pt, color=blue!70!white] (R7) -- (B2);
\draw[line width=1.5pt, color=blue!70!white] (R7) -- (C2);
\fill[color=blue!70!white] (R7) circle (2.5pt);
\fill[black] (A1) circle (4.0pt);
\fill[black] (A2) circle (4.0pt);
\fill[black] (B1) circle (4.0pt);
\fill[black] (B2) circle (4.0pt);
\fill[black] (C1) circle (4.0pt);
\fill[black] (C2) circle (4.0pt);
\end{tikzpicture}
        }
        \caption{A complete $3$-partite $3$-graph: $K^{(2)}(2, 2, 2)$.}
        \label{fig:complete_3partite_3graph}
    \end{figure}
\end{frame}

\section{Turán-Type Problems}\label{sec:turan-type}

\begin{frame}
    \frametitle{Turán-Type Problems}
    \begin{definition}
        Let $G = (V, E)$ be a $k$-graph and $n \geq |V|$ an integer.
        The \emph{Turán number} $\ex{G}{n}$ is the maximum number of edges in a $k$-graph on $n$ vertices
        that does not contain a copy of $G$ as a subgraph.
    \end{definition}
    Determining $\ex{G}{n}$ or estimating it as $n \to \infty$ is known as the \emph{Turán problem} for $G$.
    \begin{theorem}
        For all $k$-graphs $G$ there exists a constant $\alpha (G) \in [0, 1)$ such that
        \[
            \ex{G}{n} = (\alpha (G) + o(1)) \cdot \binom{n}{k} \quad \text{as } n \to \infty.
        \]
        Furthermore, $\alpha (G) = 0$ if and only if $G$ is $k$-partite.
    \end{theorem}
\end{frame}

\begin{frame}
    \frametitle{The Kővari--Sós--Turán Theorem}
    The bound $\ex{G}{n} = o(n^k)$ can be improved by a lot.
    \begin{definition}
     Let $1 < t_1 \leq v_1, \dots, 1 < t_k \leq v_k$ be integers.
        Then the \emph{generalized Zarankiewicz number} $z(v_1, \dots, v_k; t_1, \dots, t_k)$
        is the largest integer $z$ for which there exists a $k$-partite $k$-graph
        $H = (V_1, \dots V_k, F)$ with part sizes $ |V_i| = v_i$ and $|F| = z$ edges
        such that for all choices of $W_i \subset V_i$ of sizes $|W_i| = t_i$,
        $W_1 \times \dots \times W_k \not \subset F$.
    \end{definition}
    \begin{theorem}[Kővari--Sós--Turán]
        Let $0 < s \leq u$ and $0 < t \leq w$ be integers.
        Then
        \[z(u, w; s, t) \leq (s - 1)^{1 / t}(w - t + 1)u^{1 - 1 / t} + (t - 1)u\]
    \end{theorem}
    Standard arguments then show that
    $\ex{n}{K(s, t)} = \bigO{n^{2 - 1 / t}}$.
\end{frame}

\begin{frame}[fragile]
    \vspace*{-1em}
    \frametitle{Kővari–Sós–Turán: Proof Sketch and Example}

    \begin{columns}[T]

        \begin{column}{.35\textwidth}
            \begin{figure}
                \centering
                % TikZ code for KST proof sketch, v22 (granular highlights)
\begin{tikzpicture}[scale=0.8, every node/.style={transform shape, scale=0.8}]
\pgfdeclarelayer{background}\pgfdeclarelayer{main}\pgfsetlayers{background,main}
\coordinate (U0) at (0, 4.0);
\coordinate (U1) at (0, 3.0);
\coordinate (U2) at (0, 2.0);
\coordinate (U3) at (0, 1.0);
\coordinate (U4) at (0, 0.0);
\coordinate (W0) at (4.5, 3.5);
\coordinate (W1) at (4.5, 2.5);
\coordinate (W2) at (4.5, 1.5);
\coordinate (W3) at (4.5, 0.5);
\begin{pgfonlayer}{main}
\uncover<1->{
  \node[draw, thick, fill=yellow!20, rounded corners, align=center, text width=6.5cm] at (2.25, 5.0) {This graph has the maximum number of edges ($|E|=13$) to be $K_{3,2}$-free.};
  \draw[line width=0.7pt, black!25] (U0) -- (W0);
  \draw[line width=0.7pt, black!25] (U0) -- (W1);
  \draw[line width=0.7pt, black!25] (U0) -- (W2);
  \draw[line width=0.7pt, black!25] (U0) -- (W3);
  \draw[line width=0.7pt, black!25] (U1) -- (W0);
  \draw[line width=0.7pt, black!25] (U1) -- (W1);
  \draw[line width=0.7pt, black!25] (U1) -- (W2);
  \draw[line width=0.7pt, black!25] (U2) -- (W0);
  \draw[line width=0.7pt, black!25] (U2) -- (W3);
  \draw[line width=0.7pt, black!25] (U3) -- (W1);
  \draw[line width=0.7pt, black!25] (U3) -- (W3);
  \draw[line width=0.7pt, black!25] (U4) -- (W2);
  \draw[line width=0.7pt, black!25] (U4) -- (W3);
  \fill[black] (U0) circle (2.8pt) node[anchor=east] {$U_{1}$};
  \fill[black] (U1) circle (2.8pt) node[anchor=east] {$U_{2}$};
  \fill[black] (U2) circle (2.8pt) node[anchor=east] {$U_{3}$};
  \fill[black] (U3) circle (2.8pt) node[anchor=east] {$U_{4}$};
  \fill[black] (U4) circle (2.8pt) node[anchor=east] {$U_{5}$};
  \fill[black] (W0) circle (2.8pt) node[anchor=west] {$W_{1}$};
  \fill[black] (W1) circle (2.8pt) node[anchor=west] {$W_{2}$};
  \fill[black] (W2) circle (2.8pt) node[anchor=west] {$W_{3}$};
  \fill[black] (W3) circle (2.8pt) node[anchor=west] {$W_{4}$};
}
\only<2>{
  \fill[brown!40!red] (W1) circle (3.3pt);
  \fill[brown!40!red] (W2) circle (3.3pt);
  \fill[brown!40!red] (U0) circle (3.3pt);
  \fill[brown!40!red] (U1) circle (3.3pt);
  \fill[brown!40!red] (U3) circle (3.3pt);
  \draw[line width=1.0499999999999998pt, brown!40!red, dashed, -] (U3) -- (W2) node[midway, below, sloped, font=\scriptsize] {Add edge};
  \draw[line width=1.0499999999999998pt, brown!40!red] (U0) -- (W1);
  \draw[line width=1.0499999999999998pt, brown!40!red] (U0) -- (W2);
  \draw[line width=1.0499999999999998pt, brown!40!red] (U1) -- (W1);
  \draw[line width=1.0499999999999998pt, brown!40!red] (U1) -- (W2);
  \draw[line width=1.0499999999999998pt, brown!40!red] (U3) -- (W1);
  \node[draw, thick, fill=brown!40!red!20, rounded corners, align=center, text width=6.0cm, overlay] at (2.25, -1.5) {{For example, adding the edge $\{U_4, W_3\}$ creates a $K_{3,2}$ on vertices $\{U_1, U_2, U_4\}$ and $\{W_2, W_3\}$.}};
}
\uncover<3-8>{
  \node[draw, thick, fill=teal!20, rounded corners, align=center, text width=6.0cm, overlay] at (2.25, -1.5) {{For $x=U_1$, we count its $\binom{4}{2}=6$ stars.}};
}
\only<3>{
  \fill[teal] (U0) circle (3.3pt);
  \fill[teal] (W0) circle (3.3pt);
  \fill[teal] (W1) circle (3.3pt);
  \draw[line width=1.0499999999999998pt, teal] (U0) -- (W0);
  \draw[line width=1.0499999999999998pt, teal] (U0) -- (W1);
}
\only<4>{
  \fill[teal] (U0) circle (3.3pt);
  \fill[teal] (W0) circle (3.3pt);
  \fill[teal] (W2) circle (3.3pt);
  \draw[line width=1.0499999999999998pt, teal] (U0) -- (W0);
  \draw[line width=1.0499999999999998pt, teal] (U0) -- (W2);
}
\only<5>{
  \fill[teal] (U0) circle (3.3pt);
  \fill[teal] (W0) circle (3.3pt);
  \fill[teal] (W3) circle (3.3pt);
  \draw[line width=1.0499999999999998pt, teal] (U0) -- (W0);
  \draw[line width=1.0499999999999998pt, teal] (U0) -- (W3);
}
\only<6>{
  \fill[teal] (U0) circle (3.3pt);
  \fill[teal] (W1) circle (3.3pt);
  \fill[teal] (W2) circle (3.3pt);
  \draw[line width=1.0499999999999998pt, teal] (U0) -- (W1);
  \draw[line width=1.0499999999999998pt, teal] (U0) -- (W2);
}
\only<7>{
  \fill[teal] (U0) circle (3.3pt);
  \fill[teal] (W1) circle (3.3pt);
  \fill[teal] (W3) circle (3.3pt);
  \draw[line width=1.0499999999999998pt, teal] (U0) -- (W1);
  \draw[line width=1.0499999999999998pt, teal] (U0) -- (W3);
}
\only<8>{
  \fill[teal] (U0) circle (3.3pt);
  \fill[teal] (W2) circle (3.3pt);
  \fill[teal] (W3) circle (3.3pt);
  \draw[line width=1.0499999999999998pt, teal] (U0) -- (W2);
  \draw[line width=1.0499999999999998pt, teal] (U0) -- (W3);
}
\uncover<9>{
  \node[draw, thick, fill=teal!20, rounded corners, align=center, text width=6.0cm, overlay] at (2.25, -1.5) {{In the example, there are at least $5\binom{13/5}{2} = 10.4$ stars (there are actually 12)}};
}
\uncover<10-12>{
  \node[draw, thick, fill=brown!40!red!20, rounded corners, align=center, text width=6.0cm, overlay] at (2.25, -1.5) {{Each set $T \subset W$ (in this case, $T = \{W_1, W_2\})$ is in at most $s-1 = 3 - 1 = 2$ stars. In total, at most $2 \binom{4}{2}=12$ stars.}};
}
\uncover<11-12>{
  \fill[brown!40!red] (U0) circle (3.3pt);
  \fill[brown!40!red] (W1) circle (3.3pt);
  \fill[brown!40!red] (W2) circle (3.3pt);
}
\uncover<12>{
  \fill[brown!40!red] (U1) circle (3.3pt);
}
\only<11>{
  \draw[line width=1.0499999999999998pt, brown!40!red] (U0) -- (W1);
  \draw[line width=1.0499999999999998pt, brown!40!red] (U0) -- (W2);
}
\only<12>{
  \draw[line width=1.0499999999999998pt, brown!40!red] (U1) -- (W1);
  \draw[line width=1.0499999999999998pt, brown!40!red] (U1) -- (W2);
}
\uncover<13>{
  \node[draw, thick, fill=yellow!20, rounded corners, align=center, text width=6.0cm, overlay] at (2.25, -1.5) {{In the example, we conclude that $10.4 \leq 12$, which is true. For bigger values of $z$ this would fail, leading to contradiction and therefore upper bounding $z$. In fact, z=14 already fails! }};
}
\end{pgfonlayer}\end{tikzpicture}\label{fig:kst}
            \end{figure}
        \end{column}

        \begin{column}{.65\textwidth}

            \begin{itemize}

                \item<1-> \textbf{Hypothesis:}
                $H=(U, W; E)$ is a $K(s, t)$-free bipartite $k$-graph with
                $z = z(u, w; s, t)$ edges, where $|U| = u$ and $|W| = w$.

                \item<3-> \textbf{Counting Stars:}
                For each $x \in U$, there are $\binom{d_H(x)}{t}$
                sets $T \subset W$ of $t$ neighbors of $x$.

                \item<9-> \textbf{Averaging:} By a convexity argument,
                the number of stars is at least $u\binom{z/u}{t}$.

                \item<10-> \textbf{Bounding:} Because $H$ is $K(s, t)$-free,
                each set $T \subset W$ is the right component of at most $(s-1)$ stars.

                \item<13-> \textbf{Conclusion:} $u \binom{z/u}{t} \leq (s-1) \binom{w}{t}$,
                from which the theorem follows.
            \end{itemize}
        \end{column}
    \end{columns}

\end{frame}


% SLIDE 1: THEOREM AND STRATEGY
\begin{frame}
    \frametitle{Erd\H{o}s's Bound for Hypergraphs (1964)}

    \begin{theorem}[Erd\H{o}s '64]
        For integers $k \ge 2, t \ge 2$,
        $\ex{n}{\compoverset{k}{t}} = \bigO{n^{k - \frac{1}{t^{k-1}}}}.$
    \end{theorem}

    This generalizes the Kővari--Sós--Turán theorem to $k$-graphs.

    It follows from a similar bound on the corresponding generalized Zarankiewicz number,
    obtained by induction.

    Suppose that $H = (V_1, \dots, V_k; F)$ is a $k$-graph with $|W_i| = w$.
    Let $H$ have $z$ edges and no copy of $\compoverset{k}{t}$.

    We set up a bipartite $k$-graph $H' = (U, W; F')$ with
    \begin{align*}
        U &=  W_1 \times \dots \times W_{k-1} \\
        W &= W_k \\
        F' &= \{(X, y) \in U \times W \mid X \cup \{y\} \in F\}.
    \end{align*}

\end{frame}

\begin{frame}[fragile]
    \frametitle{Erd\H{o}s's Bound: Proof Sketch ($k=3, t=2$)}

    % The figure now takes up the full frame width for better legibility
    \begin{figure}
        \centering
        \scalebox{0.55}{
            % TikZ code for side-by-side dual proof sketch (v8 - simplified).
\begin{tikzpicture}[scale=0.8, every node/.style={transform shape, scale=0.8}]
% === BASE GRAPHS (H and H') ===
\uncover<1->{
\begin{scope}
\node at (-6, 3.5) {\huge$H$};
\coordinate (A1) at (-4.5, 6);
\coordinate (A2) at (-1.5, 8);
\coordinate (A3) at (1.5, 10);
\coordinate (B1) at (-4.5, 1);
\coordinate (B2) at (-1.5, -1);
\coordinate (B3) at (1.5, -3.2);
\coordinate (C1) at (6.5, 1.85);
\coordinate (C2) at (6.5, 6);
\coordinate (C3) at (7.5, 4.1);
\coordinate (R0) at (-2.3000000000000003, 3.0709999999999997);
\fill[gray!40] (R0) circle (2.0pt);
\draw[gray!40, line width=1.0pt] (R0) -- (A1);
\draw[gray!40, line width=1.0pt] (R0) -- (B1);
\draw[gray!40, line width=1.0pt] (R0) -- (C1);
\coordinate (R1) at (-2.3000000000000003, 4.15);
\fill[gray!40] (R1) circle (2.0pt);
\draw[gray!40, line width=1.0pt] (R1) -- (A1);
\draw[gray!40, line width=1.0pt] (R1) -- (B1);
\draw[gray!40, line width=1.0pt] (R1) -- (C2);
\coordinate (R2) at (-1.1000000000000005, 2.3309999999999995);
\fill[gray!40] (R2) circle (2.0pt);
\draw[gray!40, line width=1.0pt] (R2) -- (A1);
\draw[gray!40, line width=1.0pt] (R2) -- (B2);
\draw[gray!40, line width=1.0pt] (R2) -- (C1);
\coordinate (R3) at (-1.1000000000000005, 3.4099999999999997);
\fill[gray!40] (R3) circle (2.0pt);
\draw[gray!40, line width=1.0pt] (R3) -- (A1);
\draw[gray!40, line width=1.0pt] (R3) -- (B2);
\draw[gray!40, line width=1.0pt] (R3) -- (C2);
\coordinate (R4) at (-1.1000000000000005, 3.811);
\fill[gray!40] (R4) circle (2.0pt);
\draw[gray!40, line width=1.0pt] (R4) -- (A2);
\draw[gray!40, line width=1.0pt] (R4) -- (B1);
\draw[gray!40, line width=1.0pt] (R4) -- (C1);
\coordinate (R5) at (-1.1000000000000005, 4.890000000000001);
\fill[gray!40] (R5) circle (2.0pt);
\draw[gray!40, line width=1.0pt] (R5) -- (A2);
\draw[gray!40, line width=1.0pt] (R5) -- (B1);
\draw[gray!40, line width=1.0pt] (R5) -- (C2);
\coordinate (R6) at (0.09999999999999964, 3.0709999999999997);
\fill[gray!40] (R6) circle (2.0pt);
\draw[gray!40, line width=1.0pt] (R6) -- (A2);
\draw[gray!40, line width=1.0pt] (R6) -- (B2);
\draw[gray!40, line width=1.0pt] (R6) -- (C1);
\coordinate (R7) at (0.09999999999999964, 4.15);
\fill[gray!40] (R7) circle (2.0pt);
\draw[gray!40, line width=1.0pt] (R7) -- (A2);
\draw[gray!40, line width=1.0pt] (R7) -- (B2);
\draw[gray!40, line width=1.0pt] (R7) -- (C2);
\coordinate (R8) at (2.6999999999999997, 3.582);
\fill[gray!40] (R8) circle (2.0pt);
\draw[gray!40, line width=1.0pt] (R8) -- (A3);
\draw[gray!40, line width=1.0pt] (R8) -- (B3);
\draw[gray!40, line width=1.0pt] (R8) -- (C3);
\coordinate (R9) at (1.2999999999999998, 2.257);
\fill[gray!40] (R9) circle (2.0pt);
\draw[gray!40, line width=1.0pt] (R9) -- (A2);
\draw[gray!40, line width=1.0pt] (R9) -- (B3);
\draw[gray!40, line width=1.0pt] (R9) -- (C1);
\coordinate (R10) at (0.09999999999999987, 5.630000000000001);
\fill[gray!40] (R10) circle (2.0pt);
\draw[gray!40, line width=1.0pt] (R10) -- (A3);
\draw[gray!40, line width=1.0pt] (R10) -- (B1);
\draw[gray!40, line width=1.0pt] (R10) -- (C2);
\coordinate (R11) at (1.4999999999999996, 4.396);
\fill[gray!40] (R11) circle (2.0pt);
\draw[gray!40, line width=1.0pt] (R11) -- (A3);
\draw[gray!40, line width=1.0pt] (R11) -- (B2);
\draw[gray!40, line width=1.0pt] (R11) -- (C3);
\coordinate (R12) at (0.09999999999999987, 2.596);
\fill[gray!40] (R12) circle (2.0pt);
\draw[gray!40, line width=1.0pt] (R12) -- (A1);
\draw[gray!40, line width=1.0pt] (R12) -- (B3);
\draw[gray!40, line width=1.0pt] (R12) -- (C2);
\coordinate (R13) at (1.2999999999999998, 3.811);
\fill[gray!40] (R13) circle (2.0pt);
\draw[gray!40, line width=1.0pt] (R13) -- (A3);
\draw[gray!40, line width=1.0pt] (R13) -- (B2);
\draw[gray!40, line width=1.0pt] (R13) -- (C1);
\coordinate (R14) at (0.09999999999999987, 4.551);
\fill[gray!40] (R14) circle (2.0pt);
\draw[gray!40, line width=1.0pt] (R14) -- (A3);
\draw[gray!40, line width=1.0pt] (R14) -- (B1);
\draw[gray!40, line width=1.0pt] (R14) -- (C1);
\fill[black] (A1) circle (4.0pt) node[above=2pt, font=\Large] {$A_1$};
\fill[black] (A2) circle (4.0pt) node[above=2pt, font=\Large] {$A_2$};
\fill[black] (A3) circle (4.0pt) node[above=2pt, font=\Large] {$A_3$};
\fill[black] (B1) circle (4.0pt) node[right=2pt, font=\Large] {$B_1$};
\fill[black] (B2) circle (4.0pt) node[right=2pt, font=\Large] {$B_2$};
\fill[black] (B3) circle (4.0pt) node[right=2pt, font=\Large] {$B_3$};
\fill[black] (C1) circle (4.0pt) node[above=2pt, font=\Large] {$C_1$};
\fill[black] (C2) circle (4.0pt) node[above=2pt, font=\Large] {$C_2$};
\fill[black] (C3) circle (4.0pt) node[above=2pt, font=\Large] {$C_3$};
\end{scope}
\begin{scope}
\node at (21.5, 4) {\huge$H'$};
\coordinate (A1B1) at (13, 9.5);
\coordinate (A1B2) at (13, 8.0);
\coordinate (A1B3) at (13, 6.5);
\coordinate (A2B1) at (13, 5.0);
\coordinate (A2B2) at (13, 3.5);
\coordinate (A2B3) at (13, 2.0);
\coordinate (A3B1) at (13, 0.5);
\coordinate (A3B2) at (13, -1.0);
\coordinate (A3B3) at (13, -2.5);
\coordinate (C1HP) at (19, 7);
\coordinate (C2HP) at (19, 4);
\coordinate (C3HP) at (19, 1);
\node[anchor=south] at (13, 10) { \huge $U = V_1 \times V_2$ }; 
\node[anchor=south] at (19, 8.5) { \huge $W = V_3$};
\draw[line width=0.5pt, black!30] (A1B1) -- (C1HP);
\draw[line width=0.5pt, black!30] (A1B1) -- (C2HP);
\draw[line width=0.5pt, black!30] (A1B2) -- (C1HP);
\draw[line width=0.5pt, black!30] (A1B2) -- (C2HP);
\draw[line width=0.5pt, black!30] (A1B3) -- (C2HP);
\draw[line width=0.5pt, black!30] (A2B1) -- (C1HP);
\draw[line width=0.5pt, black!30] (A2B1) -- (C2HP);
\draw[line width=0.5pt, black!30] (A2B2) -- (C1HP);
\draw[line width=0.5pt, black!30] (A2B2) -- (C2HP);
\draw[line width=0.5pt, black!30] (A2B3) -- (C1HP);
\draw[line width=0.5pt, black!30] (A3B1) -- (C1HP);
\draw[line width=0.5pt, black!30] (A3B1) -- (C2HP);
\draw[line width=0.5pt, black!30] (A3B2) -- (C1HP);
\draw[line width=0.5pt, black!30] (A3B2) -- (C3HP);
\draw[line width=0.5pt, black!30] (A3B3) -- (C3HP);
\fill[black] (A1B1) circle (4.0pt) node[anchor=east, font=\Large, xshift=-4pt] {$(A_1,B_1)$};
\fill[black] (A1B2) circle (4.0pt) node[anchor=east, font=\Large, xshift=-4pt] {$(A_1,B_2)$};
\fill[black] (A1B3) circle (4.0pt) node[anchor=east, font=\Large, xshift=-4pt] {$(A_1,B_3)$};
\fill[black] (A2B1) circle (4.0pt) node[anchor=east, font=\Large, xshift=-4pt] {$(A_2,B_1)$};
\fill[black] (A2B2) circle (4.0pt) node[anchor=east, font=\Large, xshift=-4pt] {$(A_2,B_2)$};
\fill[black] (A2B3) circle (4.0pt) node[anchor=east, font=\Large, xshift=-4pt] {$(A_2,B_3)$};
\fill[black] (A3B1) circle (4.0pt) node[anchor=east, font=\Large, xshift=-4pt] {$(A_3,B_1)$};
\fill[black] (A3B2) circle (4.0pt) node[anchor=east, font=\Large, xshift=-4pt] {$(A_3,B_2)$};
\fill[black] (A3B3) circle (4.0pt) node[anchor=east, font=\Large, xshift=-4pt] {$(A_3,B_3)$};
\fill[black] (C1HP) circle (4.0pt) node[anchor=west, font=\Large, xshift=4pt] {$C_1$};
\fill[black] (C2HP) circle (4.0pt) node[anchor=west, font=\Large, xshift=4pt] {$C_2$};
\fill[black] (C3HP) circle (4.0pt) node[anchor=west, font=\Large, xshift=4pt] {$C_3$};
\end{scope}
}

% --- Link for C1 (Slide 2) ---
\only<2>{
\fill[blue] (R0) circle (2.0pt);
\draw[blue, line width=1.5pt] (R0) -- (A1);
\draw[blue, line width=1.5pt] (R0) -- (B1);
\draw[blue, line width=1.5pt] (R0) -- (C1);
\fill[blue] (R2) circle (2.0pt);
\draw[blue, line width=1.5pt] (R2) -- (A1);
\draw[blue, line width=1.5pt] (R2) -- (B2);
\draw[blue, line width=1.5pt] (R2) -- (C1);
\fill[blue] (R4) circle (2.0pt);
\draw[blue, line width=1.5pt] (R4) -- (A2);
\draw[blue, line width=1.5pt] (R4) -- (B1);
\draw[blue, line width=1.5pt] (R4) -- (C1);
\fill[blue] (R6) circle (2.0pt);
\draw[blue, line width=1.5pt] (R6) -- (A2);
\draw[blue, line width=1.5pt] (R6) -- (B2);
\draw[blue, line width=1.5pt] (R6) -- (C1);
\fill[blue] (R9) circle (2.0pt);
\draw[blue, line width=1.5pt] (R9) -- (A2);
\draw[blue, line width=1.5pt] (R9) -- (B3);
\draw[blue, line width=1.5pt] (R9) -- (C1);
\fill[blue] (R13) circle (2.0pt);
\draw[blue, line width=1.5pt] (R13) -- (A3);
\draw[blue, line width=1.5pt] (R13) -- (B2);
\draw[blue, line width=1.5pt] (R13) -- (C1);
\fill[blue] (R14) circle (2.0pt);
\draw[blue, line width=1.5pt] (R14) -- (A3);
\draw[blue, line width=1.5pt] (R14) -- (B1);
\draw[blue, line width=1.5pt] (R14) -- (C1);
\fill[blue] (C1) circle (4.5pt);
\begin{scope}
\fill[blue] (C1HP) circle (4.5pt);
\draw[line width=1.3pt, blue] (A1B1) -- (C1HP);
\fill[blue] (A1B1) circle (4.5pt);
\draw[line width=1.3pt, blue] (A1B2) -- (C1HP);
\fill[blue] (A1B2) circle (4.5pt);
\draw[line width=1.3pt, blue] (A2B1) -- (C1HP);
\fill[blue] (A2B1) circle (4.5pt);
\draw[line width=1.3pt, blue] (A2B2) -- (C1HP);
\fill[blue] (A2B2) circle (4.5pt);
\draw[line width=1.3pt, blue] (A2B3) -- (C1HP);
\fill[blue] (A2B3) circle (4.5pt);
\draw[line width=1.3pt, blue] (A3B1) -- (C1HP);
\fill[blue] (A3B1) circle (4.5pt);
\draw[line width=1.3pt, blue] (A3B2) -- (C1HP);
\fill[blue] (A3B2) circle (4.5pt);
\end{scope}
}

% --- 2D Graph for L(C1) (Slide 3) ---
\only<3>{
\draw[line width=1.5pt, blue, bend left=20] (A1) to (B1);
\draw[line width=1.5pt, blue, bend left=20] (A3) to (B1);
\draw[line width=1.5pt, blue, bend left=20] (A2) to (B3);
\draw[line width=1.5pt, blue, bend left=20] (A2) to (B2);
\draw[line width=1.5pt, blue, bend left=20] (A1) to (B2);
\draw[line width=1.5pt, blue, bend left=20] (A2) to (B1);
\draw[line width=1.5pt, blue, bend left=20] (A3) to (B2);
\fill[blue] (A3) circle (4.5pt);
\fill[blue] (A1) circle (4.5pt);
\fill[blue] (A2) circle (4.5pt);
\fill[blue] (B2) circle (4.5pt);
\fill[blue] (B1) circle (4.5pt);
\fill[blue] (B3) circle (4.5pt);
\begin{scope}
\node[draw, thick, fill=blue!20, rounded corners] at (0, -2) {2-graph $L_H(C1)$};
\end{scope}
\begin{scope}
\fill[blue] (C1HP) circle (4.5pt);
\draw[line width=1.3pt, blue] (A1B1) -- (C1HP);
\fill[blue] (A1B1) circle (4.5pt);
\draw[line width=1.3pt, blue] (A1B2) -- (C1HP);
\fill[blue] (A1B2) circle (4.5pt);
\draw[line width=1.3pt, blue] (A2B1) -- (C1HP);
\fill[blue] (A2B1) circle (4.5pt);
\draw[line width=1.3pt, blue] (A2B2) -- (C1HP);
\fill[blue] (A2B2) circle (4.5pt);
\draw[line width=1.3pt, blue] (A2B3) -- (C1HP);
\fill[blue] (A2B3) circle (4.5pt);
\draw[line width=1.3pt, blue] (A3B1) -- (C1HP);
\fill[blue] (A3B1) circle (4.5pt);
\draw[line width=1.3pt, blue] (A3B2) -- (C1HP);
\fill[blue] (A3B2) circle (4.5pt);
\end{scope}
}

% --- Link for C2 (Slide 4) ---
\only<4>{
\fill[orange] (R1) circle (2.0pt);
\draw[orange, line width=1.5pt] (R1) -- (A1);
\draw[orange, line width=1.5pt] (R1) -- (B1);
\draw[orange, line width=1.5pt] (R1) -- (C2);
\fill[orange] (R3) circle (2.0pt);
\draw[orange, line width=1.5pt] (R3) -- (A1);
\draw[orange, line width=1.5pt] (R3) -- (B2);
\draw[orange, line width=1.5pt] (R3) -- (C2);
\fill[orange] (R5) circle (2.0pt);
\draw[orange, line width=1.5pt] (R5) -- (A2);
\draw[orange, line width=1.5pt] (R5) -- (B1);
\draw[orange, line width=1.5pt] (R5) -- (C2);
\fill[orange] (R7) circle (2.0pt);
\draw[orange, line width=1.5pt] (R7) -- (A2);
\draw[orange, line width=1.5pt] (R7) -- (B2);
\draw[orange, line width=1.5pt] (R7) -- (C2);
\fill[orange] (R10) circle (2.0pt);
\draw[orange, line width=1.5pt] (R10) -- (A3);
\draw[orange, line width=1.5pt] (R10) -- (B1);
\draw[orange, line width=1.5pt] (R10) -- (C2);
\fill[orange] (R12) circle (2.0pt);
\draw[orange, line width=1.5pt] (R12) -- (A1);
\draw[orange, line width=1.5pt] (R12) -- (B3);
\draw[orange, line width=1.5pt] (R12) -- (C2);
\fill[orange] (C2) circle (4.5pt);
\begin{scope}
\fill[orange] (C2HP) circle (4.5pt);
\draw[line width=1.3pt, orange] (A1B1) -- (C2HP);
\fill[orange] (A1B1) circle (4.5pt);
\draw[line width=1.3pt, orange] (A1B2) -- (C2HP);
\fill[orange] (A1B2) circle (4.5pt);
\draw[line width=1.3pt, orange] (A1B3) -- (C2HP);
\fill[orange] (A1B3) circle (4.5pt);
\draw[line width=1.3pt, orange] (A2B1) -- (C2HP);
\fill[orange] (A2B1) circle (4.5pt);
\draw[line width=1.3pt, orange] (A2B2) -- (C2HP);
\fill[orange] (A2B2) circle (4.5pt);
\draw[line width=1.3pt, orange] (A3B1) -- (C2HP);
\fill[orange] (A3B1) circle (4.5pt);
\end{scope}
}

% --- 2D Graph for L(C2) (Slide 5) ---
\only<5>{
\draw[line width=1.5pt, orange, bend left=20] (A1) to (B1);
\draw[line width=1.5pt, orange, bend left=20] (A3) to (B1);
\draw[line width=1.5pt, orange, bend left=20] (A1) to (B3);
\draw[line width=1.5pt, orange, bend left=20] (A2) to (B2);
\draw[line width=1.5pt, orange, bend left=20] (A1) to (B2);
\draw[line width=1.5pt, orange, bend left=20] (A2) to (B1);
\fill[orange] (A3) circle (4.5pt);
\fill[orange] (A1) circle (4.5pt);
\fill[orange] (A2) circle (4.5pt);
\fill[orange] (B2) circle (4.5pt);
\fill[orange] (B1) circle (4.5pt);
\fill[orange] (B3) circle (4.5pt);
\begin{scope}
\node[draw, thick, fill=orange!20, rounded corners] at (0, -2) {2-graph $L_H(C2)$};
\end{scope}
\begin{scope}
\fill[orange] (C2HP) circle (4.5pt);
\draw[line width=1.3pt, orange] (A1B1) -- (C2HP);
\fill[orange] (A1B1) circle (4.5pt);
\draw[line width=1.3pt, orange] (A1B2) -- (C2HP);
\fill[orange] (A1B2) circle (4.5pt);
\draw[line width=1.3pt, orange] (A1B3) -- (C2HP);
\fill[orange] (A1B3) circle (4.5pt);
\draw[line width=1.3pt, orange] (A2B1) -- (C2HP);
\fill[orange] (A2B1) circle (4.5pt);
\draw[line width=1.3pt, orange] (A2B2) -- (C2HP);
\fill[orange] (A2B2) circle (4.5pt);
\draw[line width=1.3pt, orange] (A3B1) -- (C2HP);
\fill[orange] (A3B1) circle (4.5pt);
\end{scope}
}

% --- Link for C3 (Slide 6) ---
\only<6>{
\fill[green!50!black] (R8) circle (2.0pt);
\draw[green!50!black, line width=1.5pt] (R8) -- (A3);
\draw[green!50!black, line width=1.5pt] (R8) -- (B3);
\draw[green!50!black, line width=1.5pt] (R8) -- (C3);
\fill[green!50!black] (R11) circle (2.0pt);
\draw[green!50!black, line width=1.5pt] (R11) -- (A3);
\draw[green!50!black, line width=1.5pt] (R11) -- (B2);
\draw[green!50!black, line width=1.5pt] (R11) -- (C3);
\fill[green!50!black] (C3) circle (4.5pt);
\begin{scope}
\fill[green!50!black] (C3HP) circle (4.5pt);
\draw[line width=1.3pt, green!50!black] (A3B2) -- (C3HP);
\fill[green!50!black] (A3B2) circle (4.5pt);
\draw[line width=1.3pt, green!50!black] (A3B3) -- (C3HP);
\fill[green!50!black] (A3B3) circle (4.5pt);
\end{scope}
}

% --- 2D Graph for L(C3) (Slide 7) ---
\only<7>{
\draw[line width=1.5pt, green!50!black, bend left=20] (A3) to (B2);
\draw[line width=1.5pt, green!50!black, bend left=20] (A3) to (B3);
\fill[green!50!black] (A3) circle (4.5pt);
\fill[green!50!black] (B2) circle (4.5pt);
\fill[green!50!black] (B3) circle (4.5pt);
\begin{scope}
\node[draw, thick, fill=green!50!black!20, rounded corners] at (0, -2) {2-graph $L_H(C3)$};
\end{scope}
\begin{scope}
\fill[green!50!black] (C3HP) circle (4.5pt);
\draw[line width=1.3pt, green!50!black] (A3B2) -- (C3HP);
\fill[green!50!black] (A3B2) circle (4.5pt);
\draw[line width=1.3pt, green!50!black] (A3B3) -- (C3HP);
\fill[green!50!black] (A3B3) circle (4.5pt);
\end{scope}
}

% --- Common Link for T={C1, C2} (Slide 8) ---
\only<8>{
\fill[teal] (R0) circle (2.0pt);
\draw[teal, line width=1.5pt] (R0) -- (A1);
\draw[teal, line width=1.5pt] (R0) -- (B1);
\draw[teal, line width=1.5pt] (R0) -- (C1);
\fill[teal] (R1) circle (2.0pt);
\draw[teal, line width=1.5pt] (R1) -- (A1);
\draw[teal, line width=1.5pt] (R1) -- (B1);
\draw[teal, line width=1.5pt] (R1) -- (C2);
\fill[teal] (R2) circle (2.0pt);
\draw[teal, line width=1.5pt] (R2) -- (A1);
\draw[teal, line width=1.5pt] (R2) -- (B2);
\draw[teal, line width=1.5pt] (R2) -- (C1);
\fill[teal] (R3) circle (2.0pt);
\draw[teal, line width=1.5pt] (R3) -- (A1);
\draw[teal, line width=1.5pt] (R3) -- (B2);
\draw[teal, line width=1.5pt] (R3) -- (C2);
\fill[teal] (R4) circle (2.0pt);
\draw[teal, line width=1.5pt] (R4) -- (A2);
\draw[teal, line width=1.5pt] (R4) -- (B1);
\draw[teal, line width=1.5pt] (R4) -- (C1);
\fill[teal] (R5) circle (2.0pt);
\draw[teal, line width=1.5pt] (R5) -- (A2);
\draw[teal, line width=1.5pt] (R5) -- (B1);
\draw[teal, line width=1.5pt] (R5) -- (C2);
\fill[teal] (R6) circle (2.0pt);
\draw[teal, line width=1.5pt] (R6) -- (A2);
\draw[teal, line width=1.5pt] (R6) -- (B2);
\draw[teal, line width=1.5pt] (R6) -- (C1);
\fill[teal] (R7) circle (2.0pt);
\draw[teal, line width=1.5pt] (R7) -- (A2);
\draw[teal, line width=1.5pt] (R7) -- (B2);
\draw[teal, line width=1.5pt] (R7) -- (C2);
\fill[teal] (R10) circle (2.0pt);
\draw[teal, line width=1.5pt] (R10) -- (A3);
\draw[teal, line width=1.5pt] (R10) -- (B1);
\draw[teal, line width=1.5pt] (R10) -- (C2);
\fill[teal] (R14) circle (2.0pt);
\draw[teal, line width=1.5pt] (R14) -- (A3);
\draw[teal, line width=1.5pt] (R14) -- (B1);
\draw[teal, line width=1.5pt] (R14) -- (C1);
\fill[teal] (C1) circle (4.5pt);
\fill[teal] (C2) circle (4.5pt);
\fill[teal] (A1) circle (4.5pt);
\fill[teal] (B1) circle (4.5pt);
\fill[teal] (A1) circle (4.5pt);
\fill[teal] (B2) circle (4.5pt);
\fill[teal] (A2) circle (4.5pt);
\fill[teal] (B1) circle (4.5pt);
\fill[teal] (A2) circle (4.5pt);
\fill[teal] (B2) circle (4.5pt);
\fill[teal] (A3) circle (4.5pt);
\fill[teal] (B1) circle (4.5pt);
\begin{scope}
\fill[teal] (A1B1) circle (4.5pt);
\draw[line width=1.3pt, teal] (A1B1) -- (C1HP);
\draw[line width=1.3pt, teal] (A1B1) -- (C2HP);
\fill[teal] (A1B2) circle (4.5pt);
\draw[line width=1.3pt, teal] (A1B2) -- (C1HP);
\draw[line width=1.3pt, teal] (A1B2) -- (C2HP);
\fill[teal] (A2B1) circle (4.5pt);
\draw[line width=1.3pt, teal] (A2B1) -- (C1HP);
\draw[line width=1.3pt, teal] (A2B1) -- (C2HP);
\fill[teal] (A2B2) circle (4.5pt);
\draw[line width=1.3pt, teal] (A2B2) -- (C1HP);
\draw[line width=1.3pt, teal] (A2B2) -- (C2HP);
\fill[teal] (A3B1) circle (4.5pt);
\draw[line width=1.3pt, teal] (A3B1) -- (C1HP);
\draw[line width=1.3pt, teal] (A3B1) -- (C2HP);
\fill[teal] (C1HP) circle (4.5pt);
\fill[teal] (C2HP) circle (4.5pt);
\end{scope}
}

% --- 2D Graph for L(T) (Slide 9) ---
\only<9>{
\draw[line width=1.5pt, teal, bend left=20] (A1) to (B1);
\draw[line width=1.5pt, teal, bend left=20] (A3) to (B1);
\draw[line width=1.5pt, teal, bend left=20] (A2) to (B2);
\draw[line width=1.5pt, teal, bend left=20] (A1) to (B2);
\draw[line width=1.5pt, teal, bend left=20] (A2) to (B1);
\fill[teal] (A3) circle (4.5pt);
\fill[teal] (A1) circle (4.5pt);
\fill[teal] (A2) circle (4.5pt);
\fill[teal] (B2) circle (4.5pt);
\fill[teal] (B1) circle (4.5pt);
\node[draw, thick, fill=teal!20, rounded corners] at (0, -2) {2-graph $L_H(T)$};
\begin{scope}
\fill[teal] (A1B1) circle (4.5pt);
\draw[line width=1.3pt, teal] (A1B1) -- (C1HP);
\draw[line width=1.3pt, teal] (A1B1) -- (C2HP);
\fill[teal] (A1B2) circle (4.5pt);
\draw[line width=1.3pt, teal] (A1B2) -- (C1HP);
\draw[line width=1.3pt, teal] (A1B2) -- (C2HP);
\fill[teal] (A2B1) circle (4.5pt);
\draw[line width=1.3pt, teal] (A2B1) -- (C1HP);
\draw[line width=1.3pt, teal] (A2B1) -- (C2HP);
\fill[teal] (A2B2) circle (4.5pt);
\draw[line width=1.3pt, teal] (A2B2) -- (C1HP);
\draw[line width=1.3pt, teal] (A2B2) -- (C2HP);
\fill[teal] (A3B1) circle (4.5pt);
\draw[line width=1.3pt, teal] (A3B1) -- (C1HP);
\draw[line width=1.3pt, teal] (A3B1) -- (C2HP);
\fill[teal] (C1HP) circle (4.5pt);
\fill[teal] (C2HP) circle (4.5pt);
\end{scope}
}

% --- Found K(2,2) in L(T) (Slide 10) ---
\only<10>{
\draw[line width=1.5pt, teal!40, bend left=20] (A1) to (B1);
\draw[line width=1.5pt, teal!40, bend left=20] (A3) to (B1);
\draw[line width=1.5pt, teal!40, bend left=20] (A2) to (B2);
\draw[line width=1.5pt, teal!40, bend left=20] (A1) to (B2);
\draw[line width=1.5pt, teal!40, bend left=20] (A2) to (B1);
\draw[line width=2pt, purple, bend left=20] (A1) to (B1);
\draw[line width=2pt, purple, bend left=20] (A1) to (B2);
\draw[line width=2pt, purple, bend left=20] (A2) to (B1);
\draw[line width=2pt, purple, bend left=20] (A2) to (B2);
\fill[purple] (A1) circle (4.5pt);
\fill[purple] (A2) circle (4.5pt);
\fill[purple] (B1) circle (4.5pt);
\fill[purple] (B2) circle (4.5pt);
\node[draw, thick, fill=purple!20, rounded corners] at (0, -2) {$K(2,2) \subset L_H(T)$};
\begin{scope}
\fill[teal] (A3B1) circle (4.5pt);
\draw[line width=1.3pt, teal] (A3B1) -- (C1HP);
\draw[line width=1.3pt, teal] (A3B1) -- (C2HP);
\fill[purple] (A1B1) circle (4.5pt);
\draw[line width=1.3pt, purple] (A1B1) -- (C1HP);
\draw[line width=1.3pt, purple] (A1B1) -- (C2HP);
\fill[purple] (A1B2) circle (4.5pt);
\draw[line width=1.3pt, purple] (A1B2) -- (C1HP);
\draw[line width=1.3pt, purple] (A1B2) -- (C2HP);
\fill[purple] (A2B1) circle (4.5pt);
\draw[line width=1.3pt, purple] (A2B1) -- (C1HP);
\draw[line width=1.3pt, purple] (A2B1) -- (C2HP);
\fill[purple] (A2B2) circle (4.5pt);
\draw[line width=1.3pt, purple] (A2B2) -- (C1HP);
\draw[line width=1.3pt, purple] (A2B2) -- (C2HP);
\fill[purple] (C1HP) circle (4.5pt);
\fill[purple] (C2HP) circle (4.5pt);
\end{scope}
}

% --- FINAL K(2,2,2) in H (Slide 11) ---
\only<11>{
\fill[brown!60!red] (R0) circle (2.5pt);
\draw[brown!60!red, line width=1.8pt] (R0) -- (A1);
\draw[brown!60!red, line width=1.8pt] (R0) -- (B1);
\draw[brown!60!red, line width=1.8pt] (R0) -- (C1);
\fill[brown!60!red] (R1) circle (2.5pt);
\draw[brown!60!red, line width=1.8pt] (R1) -- (A1);
\draw[brown!60!red, line width=1.8pt] (R1) -- (B1);
\draw[brown!60!red, line width=1.8pt] (R1) -- (C2);
\fill[brown!60!red] (R2) circle (2.5pt);
\draw[brown!60!red, line width=1.8pt] (R2) -- (A1);
\draw[brown!60!red, line width=1.8pt] (R2) -- (B2);
\draw[brown!60!red, line width=1.8pt] (R2) -- (C1);
\fill[brown!60!red] (R3) circle (2.5pt);
\draw[brown!60!red, line width=1.8pt] (R3) -- (A1);
\draw[brown!60!red, line width=1.8pt] (R3) -- (B2);
\draw[brown!60!red, line width=1.8pt] (R3) -- (C2);
\fill[brown!60!red] (R4) circle (2.5pt);
\draw[brown!60!red, line width=1.8pt] (R4) -- (A2);
\draw[brown!60!red, line width=1.8pt] (R4) -- (B1);
\draw[brown!60!red, line width=1.8pt] (R4) -- (C1);
\fill[brown!60!red] (R5) circle (2.5pt);
\draw[brown!60!red, line width=1.8pt] (R5) -- (A2);
\draw[brown!60!red, line width=1.8pt] (R5) -- (B1);
\draw[brown!60!red, line width=1.8pt] (R5) -- (C2);
\fill[brown!60!red] (R6) circle (2.5pt);
\draw[brown!60!red, line width=1.8pt] (R6) -- (A2);
\draw[brown!60!red, line width=1.8pt] (R6) -- (B2);
\draw[brown!60!red, line width=1.8pt] (R6) -- (C1);
\fill[brown!60!red] (R7) circle (2.5pt);
\draw[brown!60!red, line width=1.8pt] (R7) -- (A2);
\draw[brown!60!red, line width=1.8pt] (R7) -- (B2);
\draw[brown!60!red, line width=1.8pt] (R7) -- (C2);
\fill[brown!60!red] (A1) circle (4.5pt);
\fill[brown!60!red] (A2) circle (4.5pt);
\fill[brown!60!red] (B1) circle (4.5pt);
\fill[brown!60!red] (B2) circle (4.5pt);
\fill[brown!60!red] (C1) circle (4.5pt);
\fill[brown!60!red] (C2) circle (4.5pt);
\node[draw, thick, fill=brown!60!red!20, rounded corners, align=center] at (0, -2) {$K(2,2,2)$ found in $H$};
\begin{scope}
\fill[brown!60!red] (A1B1) circle (4.5pt);
\draw[line width=1.3pt, brown!60!red] (A1B1) -- (C1HP);
\draw[line width=1.3pt, brown!60!red] (A1B1) -- (C2HP);
\fill[brown!60!red] (A1B2) circle (4.5pt);
\draw[line width=1.3pt, brown!60!red] (A1B2) -- (C1HP);
\draw[line width=1.3pt, brown!60!red] (A1B2) -- (C2HP);
\fill[brown!60!red] (A2B1) circle (4.5pt);
\draw[line width=1.3pt, brown!60!red] (A2B1) -- (C1HP);
\draw[line width=1.3pt, brown!60!red] (A2B1) -- (C2HP);
\fill[brown!60!red] (A2B2) circle (4.5pt);
\draw[line width=1.3pt, brown!60!red] (A2B2) -- (C1HP);
\draw[line width=1.3pt, brown!60!red] (A2B2) -- (C2HP);
\fill[brown!60!red] (C1HP) circle (4.5pt);
\fill[brown!60!red] (C2HP) circle (4.5pt);
\end{scope}
}
\end{tikzpicture}
        }\label{fig:erdos64_sketch}
    \end{figure}

\end{frame}

\end{document}