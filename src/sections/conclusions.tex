\section{Conclusions and Future Work}\label{sec:conclusions}

This thesis has focused on the algorithmic aspects of finding $k$-partite subgraphs in $k$-uniform hypergraphs,
a problem central to degenerate Turán theory.
We have presented a deterministic,
polynomial-time algorithm (\Cref{alg:kpartite}) that, given a $k$-graph $G$ on $n$ vertices with $m$ edges,
finds a complete balanced $k$-partite $k$-subgraph $\compoverset{k}{t}$.
The part size $t$ achieved, given by Equation~\eqref{eq:t} as $t \approx (\log n / \log(1/d))^{1/(k-1)}$ where $d=m/n^k$,
closely matches the parameters implicit in the non-constructive existence proofs of Erd\H{o}s for such structures.
This provides an efficient, constructive counterpart to these classical results,
demonstrating that these subgraphs can indeed be located algorithmically within polynomial time.
The recursive approach, which reduces the uniformity $k$ by analyzing appropriately defined link graphs,
generalizes previous work for the $k=2$
case by Mubayi and Turán.

Looking ahead, this work opens several promising avenues for future research.
One natural direction is the refinement of the existing algorithm and its analysis.
While the proofs of correctness for~\Cref{alg:kpartite}
(particularly~\Cref{lm:many_edges,lm:return,lm:t_prime,lm:base_case}) establish the polynomial runtime and the asymptotic
nature of $t$, some of the underlying inequalities are not tight.
A more meticulous analysis could potentially yield sharper constants in the definition of $t(n,d,k)$
or relax the minimum density requirements detailed in~\Cref{rm:min_d}.
Such improvements could enhance the algorithm's practical significance by making it applicable to smaller hypergraphs
or by guaranteeing larger $k$-partite structures for a given density.
Complementing these theoretical refinements, the implementation and experimental evaluation of~\Cref{alg:kpartite}
on synthetic or real-world hypergraph datasets would be highly valuable.
This practical work could help identify actual performance bottlenecks and compare its empirical findings with theoretical
guarantees, especially concerning the constants involved in the calculation of $t$.

Beyond optimizing the current framework, a significant extension would be to generalize the types
of structures our algorithm can find.
The current algorithm targets balanced complete $k$-partite $k$-graphs, $\compoverset{k}{t}$.
It would be interesting to adapt the algorithmic strategy to find unbalanced complete $k$-partite $k$-graphs, $
\compdots{t_1}{t_k}$, where the part sizes $t_i$ may differ yet still grow with $n$.
Theoretical bounds for this unbalanced case
exist~\cite{carvajal2024canonical}, and an algorithm finding these more general structures,
perhaps by fixing approximate ratios between part sizes, would be a valuable contribution.
Further generalizing, the presented algorithm is tailored to find blow-ups of a single edge.
A more ambitious goal would be to adapt this algorithmic framework to find $t_n$-blowups $G(t_n)$
of an arbitrary fixed $k$-graph $G$.
As discussed (\Cref{thm:quant-blowup}),
$k$-graphs with density $\pi(G) + \epsilon$ are known to contain $G(t_n)$ where $t_n = \delta (\log n)^{1/(|V(G)|-1)}$.
Developing a constructive method for these general blow-ups is a challenging but important problem.
For the specific case of $k=2$, it is known from Bollobás and Erd\H{o}s~\cite{bollobas1973structure}
that the optimal growth for $t_n$ is $\delta \log n$, which is better than the general bound if $|V(G)| > 2$.
It remains an open question whether their proof can be made constructive to yield a polynomial-time algorithm,
although the current algorithmic approach does not directly extend to this case.

Finally, the behavior of these Turán-type problems changes significantly when the density of the host hypergraph
is not merely a fixed constant but approaches the maximum possible density (i.e., $d \to 1/k!$).
In such extremely dense regimes, the guaranteed part size $t$ for $\compoverset{k}{t}$ can be much larger,
potentially polynomial in $n$ rather than logarithmic.
The current algorithm's efficiency relies on $t$ being relatively small.
Investigating whether our algorithmic approach can be modified or a new one developed to find these
much larger partite structures, even if it means potentially sacrificing strict polynomial-time complexity in $n$ when $t$
itself is large, is another intriguing area for exploration.