\section{Preliminaries}\label{sec:preliminaries}
In this section we introduce some basic definitions and results that will be used throughout the thesis.

\begin{definition}

    For an integer $k \geq 2$ a finite \emph{$k$-graph}
    is a tuple $G = (V, E)$ where $V$ is a finite set
    and $E \subseteq \binom{V}{k}$.
    We call the elements of $V \eqqcolon V(G)$ its \emph{vertices}
    and those of $E \eqqcolon E(G)$ its \emph{edges}.
\end{definition}

\begin{remark}
    If we let $k=2$ we recover the usual definition of a graph.
\end{remark}

% TODO define degrees, complete graph, etc.

\begin{definition}
    Let $G = (V, E)$ and $H = (W, F)$ be $k$-graphs.
    A \emph{homomorphism} from $G$ to $H$ is a map $f: V \to W$
    such that for every edge $e \in E$ the set $f(e) \coloneqq \{f(v) \mid v \in e\}$
    is an edge in $H$ (that is, $f(e) \in F$). If such a homomorphism exists
    and is injective, we say that $f$ is an embedding of $G$ on $H$
    and that $H$ contains $G$ as a subgraph.
    If, furthermore, $f^{-1}: \text{Im}(f) \to V$ is a homomorphism, we say that $f$
    is an \emph{induced} embedding and that $H$ contains $G$ as an \emph{induced}
    subgraph.
    We write $G \subseteq H$.
    If, in addition, $f$ is a bijection, we say that $f$ is an \emph{isomorphism}
    and that $G$ is \emph{isomorphic} to $H$.
    We write $G \cong H$.
\end{definition}

\begin{remark} % TODO maybe turn into proposition + proof?

    It is elementary to check that
    (induced) inclusion is an order relation and that
    isomorphism is an equivalence relation.
    Furthermore, isomorphism preserves (induced) inclusion.
    Therefore, we can talk about the (induced) subgraph
    condition up to isomorphism, both in the \emph{host} $k$-graph
    ($H$) and in the \emph{guest} $k$-graph ($G$).
\end{remark}

\begin{remark} \label{rem:change_vertices}
    Given a $k$-graph $G = (V, E)$ and a set $W$ satisfying $|V| = |W|$,
    we can define an edge set $E'$ on $W$ such that $G \cong (W, E')$
    by taking any bijection $f: V \to W$ and setting $E' = \{f(e) \mid e \in E\}$.
    This frees us, up to isomorphism, to change or reorder
    the vertices of a $k$-graph.
\end{remark}

\begin{proposition} \label{prop:extremal}
    Let $G = (V, E)$ be a $k$-graph with nonempty edge set and $n \geq |V|$ be an integer.
    Then there exists an integer $M_0 = ex(n, G) \in \left[ 0, \binom{n}{k}\right)$ such that
    the condition
    \[
        \text{``All $k$-graphs with $n$ vertices and $m$ edges contain $G$ as a subgraph''}
    \]
    is true for all $\binom{n}{k} \geq m > M_0$ and false for all $0 \leq m \leq M_0$.

    \begin{proof}
        Note that, if $M_0$ exists, clearly it is unique.
        Also, the condition is clearly false for $m = 0$ and
        true for $m = \binom{n}{k}$
        (the only graph $H$ with vertex set $W$, $|W|=n$ and $\binom{|V|}{k}$ vertices
        is the one having all $k$-sets of vertices so any injective map $f: V \to W$
        is an embedding of $G$ in $H$).
        We only need to show that if the condition is true for $m$ then it is true for
        all $m' \geq m$.
        Suppose it is true for $m$ and let $m' \geq m$.
        Let $H = (W, F)$ be a $k$-graph with $n$ vertices and $m'$ edges.
        We can just take $F' \subseteq F$ with $|F'| = m$.
        By hypothesis, the graph $H' = (W, F')$ contains $G$ as a subgraph,
        and the identity map in $W$ is an embedding of $H'$ in $H$:
        \[
            G \subseteq H' \subseteq H \implies G \subseteq H \qedhere
        \]
    \end{proof}

\end{proposition}

\begin{remark}
    We call $ex(n, G)$ the \emph{extremal number} of $G$.
    It is clearly invariant under isomorphism.
\end{remark}

\begin{definition}
    for an integer $p \geq k$, a $k$-graph $G = (V, E)$ is \emph{$p$-partite}
    if there exists a partition $V = V_1 \cup \dots \cup V_p$
    such that every edge $e \in E$ intersects every part $V_i$ in at most one vertex.
    We may write $G = (V_1, \dots, V_p; E)$ and say that
    $G$ is a partite $k$-graph on $V_1, \dots, V_p$.
\end{definition}

\begin{remark}
    If $p=k$, every edge intersects every part in exactly one vertex,
    so we can identify the edges with a subset of $ V_1 \times \dots \times V_k$.
\end{remark}

\begin{definition}
    A $k$-partite $k$-graph $G = (V_1, \dots, V_k; E)$ is \emph{complete}
    if every $k$-set of vertices $(v_1, \dots, v_k)$ with $v_i \in V_i$
    satisfies $\{v_1, \dots, v_k\} \in E$.
    We write $G = K(V_1, \dots, V_k)$.
\end{definition}

\begin{remark}
    $V_1, \dots, V_k$, $W_1, \dots, W_k$ are disjoint sets,
    and $|V_i| = |W_i| \eqqcolon a_i$ for all $i$ then it is elementary to check that
    \[K(V_1, \dots, V_k) \cong K(W_1, \dots, W_k)\]
    by a construction very similar to the one in Remark~\ref{rem:change_vertices}.
    This allows us to talk about \emph{the} complete $k$-partite $k$-graph on
    $a_1, \dots, a_k$ vertices, which we denote by $K(a_1, \dots, a_k)$.
    All $k$-partite $k$-graphs with part sizes $b_1 \leq a_1, \dots, b_k \leq a_k$
    are contained $K(a_1, \dots, a_k)$ as subgraphs.
    This lets us follow the exact same argument as in Proposition~\ref{prop:extremal}
    to define the following:
\end{remark}

\begin{definition}\label{def:zarankiewicz}
    let $0 < r_1 \leq n_1, \dots, 0 < r_k \leq n_k$ be integers.
    Then the \emph{Zarankiewicz number} $z(n_1, \dots, n_k; r_1, \dots, r_k)$
    is the largest integer $0 \leq z < n_1  \dots n_k$ for which there exists $k$-partite $k$-graph
    $H$ with part sizes $ |V_1| = n_1, \dots, |V_k| = n_k$ and $z$ edges
    such that no embedding $f$ of $K(W_1, \dots, W_k)$ with $|W_i| = r_i$ in it exists
    satisfying $f(W_i) \subseteq V_i$ for all $i$.
\end{definition}

The problem on finding the Zarankiewicz number was first posed by K. Zarankiewicz in 1951 for the
case of bipartite 2-graphs (that is, finding $z(m, n; s, t)$),
in terms of finding all-1 minors in a matrix.
An upper bound for it in the case $m=n, s=t$ was found by Kővari, Sós and Turán in~\cite{Kovari1954} in 1954.
This was generalized to arbitrary partite 2-graphs by C. Hyltén-Cavallius in~\cite{Hylten1958}
in 1958.
The result is stated and proved here for completeness:

\begin{theorem}
    Let $0 < m \leq s$ and $0 < n \leq t$ be integers. 
    Then 
    \[z(m, n; s, t) \leq (s - 1)^{1 / t}(n - t + 1)m^{1 - 1 / t} + (t - 1)m\]
    \begin{proof}
        Suppose that we have a graph $G = (V_1, V_2; E)$
        with $|V_1| = m$, $|V_2| = n$ and $|E| = z$ exceeding the bound.
        Let us consider the set
        \[
            P = \left\{ (x, Y) \in V_1 \times \binom{V_2}{t}
            \Big| \forall y \in Y: \{x, y\} \in E \right\}
        \]
        Counting on the first coordinate, and using Jensen's inequality, we get
        \[
            |P| = \nsum_{x \in V_1} \binom{d_G(x)}{t} \geq m \binom{z / m}{t}
        \]
        Because the function

        \[
            f(x) =
            \begin{cases}
                \binom{x}{t}, & \text{if } x \geq t - 1 \\
                0, & \text{otherwise}.
            \end{cases}
        \]
        Is convex and by our bound $z \geq (t-1)m \implies z/m \geq t - 1$.

        If we had $s$ different elements of $P$ with the same second coordinate $T$,
        they would all have different first coordinates (say $S = \{x_1, \dots, x_s\}$).
        But now, by definition of $P$, for all $a \in S, b \in T$, we have $\{a, b\} \in E$.
        This would mean that the inclusion map from $ S \cup T$ to $V_1 \cup V_2$ is an embedding of
        $K(s, t)$ in $G$, as described in Definition~\ref{def:zarankiewicz}.
        Supposing that this is not the case, by the pigeonhole principle, we have:
        \[
            |P| \leq (s - 1) \binom{n}{t}
        \]
        Putting the two inequalities together, we get:
        \[
            m \binom{z / m}{t} \leq (s - 1) \binom{n}{t}
        \]

        And the rest follows by algebraic manipulation. % TODO: finish proof


    \end{proof}

\end{theorem}


